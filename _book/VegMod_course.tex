\documentclass[12pt,oneside]{book}
\usepackage{lmodern}
\usepackage{amssymb,amsmath}
\usepackage{ifxetex,ifluatex}
\usepackage{fixltx2e} % provides \textsubscript
\ifnum 0\ifxetex 1\fi\ifluatex 1\fi=0 % if pdftex
  \usepackage[T1]{fontenc}
  \usepackage[utf8]{inputenc}
\else % if luatex or xelatex
  \ifxetex
    \usepackage{mathspec}
  \else
    \usepackage{fontspec}
  \fi
  \defaultfontfeatures{Ligatures=TeX,Scale=MatchLowercase}
\fi
% use upquote if available, for straight quotes in verbatim environments
\IfFileExists{upquote.sty}{\usepackage{upquote}}{}
% use microtype if available
\IfFileExists{microtype.sty}{%
\usepackage[]{microtype}
\UseMicrotypeSet[protrusion]{basicmath} % disable protrusion for tt fonts
}{}
\PassOptionsToPackage{hyphens}{url} % url is loaded by hyperref
\usepackage[unicode=true]{hyperref}
\hypersetup{
            pdftitle={Vegetation modelling},
            pdfauthor={Hans Verbeeck, Elizabeth Kearsley, Félicien Meunier, Marc Peaucelle},
            pdfborder={0 0 0},
            breaklinks=true}
\urlstyle{same}  % don't use monospace font for urls
\usepackage[left=3cm,right=3cm,top=2cm,bottom=2cm]{geometry}
\usepackage{natbib}
\bibliographystyle{apalike}
\usepackage{color}
\usepackage{fancyvrb}
\newcommand{\VerbBar}{|}
\newcommand{\VERB}{\Verb[commandchars=\\\{\}]}
\DefineVerbatimEnvironment{Highlighting}{Verbatim}{commandchars=\\\{\}}
% Add ',fontsize=\small' for more characters per line
\usepackage{framed}
\definecolor{shadecolor}{RGB}{248,248,248}
\newenvironment{Shaded}{\begin{snugshade}}{\end{snugshade}}
\newcommand{\KeywordTok}[1]{\textcolor[rgb]{0.13,0.29,0.53}{\textbf{#1}}}
\newcommand{\DataTypeTok}[1]{\textcolor[rgb]{0.13,0.29,0.53}{#1}}
\newcommand{\DecValTok}[1]{\textcolor[rgb]{0.00,0.00,0.81}{#1}}
\newcommand{\BaseNTok}[1]{\textcolor[rgb]{0.00,0.00,0.81}{#1}}
\newcommand{\FloatTok}[1]{\textcolor[rgb]{0.00,0.00,0.81}{#1}}
\newcommand{\ConstantTok}[1]{\textcolor[rgb]{0.00,0.00,0.00}{#1}}
\newcommand{\CharTok}[1]{\textcolor[rgb]{0.31,0.60,0.02}{#1}}
\newcommand{\SpecialCharTok}[1]{\textcolor[rgb]{0.00,0.00,0.00}{#1}}
\newcommand{\StringTok}[1]{\textcolor[rgb]{0.31,0.60,0.02}{#1}}
\newcommand{\VerbatimStringTok}[1]{\textcolor[rgb]{0.31,0.60,0.02}{#1}}
\newcommand{\SpecialStringTok}[1]{\textcolor[rgb]{0.31,0.60,0.02}{#1}}
\newcommand{\ImportTok}[1]{#1}
\newcommand{\CommentTok}[1]{\textcolor[rgb]{0.56,0.35,0.01}{\textit{#1}}}
\newcommand{\DocumentationTok}[1]{\textcolor[rgb]{0.56,0.35,0.01}{\textbf{\textit{#1}}}}
\newcommand{\AnnotationTok}[1]{\textcolor[rgb]{0.56,0.35,0.01}{\textbf{\textit{#1}}}}
\newcommand{\CommentVarTok}[1]{\textcolor[rgb]{0.56,0.35,0.01}{\textbf{\textit{#1}}}}
\newcommand{\OtherTok}[1]{\textcolor[rgb]{0.56,0.35,0.01}{#1}}
\newcommand{\FunctionTok}[1]{\textcolor[rgb]{0.00,0.00,0.00}{#1}}
\newcommand{\VariableTok}[1]{\textcolor[rgb]{0.00,0.00,0.00}{#1}}
\newcommand{\ControlFlowTok}[1]{\textcolor[rgb]{0.13,0.29,0.53}{\textbf{#1}}}
\newcommand{\OperatorTok}[1]{\textcolor[rgb]{0.81,0.36,0.00}{\textbf{#1}}}
\newcommand{\BuiltInTok}[1]{#1}
\newcommand{\ExtensionTok}[1]{#1}
\newcommand{\PreprocessorTok}[1]{\textcolor[rgb]{0.56,0.35,0.01}{\textit{#1}}}
\newcommand{\AttributeTok}[1]{\textcolor[rgb]{0.77,0.63,0.00}{#1}}
\newcommand{\RegionMarkerTok}[1]{#1}
\newcommand{\InformationTok}[1]{\textcolor[rgb]{0.56,0.35,0.01}{\textbf{\textit{#1}}}}
\newcommand{\WarningTok}[1]{\textcolor[rgb]{0.56,0.35,0.01}{\textbf{\textit{#1}}}}
\newcommand{\AlertTok}[1]{\textcolor[rgb]{0.94,0.16,0.16}{#1}}
\newcommand{\ErrorTok}[1]{\textcolor[rgb]{0.64,0.00,0.00}{\textbf{#1}}}
\newcommand{\NormalTok}[1]{#1}
\usepackage{longtable,booktabs}
% Fix footnotes in tables (requires footnote package)
\IfFileExists{footnote.sty}{\usepackage{footnote}\makesavenoteenv{long table}}{}
\usepackage{graphicx,grffile}
\makeatletter
\def\maxwidth{\ifdim\Gin@nat@width>\linewidth\linewidth\else\Gin@nat@width\fi}
\def\maxheight{\ifdim\Gin@nat@height>\textheight\textheight\else\Gin@nat@height\fi}
\makeatother
% Scale images if necessary, so that they will not overflow the page
% margins by default, and it is still possible to overwrite the defaults
% using explicit options in \includegraphics[width, height, ...]{}
\setkeys{Gin}{width=\maxwidth,height=\maxheight,keepaspectratio}
\IfFileExists{parskip.sty}{%
\usepackage{parskip}
}{% else
\setlength{\parindent}{0pt}
\setlength{\parskip}{6pt plus 2pt minus 1pt}
}
\setlength{\emergencystretch}{3em}  % prevent overfull lines
\providecommand{\tightlist}{%
  \setlength{\itemsep}{0pt}\setlength{\parskip}{0pt}}
\setcounter{secnumdepth}{5}
% Redefines (sub)paragraphs to behave more like sections
\ifx\paragraph\undefined\else
\let\oldparagraph\paragraph
\renewcommand{\paragraph}[1]{\oldparagraph{#1}\mbox{}}
\fi
\ifx\subparagraph\undefined\else
\let\oldsubparagraph\subparagraph
\renewcommand{\subparagraph}[1]{\oldsubparagraph{#1}\mbox{}}
\fi

% set default figure placement to htbp
\makeatletter
\def\fps@figure{htbp}
\makeatother

\usepackage{booktabs}
\usepackage{fancyhdr}

\AtBeginDocument{\let\maketitle\relax} % To relax 

% Remove page number on page parts
\usepackage{etoolbox}
\patchcmd{\part}{\thispagestyle{plain}}{\thispagestyle{empty}}{}{}

% Header and font 
\usepackage{fancyhdr}
\usepackage{float}

\usepackage{amsmath,color}

\pagestyle{fancy}
\fancyhf{} % sets both header and footer to nothing
\renewcommand{\headrulewidth}{0pt} % Remove line

\fancyhead[L,C,R]{} % Empty header
\fancyfoot[C]{\thepage} % Footer, center, page number
\fancyfoot[L,R]{} % Empty footer on left and right

\floatplacement{figure}{H}

\usepackage[makeroom]{cancel}
\usepackage{caption}

\title{Vegetation modelling}
\author{Hans Verbeeck, Elizabeth Kearsley, Félicien Meunier, Marc Peaucelle}
\date{2021-02-25}

\begin{document}
\maketitle

\newcommand{\plogo}{\fbox{$\mathcal{PL}$}} % Generic dummy publisher logo
\frontmatter


\begin{titlepage} % Suppresses headers and footers on the title page

	\centering % Centre everything on the title page
	
	\scshape % Use small caps for all text on the title page
	
	\vspace*{\baselineskip} % White space at the top of the page
	
	%------------------------------------------------
	%	Title
	%------------------------------------------------
	
	\vspace{12\baselineskip}
	
	\rule{\textwidth}{1.6pt}\vspace*{-\baselineskip}\vspace*{2pt} % Thick horizontal rule
	\rule{\textwidth}{0.4pt} % Thin horizontal rule
	
	\vspace{0.75\baselineskip} % Whitespace above the title
	
	{\LARGE Vegetation modelling\\} % Title
	
	\vspace{0.75\baselineskip} % Whitespace below the title
	
	\rule{\textwidth}{0.4pt}\vspace*{-\baselineskip}\vspace{3.2pt} % Thin horizontal rule
	\rule{\textwidth}{1.6pt} % Thick horizontal rule
	
	\vspace{2\baselineskip} % Whitespace after the title block
	
	%------------------------------------------------
	%	Subtitle
	%------------------------------------------------
	
	Syllabus % Subtitle or further description
	
	\vspace*{3\baselineskip} % Whitespace under the subtitle
	
	%------------------------------------------------
	%	Editor(s)
	%------------------------------------------------
	
	Written By
	
	\vspace{0.5\baselineskip} % Whitespace before the editors
	
	{\scshape Hans Verbeeck, Felicien Meunier, Marc Peaucelle \\} % Editor list
	
	\vspace{0.5\baselineskip} % Whitespace below the editor list
	
	\textit{Ghent University \\} % Editor affiliation
	
	\vfill % Whitespace between editor names and publisher logo
	
	%------------------------------------------------
	%	Publisher
	%------------------------------------------------
	
	%\plogo % Publisher logo
	
	\includegraphics[width = 50mm]{figures/UGhent2.png}
	
	\vspace{0.3\baselineskip} % Whitespace under the publisher logo
	
	2021 % Publication year
	
	%{\large publisher} % Publisher

\end{titlepage}

{
\setcounter{tocdepth}{1}
\tableofcontents
}
\mainmatter

\chapter{Introduction}\label{intro}

\section{The central role of vegetation in the Earth
system}\label{the-central-role-of-vegetation-in-the-earth-system}

Plants and vegetation play an essential role in the earth system. In
Figure \ref{fig:f1}, we have represented the planetary boundaries and
the significant environmental challenges we are facing. For some of
them, we are reaching the system's boundary, such as genetic diversity,
biogeochemical cycling, and climate change. Plants and forests
ecosystems are critical in some of these planetary problems. Therefore,
it is crucial to understand how vegetation reacts to environmental
problems (positive or negative feedback).

\begin{figure}

{\centering \includegraphics[width=0.6\linewidth]{figures/chap1/planetary_boundaries} 

}

\caption{The planetary boundaries (www.stockholmresilience.org)}\label{fig:f1}
\end{figure}

A more specific example is the global carbon budget. We are currently
facing this significant change in biogeochemical cycling due to the
rising fossil fuel emission over the last 150 years. Figure \ref{fig:f2}
represents the balance between carbon sources (fossil carbon and
land-use changes) and sinks (oceans, land, and atmosphere). The more we
emit, the more the Earth system is capturing, as shown in Fig.2. These
take us to a question, taken frequently by the models, about how long
these sinks will continue or not capture our increasing carbon
emissions. It is also important to highlight the high annual variability
of the land component being the most challenging component to predict.

\begin{figure}

{\centering \includegraphics[width=0.8\linewidth]{figures/chap1/carbon_budget} 

}

\caption{The global carbon budget (www.globalcarbonproject.org)}\label{fig:f2}
\end{figure}

\textbf{Climate Models} predicts the long-term weather variation and
average weather, how they will evolve. These models include the
atmosphere, land and ocean component (Fig.3, top). The original climate
models focus on biophysics: energy and water balances, predicting
precipitation, radiation and fluxes between the three components. More
recently, climate models have evolved into \textbf{Earth System Models}
(ESM). ESM have a more complex concept because it represents more
processes (Fig.3, bottom). Why? If you want to predict the end of the
century climate, we need to consider greenhouse gases. EMS are more
complex but also more realistic. Vegetation models are usually the land
component of ESM.

\begin{figure}

{\centering \includegraphics[width=0.8\linewidth]{figures/chap1/GCM_ESM} 

}

\caption{Scientific scope of (a) climate models and (b) earth system models. (Bonan 2019)}\label{fig:f3}
\end{figure}

Vegetation models are often the land component of an earth system model.

\begin{itemize}
\tightlist
\item
  simulates \textbf{energy fluxes}: radiation, evapotranspiration and
  sensible heat fluxes between land and the atmosphere. Depending on the
  vegetation type, the impacts are different.
\item
  simulates the \textbf{hydrology} and the \textbf{carbon cycle}.
\item
  slower processes like \textbf{vegetation dynamics}: the succession of
  forest or \textbf{land use} and \textbf{urbanization}.
\end{itemize}

\begin{figure}

{\centering \includegraphics[width=0.8\linewidth]{figures/chap1/cycles_bonan} 

}

\caption{Scientific scope of terrestrial biosphere model. (Bonan 2019)}\label{fig:f4}
\end{figure}

The biosphere acts as a coupler in models. The coupler is a system that
links different models. The vegetation provides the link between land
and atmosphere component.

\section{Why do we need modelling?}\label{why-do-we-need-modelling}

Modelling has proven to be a handy tool for:

\begin{itemize}
\tightlist
\item
  For \textbf{understanding}: we need good theoretical foundations
  (understand processes) to generalize in space and time.
\item
  For \textbf{prediction}, how vegetation responds to expected changes
  (temperature or CO2) to implant management strategies and policies.
\item
  For \textbf{data integration}: a framework to bring together multiple
  data sources.
\end{itemize}

\pagebreak

\section{Model types}\label{model-types}

\begin{center}
\captionof{table}{Continuum of terrestrial biosphere/ecosystem models. (Bonan 2019)}
\label{table:example}

\begin{center}\includegraphics[width=0.8\linewidth]{figures/chap1/table_model_types} \end{center}
\end{center}

How can we look at these models' types? We can divide these model types
into empirical models or process-based models. \textbf{Empirical models}
are based on data and correlations, not describing precisely describing
the biophysical processes --- \textbf{process-based models} describing
the biophysical processes and causal relations between the variables
(Table \ref{table:example}). Most models are hybrid.

\begin{center}
\captionof{table}{Continuum of process-based versus empirical models. (Adams et al. 2013)}
\label{table:empricial}

\begin{center}\includegraphics[width=0.8\linewidth]{figures/chap1/tables_PB_empirical} \end{center}
\end{center}

The model type depends on: - Purpose: will it be used for management
support, policy support, research. - Question: different people will be
interested in different questions (forester, ecologist,\ldots{}) -
Scale: models that are to be applied for local use can be much more
detailed than worldwide models because data gathering is much more
straightforward on a small scale.

\section{The history of vegetation
models}\label{the-history-of-vegetation-models}

The first vegetation models have emerged in 1960 and 1970. There are
some parallel evolutions during that time because of computers emerging,
making it possible to study more complex systems.

One of the first models was the \textbf{box models} (1960); these models
describe the flow of mass and energy thought boxes. These models still
exist in currently biogeochemical models, where arrows represent the
fluxes between the pools. In parallel, the \textbf{gap models} had
emerged. They were developed by forestry scientists using forest
inventories derived from growth, regeneration, and mortality in response
to environmental variables. In 1973, the first model was developed to
derived global net primary productivity, relating in an empirical way to
climate variables (temperature) with vegetation productivity---the first
attempt to make a global upscaling of a process. In the 1980s surged the
first \textbf{land surface models}. Land surface models start the models
where the other models start to integrate -- \textbf{Terrestrial
biosphere models} (see Figure \ref{fig:f7}), which are now the
state-of-the-art land components ESMs. Vegetation modelling is a very
interdisciplinary field because it involves knowledge of different
scientific fields, making it difficult to find a common terminology. EMS
are still not good to simulate vegetation dynamics.

\begin{figure}

{\centering \includegraphics[width=0.8\linewidth]{figures/chap1/timeline} 

}

\caption{Timeline showing the parallel development of model types and the integration of model types into land surface models towards terrestrial biosphere models. (Bonan 2019)}\label{fig:f7}
\end{figure}

\section{Components of a model}\label{components-of-a-model}

What is a vegetation model?

\begin{itemize}
\tightlist
\item
  \textbf{Dynamic global vegetation models} (DGVMs) are powerful tools
  to project past, current and future vegetation patterns and associated
  biogeochemical cycles (Scheiter et al., 2013).
\item
  A \textbf{Dynamic Global Vegetation Model} (DGVM) is a computer
  program that simulates shifts in potential vegetation and its
  associated biogeochemical and hydrological cycles as a response to
  shifts in climate. DGVMs use time series of climate data and, given
  constraints of latitude, topography, and soil characteristics,
  simulate monthly or daily dynamics of ecosystem processes. DGVMs are
  used most often to simulate the effects of future climate change on
  natural vegetation and its carbon and water cycles (Wikipedia 2021).
\end{itemize}

\begin{center}
\captionof{table}{Definition of key model components and examples for a typical TBM}
\label{table:components}

\begin{center}\includegraphics[width=0.8\linewidth]{figures/chap1/table_components} \end{center}
\end{center}

\subsection{Processes}\label{processes}

Key component because we are focusing on process-based models. There is
an all list of processes (energy, water, turbulent transport, canopy
scaling, carbon, nitrogen, trace gasses, demography,\ldots{}) that
integrates the models, especially the more complex ones. These processes
will be discussed in detail in the following theory lectures and how to
translate them into equations.

\subsection{Equations}\label{equations}

Representations of the processes. However, it is impossible to insert
any equation into the model because there are important constraints,
such as the specific time. For example, it makes little sense to resolve
the equation for forest composition (succession) every day. This is a
prolonged process with an extremely low variance between consecutive
days. The solution for the equation for photosynthesis, on the other
hand, varies significantly throughout the day and between consecutive
days (cloudy day vs sunny day). There are three types of equations
within the model: - \textbf{prognostic equations}: time derivatives of
differential equations -- calculate the state's change over time -
\textbf{conservation equations}: conservation of mass and energy -
\textbf{diagnostic equations}: linking multiple variables independent
from the time. Often there is no analytical solution in biological
systems; therefore, we must use numerical methods to solve the
equations.

\subsection{Parameters}\label{parameters}

Constants in the model. Some parameters are quite uncertain because we
can not measure very well or on the scale, we want to use. For example,
we can be confident of the photosynthetic capacity of a leaf. Upscaling
this result so that it is applicable for a forest or multiple PTF (plant
functional types) induces uncertainty. The more parameters a model uses,
the more uncertainties that are to be taken into account!

\subsection{Time Steps}\label{time-steps}

Vegetation models run at multiple timescales (combining processes that
are resolved at multiple timescales). Models present fast processes,
which are calculated every hour (photosynthesis and energy balance),
intermediate processes calculated daily (carbon allocation and growth)
and slow processes in order of years (mortality) (Fig.6).

\begin{figure}

{\centering \includegraphics[width=0.8\linewidth]{figures/chap1/time_steps} 

}

\caption{Structure of a vegatation model indicating the different time steps at which each process is simulated (Williams et al. 2009)}\label{fig:f9}
\end{figure}

\subsection{Spatial structure}\label{spatial-structure}

Division of space or how many times we repeat our calculations in space.
The global models have a spatial grid of 100km or even more and divide
the landscape into patches. In each patch, they simulate the forest (or
savannas, grassland\ldots{}). Models also have a horizontal grid or
horizontal layering: some models consider multiple soil layers. The same
is true for above ground layers, where some models divide the canopy
into multiple layers (Figure \ref{fig:f10}). For example, the Ecosystem
Demography Model (ED2.2) divides the forest into multiple grid cells
where the same meteorological conditions apply within each grid cell.
Then within each cell, he has different sites with different soils. Each
site is divided into multiple patches (forests with a similar
disturbance history). Into each patch, we simulate different cohorts of
trees (Figure \ref{fig:f11}).

\begin{figure}

{\centering \includegraphics[width=0.8\linewidth]{figures/chap1/grid_vert_hor} 

}

\caption{Three dimensional grid of a TBM structured in terms of longitude x latitude x level. The number of soil and canopy layers and the geographical resolution is model dependent, (Bonan 2019)}\label{fig:f10}
\end{figure}

\begin{figure}

{\centering \includegraphics[width=0.8\linewidth]{figures/chap1/grid_ED2} 

}

\caption{Example: the spatial multi-level grid structure of of the ED2 vegetation model (Longo et al. 2019)}\label{fig:f11}
\end{figure}

\subsection{Model code, complexity and
uncertainty}\label{model-code-complexity-and-uncertainty}

There is a gap between equations and model code. Also, it is a variety
of how a specific process is implemented in an equation. Usually, large
models also contained a ``technical debt'', which means over the years,
multiple modellers have continued working on models and add code lines,
but at some point, the code is so large that anyone knows all code.

Models are always a simplification of the real world, but they tend to
become overly complex

More complex models (adding more processes) become more realistic, but
we also add more uncertainty sources. Therefore, we should choose our
model carefully based on a research question.

\subsection{Data}\label{data}

Not possible to develop models without data. In general, the more data
(multiple data sources), the better.

\section{Modelling workflow and structure of the
course}\label{modelling-workflow-and-structure-of-the-course}

Vegetation modelling is a multidisciplinary field. This course will
mainly focus on the mathematical formulation of processes and
translating these equations into a working model.

\begin{figure}

{\centering \includegraphics[width=0.8\linewidth]{figures/chap1/course_overview} 

}

\caption{Progression through spatial and temporal scales throughout this course}\label{fig:f12}
\end{figure}

The construction of a model is a continuous process -- a model is never
finished. As Figure 10 shows us, we start by describing our system in
the form of equations, then running the computer program to characterize
the model, perform parameter estimation and interpretation, and then
apply it to other locations and validate against independent data.

\begin{figure}

{\centering \includegraphics[width=0.8\linewidth]{figures/chap1/williams_fusion} 

}

\caption{Model data fusion in every step of the model development cycle (Williams et al. 2009)}\label{fig:f13}
\end{figure}

\begin{figure}

{\centering \includegraphics[width=0.8\linewidth]{figures/chap1/dietze_workflow} 

}

\caption{Methodological workflow of model data fusion (Dietze: Ecological Forecasting)}\label{fig:f14}
\end{figure}

\part{Biophysical and physiological
models}\label{part-biophysical-and-physiological-models}

\chapter{Modelling plant basic
processes}\label{modelling-plant-basic-processes}

\chaptermark{photsynthesis}

\textbf{For all processes, we provide an overview of existing models and
approaches and we will detail only one of them for the practical course.
This also apply for the other chapters, the idea of the course is to be
more conceptual about how we model vegetation and the different
applications and assumptions.}

\section{Photosynthesis models}\label{photosynthesis-models}

\subsection{Refreshing the basic
knowledge}\label{refreshing-the-basic-knowledge}

\begin{figure}

{\centering \includegraphics[width=0.8\linewidth]{figures/chap2/leaf_level_processes} 

}

\caption{Leaf level processes transpiration and photosysnthesis are strongy interlinked and both regulated by stomatal conductance}\label{fig:f21}
\end{figure}

\begin{figure}

{\centering \includegraphics[width=0.8\linewidth]{figures/chap2/photosynthesis_obs} 

}

\caption{Photosynthesis in relation to (a) photosynthetically active radiation,(b) temperature, (c) vapor pressure deficit at 25°C and 35°C,(d) foliage water potential, (e) ambient CO2 concentration, and (f) foliage water potential for jack pine trees (Pinus banksiana). Bonan (2019)}\label{fig:f22}
\end{figure}

\begin{figure}

{\centering \includegraphics[width=0.8\linewidth]{figures/chap2/Trespons_interactions} 

}

\caption{Temperature responses of photosynthesis, respiration and net CO2 exchange, interaction with CO2 concentration (A) and light  (B)  Schulze ()}\label{fig:f23}
\end{figure}

\subsection{C3 photosynthesis}\label{c3-photosynthesis}

\subsubsection{Light response curve
models}\label{light-response-curve-models}

\begin{figure}

{\centering \includegraphics[width=0.8\linewidth]{figures/chap2/LRC} 

}

\caption{Conceptual figure of a leaf-level light reponse curve}\label{fig:f24}
\end{figure}

\begin{figure}

{\centering \includegraphics[width=0.8\linewidth]{figures/chap2/hyperbola} 

}

\caption{Co-limitation illustrated for photosynthetic response to light. The two dashed lines show the rates Amax adn EI The solid lines show the co-limited rate. (Bonan 2019)}\label{fig:f25}
\end{figure}

\subsubsection{Light use efficiency
models}\label{light-use-efficiency-models}

\begin{figure}

{\centering \includegraphics[width=0.8\linewidth]{figures/chap2/MODIS_GPP} 

}

\caption{MODIS based GPP map of the US, based on a LUE model.}\label{fig:f26}
\end{figure}

\subsubsection{The Farquahar model}\label{the-farquahar-model}

\begin{figure}

{\centering \includegraphics[width=0.8\linewidth]{figures/chap2/conductance} 

}

\caption{Diffusion of CO2 from free air across the leaf boundary layer and through stomata to the intercellular space. Diffusion to the chloroplast is additionally regulated by mesophyl conductance. (Bonan 2019)}\label{fig:f27}
\end{figure}

\begin{figure}

{\centering \includegraphics[width=0.8\linewidth]{figures/chap2/simulated_responses} 

}

\caption{Simulated responses of C3 photosynthesis in relation to (a) intercellular CO2 (at I↓ = 2000 μmol m–2 s–1) and (b) photosynthetically active radiation (at ci = 266 μmol mol–1). (Bonan 2019)}\label{fig:f28}
\end{figure}

\begin{itemize}
\tightlist
\item
  UCL 4.6.1
\item
  Bonan, Chapter 11.1 The FvCB model Most equations between 11.1 and
  11.31 Figure 11.2 a and b Table 11.1 for parameters values + a few
  simulations to illustrate Table 11.4
\end{itemize}

\subsection{Parameter and temperature
dependencies}\label{parameter-and-temperature-dependencies}

\begin{figure}

{\centering \includegraphics[width=0.8\linewidth]{figures/chap2/temp_responses} 

}

\caption{Relative temperature responses of the parameters of the Farquhar model (Bonan 2019)}\label{fig:f28bis}
\end{figure}

\begin{figure}

{\centering \includegraphics[width=0.8\linewidth]{figures/chap2/vcmax_jmax} 

}

\caption{Linear relation between observed Vcmax and Jmax values for Beech (Verbeeck et al. 2008)}\label{fig:f29}
\end{figure}

\begin{figure}

{\centering \includegraphics[width=0.8\linewidth]{figures/chap2/full_farquhar} 

}

\caption{Equations of the full Farquhar model}\label{fig:f210}
\end{figure}

\begin{itemize}
\tightlist
\item
  Bonan, Chapter 11.2 Equations 11.34-11.37 Table 11.2 Figure 11.3 for
  illustration
\end{itemize}

Summary with Table 11.5 and Figure 11.4

\subsection{C4 photosynthesis}\label{c4-photosynthesis}

\begin{itemize}
\tightlist
\item
  Bonan, Chapter 11.7 PEP carboxylase Equations 11.69-11.74 Find an
  illustration
\end{itemize}

\begin{figure}

{\centering \includegraphics[width=0.8\linewidth]{figures/chap2/c3_c4} 

}

\caption{Comparison of C3 and C4 photosynthesis in response to (a) photosynthetically active radiation, (b) ambient CO2 concentration, (c) leaf temperature, and (d) vapor pressure deficit. In this figure, stomatal conductance is calculated using the Ball–Berry model and ci is obtained from the diffusion equation}\label{fig:f210b}
\end{figure}

\section{Stomatal models}\label{stomatal-models}

\subsection{Refreshing the basic
knowledge}\label{refreshing-the-basic-knowledge-1}

\begin{figure}

{\centering \includegraphics[width=0.8\linewidth]{figures/chap2/conductance} 

}

\caption{Diffusion of CO2 from free air across the leaf boundary layer and through stomata to the intercellular space. Diffusion to the chloroplast is additionally regulated by mesophyl conductance. (Bonan 2019)}\label{fig:f211}
\end{figure}

\begin{figure}

{\centering \includegraphics[width=0.8\linewidth]{figures/chap2/gs_obs} 

}

\caption{Observed responses of stomatal conductance for Pinus banksiana. (Bonan 2019)}\label{fig:f212}
\end{figure}

\subsection{Empirical multiplicative
models}\label{empirical-multiplicative-models}

\begin{itemize}
\tightlist
\item
  Bonan, Chapter 12.2
\end{itemize}

\subsection{Semiempirical photosynthesis-based
models}\label{semiempirical-photosynthesis-based-models}

\begin{itemize}
\tightlist
\item
  Bonan, Chapter 12.3
\end{itemize}

\begin{figure}

{\centering \includegraphics[width=0.8\linewidth]{figures/chap2/ball_berry} 

}

\caption{Relationship between stomatal conductance and Anhs/cs for soybean.(Bonan 2019)}\label{fig:f213}
\end{figure}

\begin{figure}

{\centering \includegraphics[width=0.8\linewidth]{figures/chap2/numerical_solution} 

}

\caption{Flow diagram of the iterative procedure to numerically calculate ci.(Bonan 2019)}\label{fig:f214}
\end{figure}

\subsection{WUE models and optimality
theory}\label{wue-models-and-optimality-theory}

\begin{itemize}
\tightlist
\item
  Bonan, Chapter 12.4 + add optimality approach from Prentice et al.
\end{itemize}

\begin{figure}

{\centering \includegraphics[width=0.8\linewidth]{figures/chap2/optimality} 

}

\caption{Flow diagram of leaf flux calculations to numerically solve for stomatal conductance that optimizes water-use efficiency.(Bonan 2019)}\label{fig:f215}
\end{figure}

\subsection{Soil drought stress}\label{soil-drought-stress}

\begin{figure}

{\centering \includegraphics[width=0.8\linewidth]{figures/chap2/leafWP} 

}

\caption{Leaf carbon uptake in response to leaf water potential for multiple tree species.}\label{fig:f216}
\end{figure}

\begin{figure}

{\centering \includegraphics[width=0.8\linewidth]{figures/chap2/SWfactor} 

}

\caption{Soil moisture wetness factor in relation to volumetric water content. (Bonan 2019)}\label{fig:f217}
\end{figure}

\subsection{Hydraulic models}\label{hydraulic-models}

\begin{figure}

{\centering \includegraphics[width=0.8\linewidth]{figures/chap2/hydraulics} 

}

\caption{Flow of water and representative water potentials along the soil–plant–atmosphere continuum. Also shown are conductances along the hydraulic pathway.(Bonan 2019)}\label{fig:f218}
\end{figure}

\begin{figure}

{\centering \includegraphics[width=0.8\linewidth]{figures/chap2/SPA} 

}

\caption{Depiction of (a) plant hydraulics and (b) leaf gas exchange in the Soil–Plant–Atmosphere (SPA) model. SPA is a multilayer canopy model.(Bonan 2019)}\label{fig:f219}
\end{figure}

\begin{figure}

{\centering \includegraphics[width=0.8\linewidth]{figures/chap2/modelling_approaches} 

}

\caption{Simulated stomatal responses for various modelling approaches. (Bonan 2019)}\label{fig:f220}
\end{figure}

Figure 13.1 The soil-plant-atmosphere model Leaf water potential Plant
water uptake Resistance analogy Multinode models

\section{Upscaling from leaf to
canopy}\label{upscaling-from-leaf-to-canopy}

\begin{itemize}
\tightlist
\item
  Quickly introduce the problem of scaling in ecology (review paper of
  Jerome Chave) and refer to chapter 10 on upscalling
\item
  Canopy integration: LAI layers, etc\ldots{} Nice transition to chap 3
  with the interception of light by the canopy
\end{itemize}

Leaf microclimate and boundary layer processes in relation to leaf
dimension for sun and shade conditions.

\begin{figure}

{\centering \includegraphics[width=0.8\linewidth]{figures/chap2/sun_shade} 

}

\caption{Leaf microclimate and boundary layer processes in relation to leaf dimension for sun and shade conditions.(Bonan 2019)}\label{fig:f221}
\end{figure}

\begin{figure}

{\centering \includegraphics[width=0.8\linewidth]{figures/chap2/GPPcontrols} 

}

\caption{Controlling facors on ecosystem GPP. (Chapin)}\label{fig:f222}
\end{figure}

\section{Case studies}\label{case-studies}

\subsection{Case study 2.1 Ozone impact on global
GPP}\label{case-study-2.1-ozone-impact-on-global-gpp}

\begin{figure}

{\centering \includegraphics[width=0.8\linewidth]{figures/chap2/ozone} 

}

\caption{Simulated global GPP reduction in response to current and future atmospheric ozone concentrations}\label{fig:f223}
\end{figure}

\subsection{Case study 2.2 Drought impact on rainforest
GPP}\label{case-study-2.2-drought-impact-on-rainforest-gpp}

\begin{figure}

{\centering \includegraphics[width=0.8\linewidth]{figures/chap2/fisher1} 

}

\caption{Simulated (SPA model) and observed sapflow for a drought experiment in the Amazon; Fisher et al. 2007}\label{fig:f224}
\end{figure}

\begin{figure}

{\centering \includegraphics[width=0.8\linewidth]{figures/chap2/fisher2} 

}

\caption{Simulated (SPA model) gs and GPP for a drought experiment in the Amazon. Fisher et al. 2007}\label{fig:f225}
\end{figure}

\chapter{Modelling radiation, vegetation canopies, and energy
balance}\label{modelling-radiation-vegetation-canopies-and-energy-balance}

\chaptermark{Light}

\section{Introduction}\label{introduction}

\begin{figure}

{\centering \includegraphics[width=0.8\linewidth]{figures/chap3/f31_LAD} 

}

\caption{Generalized profiles of leaf area density in plant canopies. (Bonan)}\label{fig:f31}
\end{figure}

\begin{figure}

{\centering \includegraphics[width=0.8\linewidth]{figures/chap3/f32_cLAI} 

}

\caption{Cumulative LAI and WAI in a deciduous oak-hickory forest. (Bonan)}\label{fig:f32}
\end{figure}\begin{figure}

{\centering \includegraphics[width=0.8\linewidth]{figures/chap3/f33_Langle} 

}

\caption{Illustration of a leaf (thick line) oriented at an angle Θℓ to horizontal. (Bonan)}\label{fig:f33}
\end{figure}

\begin{figure}

{\centering \includegraphics[width=0.8\linewidth]{figures/chap3/f34_angle_distr} 

}

\caption{Planophile, erectophile, plagiophile, and spherical leaf angle distributions showing (a) the probability density function f(Θℓ) and (b) the cumulative distribution F(Θℓ). (Bonan)}\label{fig:f34}
\end{figure}

\begin{figure}

{\centering \includegraphics[width=0.8\linewidth]{figures/chap3/f35_architecture} 

}

\caption{Illustration of leaf angle distributions and canopy architecture in general influences radiation attenuation in vegetation canopies.}\label{fig:f35}
\end{figure}

\begin{figure}

{\centering \includegraphics[width=0.8\linewidth]{figures/chap3/f36_obs_profile} 

}

\caption{Profile of light and foliage in a stand of herbaceous plants approximately 130 cm tall. The horizontal axis shows transmittance as a fraction of incident radiation (top axis) and foliage mass (bottom axis) at various heights in the canopy. (Bonan)}\label{fig:f36}
\end{figure}

\section{Radiative transfer
modelling}\label{radiative-transfer-modelling}

\begin{figure}

{\centering \includegraphics[width=0.8\linewidth]{figures/chap3/f37_RT_principle} 

}

\caption{Representation of a canopy as (a) one-dimensional with a vertical profile of leaf area (shown by grayscale gradation in which darker shading denotes more leaves) that is horizontally homogenous and (b) threedimensional with vertical and spatial structure determined by crown geometry and spacing. (Bonan)}\label{fig:f37}
\end{figure}

\subsection{Leaf optical properties}\label{leaf-optical-properties}

\begin{figure}

{\centering \includegraphics[width=0.8\linewidth]{figures/chap3/f37_leaf_optical} 

}

\caption{Spectrum of absorptance, reflectance and transmittance of a typical plant leaf (Jones, 2014)}\label{fig:f37b}
\end{figure}

\begin{figure}

{\centering \includegraphics[width=0.8\linewidth]{figures/chap3/f38_table_optical} 

}

\caption{Table showing typical reflectance and absorptance values for leaves and vegetation canopies of different Plant Functional Types (PFT).(Jones, 2014)}\label{fig:f38}
\end{figure}

\subsection{Light transmission without
scattering}\label{light-transmission-without-scattering}

\begin{figure}

{\centering \includegraphics[width=0.8\linewidth]{figures/chap3/f39_beer} 

}

\caption{Transmission of solar radiation through a homogeneous medium in the absence of scattering. In this example, n non-overlapping opaque particles each with cross-sectional area a oriented perpendicular to the path of light are placed in a medium with cross-sectional area A and thickness dz. The radiation absorbed in the medium is dI.(Bonan)}\label{fig:f39}
\end{figure}

\begin{figure}

{\centering \includegraphics[width=0.8\linewidth]{figures/chap3/f310_Kb} 

}

\caption{Transmission of direct beam radiation τb in relation to leaf area index for typical values of the extinction coefficient Kb. (Bonan)}\label{fig:f310}
\end{figure}

\begin{figure}

{\centering \includegraphics[width=0.8\linewidth]{figures/chap3/f311_LLh} 

}

\caption{Extinction coefficient in relation to solar zenith angle Ζ and leaf inclination angle Θℓ. In each panel, a unit leaf area (L = 1), shown with a thick line, is projected onto a horizontal surface LH so that Kb = LH. The leaf inclination angle is 0° (bottom panels), 30° (middle panels), and 60° (top panels). In the left and middle columns, the leaf is oriented towards the Sun (Αℓ − Α = 0°) and the solar zenith angle is 0° (left column) and 15° (middle column). In the right column, Ζ = 15°, but the leaf is oriented away from the Sun (Αℓ − Α = 180°). In each panel, the arrows indicate the solar beam (Bonan)}\label{fig:f311}
\end{figure}

\begin{figure}

{\centering \includegraphics[width=0.8\linewidth]{figures/chap3/f312_Kb_angle} 

}

\caption{Extinction coefficients for horizontal, spherical, and vertical leaf angle distributions. (a) Direct beam radiation Kb in relation to solar zenith angle. (b) Diffuse radiation Kd in relation to leaf area index(Bonan)}\label{fig:f312}
\end{figure}

\begin{figure}

{\centering \includegraphics[width=0.8\linewidth]{figures/chap3/f313_sun_shade} 

}

\caption{Radiative transfer and sunlit leaf area index for a canopy of horizontal leaves (top panels) with Kb = 1 and vertical leaves (bottom panels) with Kb = 0.112. The left-hand panels show a canopy consisting of four layers of leaves. Each thick black line represents a leaf area index of 0.1 m2 m–2. The thin lines depict interception or transmission of beam radiation with a zenith angle of 10°. The middle panels show cumulative leaf area index and sunlit leaf area index with depth in the canopy. The right-hand panels show direct beam transmittance with depth in the canopy. (Bonan)}\label{fig:f313}
\end{figure}

\begin{figure}

{\centering \includegraphics[width=0.8\linewidth]{figures/chap3/f314_sunlit} 

}

\caption{Sunlit leaf area index in relation to total leaf area index for horizontal, spherical, and vertical foliage orientations with solar zenith angle Ζ = 30°. Kb = 1, 0.577, and 0.368 for horizontal, spherical, and vertical foliage. (Bonan)}\label{fig:f314}
\end{figure}

\begin{figure}

{\centering \includegraphics[width=0.8\linewidth]{figures/chap3/f315_clumping} 

}

\caption{Images illustrating leaf/canopy clumping a various scales: leaf, crown, stand.}\label{fig:f315}
\end{figure}

\subsection{Diffuse transmittance}\label{diffuse-transmittance}

\begin{figure}

{\centering \includegraphics[width=0.8\linewidth]{figures/chap3/f316_diffuse} 

}

\caption{Illustration of direct beam and diffuse radiation. The sky forms a bowl, or inverted hemisphere, over a horizontal surface. Shown is a cross section of the sky hemisphere. Direct beam (solid line) originates from the    direction of the Sun with zenith angle Ζ. Diffuse radiation (dashed lines) can be treated as independent beams of radiation each with an angle Ζ. The shaded region is the relative contribution between sky angles Ζ1 and Ζ2 to total sky irradiance.(Bonan)}\label{fig:f316}
\end{figure}

\begin{figure}

{\centering \includegraphics[width=0.8\linewidth]{figures/chap3/f317_diff_trans} 

}

\caption{Transmittance of diffuse radiation τd in relation to leaf area index for a spherical leaf distribution. Show are the transmittances for sky zones of 0°–30°, 30°–60°, and 60°–90° and also the total transmittance. Fill patterns show the contribution of each sky zone to total transmittance.(Bonan)}\label{fig:f317}
\end{figure}

\begin{figure}

{\centering \includegraphics[width=0.8\linewidth]{figures/chap3/f318_diff_dir_trans} 

}

\caption{Transmission of solar radiation through a canopy with spherical leaf distribution in relation to leaf area index. The solid lines show direct beam transmittance τb for solar zenith angles of 0°–80° (in 10° increments).The dashed line shows the diffuse transmittance τd. (Bonan)}\label{fig:f318}
\end{figure}

\subsection{The Norman Model(1979)}\label{the-norman-model1979}

\begin{figure}

{\centering \includegraphics[width=0.8\linewidth]{figures/chap3/f319_Norman} 

}

\caption{Radiative fluxes in a canopy of N leaf layers. The vertical profile is oriented with i = 1 the leaf layer at the bottom of the canopy, leaf layer i + 1 above layer i, and i = N the leaf layer at the top of the canopy. Each layer has a leaf area index ΔL. is the downward diffuse shortwave flux onto layer i, is the upward diffuse shortwave flux above layer i, and is the unscattered direct beam flux onto layer i. and are the corresponding downward and upward fluxes of longwave radiation. These depend on leaf Tℓand ground Tg temperatures. Thick arrows denote boundary conditions of diffuse solar radiation , direct beam solar radiation, and atmospheric longwave radiation at the top of the canopy.(Bonan)}\label{fig:f319}
\end{figure}

\subsection{The Goudriaan and van Laar Model
(1994)}\label{the-goudriaan-and-van-laar-model-1994}

\begin{figure}

{\centering \includegraphics[width=0.8\linewidth]{figures/chap3/f320_goudriaan} 

}

\caption{ Derivation of absorbed direct beam solar radiation for a leaf layer with leaf area index ΔL (Goudriaan 1982). ρc is the reflectance of the leaf layer.(Bonan)}\label{fig:f320}
\end{figure}

\subsection{The Two-Stream
approximation}\label{the-two-stream-approximation}

\begin{figure}

{\centering \includegraphics[width=0.8\linewidth]{figures/chap3/f321_two_stream} 

}

\caption{Fluxes for (a) direct beam and (b) diffuse radiation in the twostream approximation for a canopy with leaf area index L.(Bonan)}\label{fig:f321}
\end{figure}

\subsection{Longwave radiation}\label{longwave-radiation}

\begin{figure}

{\centering \includegraphics[width=0.8\linewidth]{figures/chap3/f322_LW} 

}

\caption{Longwave radiation fluxes represented for a single leaf layer.(a) Norman’s (1979) numerical model. Shown is the radiative balance for leaf layer i + 1 located above leaf layer i. (b) A simplified model to allow only forward scattering (ρℓ = 0 and τℓ = ωℓ = 1 − εℓ) and to permit an analytical solution integrated over a canopy. In both panels, emitted radiation is excluded. Thick lines denote fluxes incident onto the layer. (Bonan)}\label{fig:f322}
\end{figure}

\section{Representing canopy structure in
models}\label{representing-canopy-structure-in-models}

\subsection{Big-leaf models}\label{big-leaf-models}

\begin{figure}

{\centering \includegraphics[width=0.8\linewidth]{figures/chap3/f323_bigleaf} 

}

\caption{Scaling of leaf fluxes to the canopy using a big-leaf model. (a) Shown are leaf sensible heat, transpiration, and CO2 fluxes in relation to various conductances. Fluxes are exchanged between the leaf and air around the leaf. Also shown is the total resistance. (b) Shown are big-leaf canopy fluxes in which leaf fluxes are scaled by the average conductance and leaf area index and are further modified by turbulent transport in the atmospheric surface layer. Surface layer processes are commonly omitted for CO2 exchange. Only a single big leaf is shown, but separate sunlit and shaded big leaves can be similarly depicted. (Bonan)}\label{fig:f323}
\end{figure}

\begin{figure}

{\centering \includegraphics[width=0.8\linewidth]{figures/chap3/f324_vcamx_profile} 

}

\caption{Canopy profiles of relative photosynthetic capacity in relation to cumulative leaf area index. Thin lines show exponential profiles using values of Kn for 16 temperate broadleaf forests and two tropical forests ranging from 0.10 to 0.43 (Lloyd et al. 2010). The two thick lines show observed profiles of Vcmax and Jmax from Niinemets and Tenhunen (1997) obtained for sugar maple (Acer saccharum). (Bonan)}\label{fig:f324}
\end{figure}

\subsection{Multilayer models}\label{multilayer-models}

\begin{figure}

{\centering \includegraphics[width=0.8\linewidth]{figures/chap3/f325_multilayer_process} 

}

\caption{Overview of the main processes in a multilayer canopy model.The canopy is represented by N leaf layers with layer i + 1 above layer i. (a) Diffuse and direct beam solar radiation is transmitted or intercepted. The intercepted portion is absorbed or scattered in the forward and backward direction. Longwave radiation is similar to diffuse radiation. (b) Leaf sensible heat, transpiration, and CO2 fluxes depend on absorbed radiation and leaf boundary layer and stomatal conductances. Sensible heat is exchanged from both sides of the leaf. Water vapor and CO2 can be exchanged from one or both sides of the leaf depending on stomata. Leaf temperature is the temperature that balances the energy budget. (c) Stomatal conductance depends on leaf water potential. Plant water uptake for a canopy layer is in relation to belowground soil and root conductance and aboveground stem conductance acting in series and also a capacitance term. See Figure 13.4a for more details. (d) Scalar profiles are calculated from a conductance network. Leaf fluxes provide the source or sink of heat, water vapor, and CO2, along with soil fluxes. (e) Sensible heat, latent heat, and heat storage in soil depend on the ground temperature that balances the soil energy budget. (f) The wetted fraction of the canopy layer depends on the portion of precipitation that is intercepted. (Bonan)}\label{fig:f325}
\end{figure}

\begin{figure}

{\centering \includegraphics[width=0.8\linewidth]{figures/chap3/f326_multilayer_solving} 

}

\caption{Flow diagram of processes in a multilayer canopy model. The shaded area denotes leaf processes resolved at each layer in the canopy. This is a generalized diagram of the required calculations for a dry leaf. Specific models differ in how the equation set is solved and the iterative calculations. Evaporation of intercepted water requires additional complexity.(Bonan)}\label{fig:f326}
\end{figure}

\subsection{3D ray tracing models}\label{d-ray-tracing-models}

\begin{figure}

{\centering \includegraphics[width=0.8\linewidth]{figures/chap3/f327_DART} 

}

\caption{Example of the PROSPECT leaf optical model and the DART 3D ray tracing model.}\label{fig:f327}
\end{figure}

\begin{figure}

{\centering \includegraphics[width=0.8\linewidth]{figures/chap3/f328_TLS_RT} 

}

\caption{Example of a study that uses terrestrial laser scanning (TLS) to construct a full 3D model of a forest as input for a 3D ray tracing model (Kükenbrink et al. 2020) }\label{fig:f328}
\end{figure}

\section{Ecosystem energy balance}\label{ecosystem-energy-balance}

\subsection{Basic principles}\label{basic-principles}

\subsection{Surface radiation balance}\label{surface-radiation-balance}

\begin{figure}

{\centering \includegraphics[width=0.8\linewidth]{figures/chap3/f329_rad_balance} 

}

\caption{Radiative balance of an opaque gray body receiving downwelling solar S↓ and longwave L↓ radiation.(Bonan)}\label{fig:f329}
\end{figure}

\subsection{Bulk surface energy
balance}\label{bulk-surface-energy-balance}

\begin{figure}

{\centering \includegraphics[width=0.8\linewidth]{figures/chap3/f330_E_balance} 

}

\caption{Conductance networks for sensible heat flux (top) and latent heat flux (bottom) for various depictions of the land surface. This chapter describes the bulk surface and big-leaf canopies. (Bonan)}\label{fig:f330}
\end{figure}

\subsection{Leaf energy balance}\label{leaf-energy-balance}

\begin{figure}

{\centering \includegraphics[width=0.8\linewidth]{figures/chap3/f331_leaf_E_balance} 

}

\caption{Biophysics and biochemistry of leaves. (a) The radiative environment consists of solar radiation (left) and longwave radiation (right). (b) Leaf fluxes include CO2, H2O, and heat through the boundary layer. These fluxes are shown as a network of conductances for the adaxial (upper) and abaxial (lower) leaf surfaces. For H2O and CO2, the conductance for each surface is obtained from stomatal and boundary layer conductances acting in series. The total conductance is defined by the adaxial and abaxial surfaces acting in parallel. (c) Stomata open to absorb CO2 for photosynthesis, but, in doing so, water is lost as transpiration. (Bonan)}\label{fig:f331}
\end{figure}

\section{Case studies}\label{case-studies-1}

\subsection{Case study 3.1}\label{case-study-3.1}

\begin{figure}

{\centering \includegraphics[width=0.8\linewidth]{figures/chap3/f331_leaf_E_balance} 

}

\caption{Principle of the effect of increased diffuse raditaion on leaf/canopy photosynthesis. (Knohl et al. 2008)}\label{fig:f332}
\end{figure}

\begin{figure}

{\centering \includegraphics[width=0.8\linewidth]{figures/chap3/f333_knohl2} 

}

\caption{Resulting impact of changing diffuse fraction on carbon and water fluxes and WUE (Knohl et al. 2008)}\label{fig:f333}
\end{figure}

\subsection{Case study 3.2}\label{case-study-3.2}

\begin{figure}

{\centering \includegraphics[width=0.8\linewidth]{figures/chap3/f334_chen1} 

}

\caption{Global map of LAI trend between 1981 and 2016 based on remote sensing (Chen et al. 2021).}\label{fig:f334}
\end{figure}

\begin{figure}

{\centering \includegraphics[width=0.8\linewidth]{figures/chap3/f335_chen2} 

}

\caption{Simulated impact of different factors contributing to the increased global land C sink since 1981 (Chen et al. 2021) }\label{fig:f335}
\end{figure}

\chapter{Temporal and seasonal
dynamics}\label{temporal-and-seasonal-dynamics}

\chaptermark{dynamics}

\section{Phenology}\label{phenology}

\begin{itemize}
\tightlist
\item
  UCL 4.5.2 The study of life-cycle events. Here refers to the temporal
  dynamics of vegetation. Broader sense of phenology. Applies to most
  plant. Circadian to seasonal cycles Tissue turnover and senescence
\end{itemize}

\subsection{The example of crop
phenology}\label{the-example-of-crop-phenology}

\begin{itemize}
\tightlist
\item
  UCL 4.5.1 Definition +Simple example of phenology (wheat for example:
  germination + spread + full coverage + allocation to storage organs +
  ripening)
\end{itemize}

\section{Mechanisms of phenology and evidence of
changes}\label{mechanisms-of-phenology-and-evidence-of-changes}

\begin{itemize}
\tightlist
\item
  UCL 4.5.3
\end{itemize}

\subsection{Overview of controls at different
levels}\label{overview-of-controls-at-different-levels}

\begin{itemize}
\tightlist
\item
  temperature
\item
  light
\item
  water
\item
  Nutrients
\item
  Drivers of seasonality and phenology
\end{itemize}

\subsection{Vegeation index and changes over
time}\label{vegeation-index-and-changes-over-time}

\begin{itemize}
\tightlist
\item
  This is also why phenology is a ``metrics'' of climate change
\end{itemize}

\subsection{Seasonality feedbacks}\label{seasonality-feedbacks}

\begin{itemize}
\tightlist
\item
  Phenology affects phenology (phenophases are linked, one perturbation
  will affect the next phenophase in the cycle)
\item
  The control of phenology on climate: example, early spring leaf
  unfolding excacerbates drought in summer
\end{itemize}

\section{Models of phenology}\label{models-of-phenology}

\begin{itemize}
\tightlist
\item
  UCL 4.5.4
\end{itemize}

\subsection{budburst models}\label{budburst-models}

\subsection{senescence models}\label{senescence-models}

\subsection{Phenology in DGVMs}\label{phenology-in-dgvms}

\begin{itemize}
\tightlist
\item
  Figures from Zhang et al. 2003
\end{itemize}

\part{Modelling vegetation
dynamics}\label{part-modelling-vegetation-dynamics}

\chapter{Modelling plant growth and biogeochemical cycles in vegetation
models}\label{modelling-plant-growth-and-biogeochemical-cycles-in-vegetation-models}

\chaptermark{Grotwh} - UCL 4.2.2

based on Bonan Chapter 17

\section{Process-based growth
modelling}\label{process-based-growth-modelling}

\subsection{C-allocation models}\label{c-allocation-models}

\begin{itemize}
\tightlist
\item
  C pools: Allocation to leaf, wood, fruit
\end{itemize}

\subsection{Applications of growth modelling in forestry and
agriculture}\label{applications-of-growth-modelling-in-forestry-and-agriculture}

\begin{itemize}
\tightlist
\item
  short, link with inventory course
\end{itemize}

\section{Carbon cycle models: stocks and
fluxes}\label{carbon-cycle-models-stocks-and-fluxes}

This chapter will develop the ecological foundation and mathematics to
describe ecosystem carbon dynamics using biogeochemical models.
Biogeochemical models abstract an ecosystem as pools of carbon and the
flows of carbon among these pools.

Use specific model as an example to illustrate? In Bonan: CASA-CNP model

\subsection{Model structure}\label{model-structure}

Biogeochemical models simulate processes of allocation of photosynthetic
carbon gain to plant parts (e.g., foliage, fine root, wood), turnover of
plant biomass as litterfall, transformation of litter to soil organic
matter, and carbon loss during respiration.

Principles: - net carbon input is equal to gross primary production
minus autotrophic respiration; - carbon flows from donor to receiver
pools at a rate that depends on the donor pool size and its chemical
quality as modified by the environment; - mass balance is maintained as
carbon flows through the system of interconnected pools; - decay of
litter and soil organic matter releases CO 2 as heterotrophic
respiration.

Models: a system of first-order, linear differential equations to
describe carbon pools and fluxes (typically time step of one day)

Pools and fluxes to be included: - Carbon gain from gross primary
production minus autotrophic respiration - Allocation of carbon to
growth of leaves, wood, and roots pools (partitioning varies with light
availability, soil temperature, soil moisture, and nutrients + temporal
for leaves (ref to phenology) - Carbon turnover (comprising litterfall,
background mortality, and disturbances) + turnover rates depending on
the plant material - litter pools: metabolic litter, structural litter,
coarse woody debris (vary in chemical quality and turnover rate; base
turnover rates are modified for soil temperature and soil moisture
(environmental scaling factors)) - decomposition to soil organic matter
pools: fast SOM, slow SOM, passive SOM (vary in chemical quality and
turnover time) - portion of the decomposition flow lost as heterotrophic
respiration

Bonan - Figure 17.2: structure of a typical biogeochemical model -
equations 17.1 -- 17.10

Additional details? - maintenance respiration and growth respiration -
storage pool of nonstructural carbohydrates - some models separate wood
into live stems (sapwood) and dead stems, roots into fine roots and
coarse roots, and coarse roots into live pools and dead pools to account
for the different physiological functioning of these biomass components

\subsection{Allocation and turnover
parameterization}\label{allocation-and-turnover-parameterization}

\begin{itemize}
\item
  types of allocation models (see Campioli et al 2013 and work of
  Fatichi et al. )
\item
  allocation parameters
\item
  fixed allocation and dynamic allocation (specified by biome or based
  on environmental conditions)
\item
  Optimality models: plants optimally allocate resources to balance
  light acquisition (foliage), structural support and water transport
  (stems), and water and nutrient uptake (roots).
\item
  allocation based on scaling relationships among plant components
  (specified ratios of foliage, root, and wood biomass)
\item
  Turnover rates vary depending on plant material and are specified as a
  fraction of biomass.
\item
  Turnover rates are commonly estimated as the inverse of residence time
  or longevity
\item
  biogeochemical models can be applied to any type of ecosystem such as
  grassland, savanna, forest, shrubland, and tundra
\end{itemize}

\section{Nutrient cycle models: soil biogeochemical
models}\label{nutrient-cycle-models-soil-biogeochemical-models}

\subsection{Nitrogen cycle}\label{nitrogen-cycle}

\begin{itemize}
\item
  Bonan Chapter 17.6 Nitrogen Cycle
\item
  Bonan Figure 17.8: Depiction of the nitrogen cycle
\item
  only \textasciitilde{}recently added in most biogeochemical models
\item
  closely coupled to carbon cycle
\item
  important role to limit plant productivity
\item
  similar to carbon with an associated nitrogen pool and transfer.
\item
  cycling of nitrogen can be represented by a system of linear
  differential equations similar to that for carbon.
\item
  allocation of plant nitrogen uptake up to plant pools
\item
  loss of nitrogen in litterfall + portion is reabsorbed
\item
  soil nitrogen cycle is more complex (various forms)
\item
  decomposition of litter and soil organic matter, (mineralization and
  immobilization)
\item
  nitrification, denitrification, leaching, ammonia volatilization
\item
  additional inputs from biological nitrogen fixation, atmospheric
  deposition, and fertilizer
\item
  some examples of models? CLM?
\item
  All models simulate a decrease in plant growth when soil mineral
  nitrogen is insufficient to meet demand, but they differ in the manner
  in which this is implemented.
\item
  maybe also more soil oriented model (CENTURY , \ldots{}.)
\item
  discuss different approaches?
\end{itemize}

\subsection{Phosphorus cycle}\label{phosphorus-cycle}

Not in Bonan - Some models additionally include phosphorus (Wang et al.
2010; Yang et al. 2014; Goll et al. 2017) - ORCHIDEE, CLM-CNP

\subsection{Other nutrients}\label{other-nutrients}

\begin{itemize}
\tightlist
\item
  K and Mg in Eucalyptus and other tropical plantations
\end{itemize}

\section{Water balance}\label{water-balance}

\begin{itemize}
\item
  focus on the surface energy balance and vertical water movement in the
  soil--plant atmosphere system (e.g., soil moisture control of
  evapotranspiration)
\item
  Specific components in terrestrial biosphere models:\\
  Interception, throughfall, stemflow, infiltration, surface runoff,
  soil water redistribution, subsurface runoff, snow melt, evaporation,
  transpiration, plant water uptake, stomatal conductance
\end{itemize}

\subsection{A bucket model hydrologic
cycle}\label{a-bucket-model-hydrologic-cycle}

\begin{itemize}
\tightlist
\item
  refer to hydrology courses in the program
\item
  change in soil water is the difference between precipitation and
  evapotranspiration, excess runs off
\item
  based on maximum water-holding capacity
\end{itemize}

\chapter{Representing biodiversity in vegetation
models}\label{representing-biodiversity-in-vegetation-models}

\chaptermark{Biodiversity}

\section{Why and how representing biodiversity in vegetation
models?}\label{why-and-how-representing-biodiversity-in-vegetation-models}

We can start with the applications? I think it is more interesting than
finishing with the applications. Application: Conservation, ecosystem
resilience, vegetation-atmosphere feedbacks

Biodiversity refers here to functional diversity.

\section{Functional diversity}\label{functional-diversity}

\begin{itemize}
\tightlist
\item
  check the book " Terrestrial Ecosystems in a changing world" from P.
  Canadell (2007)
\item
  Part C: Landscapes under changing disturbance regimes:

  \begin{itemize}
  \tightlist
  \item
    PFT
  \item
    Fire and disturbances
  \item
    upscalling
  \item
    construction, evaluation and examples of DGVM applications
  \end{itemize}
\end{itemize}

\subsection{Definition of funcional diversity and plant functional
traits}\label{definition-of-funcional-diversity-and-plant-functional-traits}

Any morphological, physiological or phenological feature measurable at
the individual level, from the cell to the whole-organism level, without
reference to the environment or any other level of organization. It is
functional if it affects fitness indirectly via its effects on growth,
reproduction and survival.

\begin{itemize}
\tightlist
\item
  Seminal papers from the plant functional trait community Violle 2007,
  Lavorel, garnier, Shipley, etc\ldots{}
\end{itemize}

\includegraphics{figures/Violle2007.jpg} - Mention here the interesting
summer school:
\url{http://www.cef-cfr.ca/index.php?n=MEmbres.AlisonMunsonPlantTraits?userlang=en}
- List of reference papers in the link above - 3 types of traits:
dynamic, response \& constant, that are linked to the processes we
studied in the previous chapter (slow/fast processes)

\subsection{Representing 400 000 plant species in a single model: the
Plant Functional Type
approach}\label{representing-400-000-plant-species-in-a-single-model-the-plant-functional-type-approach}

Short description in UCL 4.3.2\\
No description in Bonan

\begin{itemize}
\tightlist
\item
  Lack of observations for every species
\item
  Computing resources problem (refers to the history of DGVMs from the
  introduction)
\item
  A simplification based on biome description and plant functionning at
  the ecosystem level
\item
  Different definitions of PFT: statistical classification, etc.
\item
  Table of classical PFTs used in models here.
\item
  PFT mapping: multi-obs approach based on remote sensing.
\item
  First use of PFTs managed to reproduce well the gradients at the
  global scale, but now it is unsuficcient.
\end{itemize}

\subsection{Limits of the PFT representation in the context of global
change}\label{limits-of-the-pft-representation-in-the-context-of-global-change}

\begin{itemize}
\tightlist
\item
  Including acclimation and adaptation processes
\item
  Dynamic vegetation: Accounting for non-random species turnover
\item
  Quantifying vegetation-environment feedbacks
\item
  Quantifying impacts of biodiversity on ecosytem functioning and
  climate
\end{itemize}

\subsection{From model parameters to plant
traits}\label{from-model-parameters-to-plant-traits}

\begin{itemize}
\tightlist
\item
  Reconciliating modelling with functional ecology.
\item
  Existing databases (TRY)
\item
  Empirical approach: More PFTs with traits instead of model-specific
  parameters, trait-trait, trait-environment relationships
\item
  Trade-offs: modeling plant strategies --\textgreater{} LES, PES, RES,
  all the ES :D
\item
  Role of data assimilation in regions without data and to assess
  spatial variability of vegetation properties
\item
  But: requires lots of observations in space and time.
\end{itemize}

\subsection{Eco-evolutive optimality
approaches}\label{eco-evolutive-optimality-approaches}

\begin{itemize}
\tightlist
\item
  New generation of models
\item
  PPA, coordination, ect\ldots{}
\item
  Paper from Oskar.
\end{itemize}

\section{Competition models}\label{competition-models}

Not in Bonan nor UCL\\
In Bonan Chapter 19 on demography, gap models, etc\ldots{} which is a
part of competition.

\subsection{Representation of PFTs in vegetation
models}\label{representation-of-pfts-in-vegetation-models}

\begin{itemize}
\tightlist
\item
  Parameterization and calibration of PFTs--\textgreater{} data
  assimilation, traits, model-specific parameters
\item
  representation by pixels
\item
  shared processes, different processes
\item
  interaction between PFTs
\item
  Depend on the model: individual/cohort/big leaf
\end{itemize}

\subsection{Competition for ressources / Plant
strategy?}\label{competition-for-ressources-plant-strategy}

\begin{itemize}
\tightlist
\item
  In fact we can extend the trait based approach and plant strategy
  (PES, etc) in the competition and community section?
\item
  Mortality, turnover, etc..
\end{itemize}

\subsection{Representation of trait
distributions}\label{representation-of-trait-distributions}

\begin{itemize}
\tightlist
\item
  Trade-offs
\end{itemize}

\section{Communities}\label{communities}

\begin{itemize}
\tightlist
\item
  Successions and impact on cycles, species composition etc.
\end{itemize}

\section{What about crops?}\label{what-about-crops}

\begin{itemize}
\tightlist
\item
  Not our focus but we don't forget it. A few words to say that specific
  crop models exists
\item
  Diversity is not a problem anymore
\item
  Plant functional traits are still central to crop modelling, but
  competition and diversity are no more an issue.
\item
  Other problematics specific to agriculture, such as agro-ecosystems
  where we have multi layers of vegetation (Trees over crops)
  --\textgreater{} Very interesting modelling problem and application
  especially in arid/semi-arid and tropical regions
\end{itemize}

\chapter{Modelling vegetation dynamics and
demography}\label{modelling-vegetation-dynamics-and-demography}

\chaptermark{Dynamics}

Bonan Chapter 19.2 Another class of models, known as individual plant or
ecosystem demography models, retains the complexity of individual plants
or cohorts of similar plants. In these models, ecosystem properties such
as carbon storage are the outcome of demographic processes.

\begin{itemize}
\tightlist
\item
  plant populations
\item
  community composition
\item
  ecosystem structure
\item
  driven by demographic processes of recruitment, establishment, growth,
  and mortality
\end{itemize}

\section{Gap models, individual and cohort based
models}\label{gap-models-individual-and-cohort-based-models}

\begin{itemize}
\item
  small scale models; landscape represented as a mosaic of hundreds of
  independent forest patches, each of which can differ in species
  composition and stage of development in response to disturbance that
  creates an opening in the canopy
\item
  models track the establishment, growth, and death of individual trees
  in an area of land.
\item
  Each tree is characterized by its species, stem diameter, height, and
  age.
\item
  trees compete for light, soil moisture, and nutrients.
\item
  patch undergoes temporal changes in the density, size, and composition
  of trees with the formation of a gap in the canopy
\item
  Community composition, biomass, productivity, and biogeochemical
  cycles are emergent outcomes of individual trees interacting among
  themselves and with the environment to acquire the resources necessary
  for growth and survival
\item
  cohort-based models define patches based on age since disturbance and
  simulate the dynamics of cohorts of similar plant functional types
  rather than tracking every individual.
\item
  Common to each model is the representation of vegetation demography,
  with age- and size-dependent growth and mortality and in which growth
  is constrained by allometric relationships of stem diameter with
  height, sapwood area, leaf area, and biomass. cohort models
  --\textgreater{} modelling size distributions
\end{itemize}

\section{Allometric relationships}\label{allometric-relationships}

\begin{itemize}
\tightlist
\item
  link with growth modelling of previuous chapter
\item
  allometric relationships are a critical driver of individual tree
  growth.
\item
  Height is important for its effect on stem diameter increment, both
  directly through tree volume growth and indirectly through shading.
\item
  Biomass allocation: empirical equations that constrain foliage, stem,
  and root mass for a given size tree
\item
  relationship between stem diameter and leaf area drives light
  extinction in the canopy
\item
  annual growth of a tree is calculated from its diameter and height as
  modified by light, climate, and site conditions. Growth curves figure
  19.5
\end{itemize}

\section{Competition for light}\label{competition-for-light}

\begin{itemize}
\tightlist
\item
  critical driver of forest dynamics
\item
  shading of smaller individuals by taller trees
\item
  vertical profile of leaf area in the patch (vertical structure in
  which trees are arranged into canopy layers)
\item
  height of a tree determines its location in the cumulative leaf area
  profile
\item
  light extinction coefficient
\item
  figure 19.6: representation of plant canopies
\end{itemize}

\section{Seed dispersal and
recruitment}\label{seed-dispersal-and-recruitment}

\begin{itemize}
\item
  regeneration: stochastic process
\item
  seeds of species are assumed to be present on-site
\item
  available light at the forest floor, climate tolerances, and other
  site conditions determine which species become established.
\item
  sprouting based on size
\item
  Species are characterized by life history characteristics + maybe add
  example of herb layer models of FORNALAB
\end{itemize}

\section{Mortality}\label{mortality}

\begin{itemize}
\item
  stochastic process
\item
  Trees die with a constant probability each year
\item
  The probability of mortality increases when tree growth is less than
  some minimum
\item
  disturbance related mortality : Wildfire and insect outbreaks can be
  included
\item
  The occurrence of fire is treated stochastically with an annual
  probability of burning. An individual patch may, for example, have a
  1\% change of burning in any given year.
\end{itemize}

\part{Upscaling and
applications}\label{part-upscaling-and-applications}

\chapter{Spatial heterogeneity, landscape scale,
metapopulations}\label{spatial-heterogeneity-landscape-scale-metapopulations}

\chaptermark{Heterogeneity}

\section{Patch dynamics}\label{patch-dynamics}

Some references: - Book: The ecology of natual disturance and patch
dynamics, Pickett \& White, 2013

\subsection{Spatial heterogeneity:
Definitions}\label{spatial-heterogeneity-definitions}

\begin{itemize}
\tightlist
\item
  Definition of Patch Dynamics, Perturbation, Disturbance:
\item
  Spatial heterogeneity
\item
  Resilience and shifts
\end{itemize}

\subsection{Impact of heterogeneity on ecosystem fonctionning and
environmental
feedbacks}\label{impact-of-heterogeneity-on-ecosystem-fonctionning-and-environmental-feedbacks}

\begin{itemize}
\tightlist
\item
  Show examples here of the impact of heterogeneity
\item
  Application in the design of nature reserves for example
\end{itemize}

\subsection{Heterogeneity is a matter of
resolution}\label{heterogeneity-is-a-matter-of-resolution}

\begin{itemize}
\tightlist
\item
  Imbricated levels of heterogeneity depending on spatial and temporal
  resolution
\item
  Heterogeneity is also a matter of the studied question: important in
  term of modelling since it will govern how processes are implemented
\end{itemize}

\subsection{Representation in Vegetation models: what are the drivers of
spatial
heterogeneity?}\label{representation-in-vegetation-models-what-are-the-drivers-of-spatial-heterogeneity}

\begin{itemize}
\tightlist
\item
  List here the different drivers
\item
  Heterogeneity is a patchwork of homogeneity in most models
\item
  But we can still represent dynamics in heterogeneity --\textgreater{}
  mortality, growth and shifts in species composition
\end{itemize}

\subsection{Disturbances and Patch
dynamics}\label{disturbances-and-patch-dynamics}

\begin{itemize}
\tightlist
\item
  We listed the different drivers above, we will now discuss in detail
  the most important aspects affecting patch dynamics
\item
  Link to land-use and disturbance
\end{itemize}

\section{Land-use changes}\label{land-use-changes}

\begin{itemize}
\tightlist
\item
  Land use is linked to spatial heterogeneity and patch dynamics
\end{itemize}

\subsection{Role of Land-use in global emissions and biogeochemicel
cycles}\label{role-of-land-use-in-global-emissions-and-biogeochemicel-cycles}

\begin{itemize}
\tightlist
\item
  impact C stocks and fluxes
\item
  impact on nutrients (depletion over rotations, etc\ldots{})
\item
  important impact on respiration
\item
  vegetation cover and biophysical impact: albedo, etc\ldots{}
\item
  Specific case of deforestation, one of the most important imapct (make
  a paragraph on that?)
\item
  How are fluxes attributed to land use in gas emission assessments?
  --\textgreater{} central role of vegetation modeling
\end{itemize}

\subsection{The important role of land use in the water
cycle}\label{the-important-role-of-land-use-in-the-water-cycle}

\begin{itemize}
\tightlist
\item
  Affects regional precipitations
\item
  Affects water routing --\textgreater{} Compared to the local impact on
  vegetation, here we touch something that will have an impact for the
  surrounding regions
\end{itemize}

\subsection{Monitoring land-use}\label{monitoring-land-use}

\begin{itemize}
\tightlist
\item
  remote sensing, rapid link to other courses
\end{itemize}

\subsection{How Land-use is represented in vegetation
models?}\label{how-land-use-is-represented-in-vegetation-models}

\begin{itemize}
\tightlist
\item
  Compared to vegetation dynamic which is process-based, here land use
  is imposed.
\item
  management
\item
  urban areas
\end{itemize}

\section{Natural and Anthropogenic
disturbances}\label{natural-and-anthropogenic-disturbances}

\begin{itemize}
\tightlist
\item
  We provide an overview of disturbances but we will detail only one of
  each: Fires and Management
\end{itemize}

\subsection{Wind and extrem events}\label{wind-and-extrem-events}

\begin{itemize}
\tightlist
\item
  Modelling storms
\item
  Modelling heat and cold waves, frost impact
\end{itemize}

\subsection{Herbivory}\label{herbivory}

\begin{itemize}
\tightlist
\item
  Yes herbivory is represented in vegetation models :D
\item
  Palability traits/ fixed fraction/ insects
\end{itemize}

\subsection{Modelling fires}\label{modelling-fires}

\begin{itemize}
\tightlist
\item
  In UCL Practical chap. 6
\item
  For estimating the impact on ecosystems
\item
  To be able to predict fires
\item
  Observation of fires and quantifications of fluxes
\item
  Fires and deposition
\item
  Aerosols
\item
  Modelling ``fire'' traits, drought and temperature stress in models to
  simulate fires
\end{itemize}

\subsection{Human activity: Management and urban
areas}\label{human-activity-management-and-urban-areas}

\begin{itemize}
\tightlist
\item
  Forest management: existing models, representation of forestry and use
  of models
\item
  Fertilization and irrigation in vegetation models
\item
  Urban areas in vegetation models
\item
  Concrete application: Paper of Luyssaert: forest management in Europe
  did not help in mitigating climate change.
\end{itemize}

\subsection{The specific case of CO2 and temperature
increase}\label{the-specific-case-of-co2-and-temperature-increase}

\begin{itemize}
\tightlist
\item
  conclude the chapter here by refering to climate change, one of the
  biggest ``Continuous'' distrubance compared to previous ``discrete''
  disturbances
\item
  Simulating acclimation and adaptation
\item
  refers to chapter 2 for acclimation of processes
\item
  refers to chapter 11 for scenarios
\end{itemize}

\chapter{Upscaling from the leaf to the
globe}\label{upscaling-from-the-leaf-to-the-globe}

\chaptermark{Globe} Some references: - Scalling processes and problems,
Jarvis 1995 - upscalling in global change research, Harvey 2000

\section{Spatial and temporal non-linearities: Cascading effect in the
Earth
system}\label{spatial-and-temporal-non-linearities-cascading-effect-in-the-earth-system}

\begin{itemize}
\tightlist
\item
  spatial upscalling
\item
  temporal upscalling
\item
  classification of upscalling problems:
\item
  Spatial variability + process nonlinearity
\item
  Minimim scale to observe the process
\item
  Different processes dominate at different scales
\item
  Feedbacks between scales
\item
  Development of emergent properties
\item
  Edge effects
\item
  Temporal lag dependent on spatial scale change
\item
  Collective response with differential effects
\item
  Solutions to upscaling problems:
\item
  Ignore (easy solution)
\item
  Increase model resolution (now more and more possible thanks to
  computing ressources, and data assimilation)
\item
  etc\ldots{} ** Nice review in Harvey 2000 **
\end{itemize}

--\textgreater{} Solution depends on the application, show some examples
here

\section{Land surface models}\label{land-surface-models}

\begin{itemize}
\tightlist
\item
  Dependence to other disciplines (Biology, ecology, physics, chemistry
  (VOC, etc), hydrology, pedology, datascience and mathematics,
  etc\ldots{})
\item
  Figure 1.7 from Bonan
\item
  UCL 4.2: Land surface schemes
\item
  Focus on the coupling of different models and what it implies, not the
  technical aspects \#\#\# Soil-Vegetation-Atmosphere-Transer models
\item
  Description of SVAT models, regroups what we studied in Chap1-9
\end{itemize}

\section{DVGMs as a part of Earth System
Models}\label{dvgms-as-a-part-of-earth-system-models}

\begin{itemize}
\tightlist
\item
  Partially in UCL 4.3.3
\end{itemize}

\subsection{One Biosphere}\label{one-biosphere}

\begin{itemize}
\tightlist
\item
  Chapter 1 of Bonan, specifically 1.5
\item
  Coupling to other components
\end{itemize}

\subsection{Atmosphere, Ocean, lakes and urban
areas}\label{atmosphere-ocean-lakes-and-urban-areas}

\begin{itemize}
\tightlist
\item
  Rapid desciption of other models
\item
  Reference here to previous chapter on heterogeneity
\end{itemize}

\subsection{Coupling of processes with different time steps and regional
scale}\label{coupling-of-processes-with-different-time-steps-and-regional-scale}

\subsection{Simulating feedbacks}\label{simulating-feedbacks}

\begin{itemize}
\tightlist
\item
  Nice transition to chap 11 with future scenarii
\end{itemize}

\chapter{Model projections and scenario
analysis}\label{model-projections-and-scenario-analysis}

\chaptermark{Projections}

\section{Climate scenarios}\label{climate-scenarios}

\subsection{Representative Concentration Pathway (RCP
scenarios)}\label{representative-concentration-pathway-rcp-scenarios}

\begin{itemize}
\tightlist
\item
  How scenarios are defined
\item
  How current emissions are measured and attributed to different
  factors?
\end{itemize}

\subsection{Different models, different
RCP}\label{different-models-different-rcp}

\begin{itemize}
\tightlist
\item
  The central role of ESM: coupling to Atmosphere and Ocean and
  feedbacks
\item
  Here refers to preivous Chapter 10
\item
  list some examples and differences: IPSL, HadGEM, etc\ldots{}
\end{itemize}

\subsection{Use of RCP in vegetation
modelling}\label{use-of-rcp-in-vegetation-modelling}

\begin{itemize}
\tightlist
\item
  ENSEMBLE simulations
\item
  IPCC
\item
  Example of applications
\end{itemize}

\subsection{How can we evaluate future
scenarios?}\label{how-can-we-evaluate-future-scenarios}

\begin{itemize}
\tightlist
\item
  FACE
\item
  Rainfall exclusion experiments
\item
  Natural gradient (Iceland and soil temperature based on volcano and
  geothermy)
\end{itemize}

\subsection{The central role of Paleo studies and historical
datasets.}\label{the-central-role-of-paleo-studies-and-historical-datasets.}

\begin{itemize}
\tightlist
\item
  Good performance for past and current conditions is mandatory to
  evaluate future scenarios
\item
  Here remind the central role of experiments and monitoring
\end{itemize}

\section{Land-use scenarios}\label{land-use-scenarios}

\subsection{Construction of Land-use
scenarios}\label{construction-of-land-use-scenarios}

\begin{itemize}
\tightlist
\item
  Hyde for historical land-use,
  \url{https://themasites.pbl.nl/tridion/en/themasites/hyde/}
\item
  Scenario for future land-use
\end{itemize}

--\textgreater{} We can follow the same structure as for RCP?

\subsection{How can we evaluate land use
scenarios?}\label{how-can-we-evaluate-land-use-scenarios}

\begin{itemize}
\tightlist
\item
  Based on historical data
\item
  Remote sensing
\end{itemize}

\section{Management scenarios}\label{management-scenarios}

\subsection{Construction of Land-use
scenarios}\label{construction-of-land-use-scenarios-1}

\subsection{How can we evaluate management
scenarios?}\label{how-can-we-evaluate-management-scenarios}

\section{Some concrete applications of vegetation
models}\label{some-concrete-applications-of-vegetation-models}

\begin{itemize}
\tightlist
\item
  As a conclusion of the whole course I see a nice diagram that we
  constructed throughout the course with small boxes added to each
  others and we link that to all the possible application
\end{itemize}

\part{Appendix}\label{part-appendix}

\chapter*{Contributing to this
document}\label{contributing-to-this-document}
\addcontentsline{toc}{chapter}{Contributing to this document}

\section*{First steps}\label{first-steps}
\addcontentsline{toc}{section}{First steps}

First, visit the course webpage on
\url{https://github.com/femeunier/VegMod_course}, and fork it to your
own github account. Open a RStudio session and (if it is your first time
with git) introduce yourself:

\begin{Shaded}
\begin{Highlighting}[]
\FunctionTok{git}\NormalTok{ config --global user.name }\StringTok{"FULLNAME"}
\FunctionTok{git}\NormalTok{ config --global user.email you@yourdomain.example.com}
\end{Highlighting}
\end{Shaded}

Note that you can do every single step below using the terminal and the
git tabs in RStudio. Clone the newly forked folder to your local
machine:

\begin{Shaded}
\begin{Highlighting}[]
\FunctionTok{git}\NormalTok{ clone https://github.com/femeunier/VegMod_course.git}
\end{Highlighting}
\end{Shaded}

or using SSH (to set up it first, see for instance
\url{https://help.github.com/en/github/authenticating-to-github/connecting-to-github-with-ssh})

\begin{Shaded}
\begin{Highlighting}[]
\FunctionTok{git}\NormalTok{ clone git@github.com:femeunier/VegMod_course.git}
\end{Highlighting}
\end{Shaded}

Define upstream

\begin{Shaded}
\begin{Highlighting}[]
\BuiltInTok{cd}\NormalTok{ VegMod_course}
\FunctionTok{git}\NormalTok{ remote add upstream git@github.com:femeunier/VegMod_course.git}
\end{Highlighting}
\end{Shaded}

\section*{New pull request}\label{new-pull-request}
\addcontentsline{toc}{section}{New pull request}

Get the latest code from the main repository

\begin{Shaded}
\begin{Highlighting}[]
\FunctionTok{git}\NormalTok{ pull upstream master}
\end{Highlighting}
\end{Shaded}

Create a new branch (here new\_branch is the new branch's name)

\begin{Shaded}
\begin{Highlighting}[]
\FunctionTok{git}\NormalTok{ checkout -b new_branch}
\end{Highlighting}
\end{Shaded}

Do some coding, add files and commit them

\begin{Shaded}
\begin{Highlighting}[]
\FunctionTok{git}\NormalTok{ add filepath}
\FunctionTok{git}\NormalTok{ commit -m “Message”}
\end{Highlighting}
\end{Shaded}

Push your changes to your github (when a feature is working, a set of
bugs are fixed, or you need to share progress with others).

\begin{Shaded}
\begin{Highlighting}[]
\FunctionTok{git}\NormalTok{ push origin new_branch}
\end{Highlighting}
\end{Shaded}

Before submitting code back to the main repository, make sure that book
compiles (buikd book). Open the PR online by visiting your github
repository. To ease those previous steps you can take advantage of the
git GUI in RStudio. To do so, create a new project from an existing
directory.

\chapter*{Supporting material}\label{supporting-material}
\addcontentsline{toc}{chapter}{Supporting material}

Crash course, basic programming (R), theory about model evaluation etc.

\part{Practicals}\label{part-practicals}

\chapter*{Practical A}\label{practical-a}
\addcontentsline{toc}{chapter}{Practical A}

PC-room, supervised exercise

Simple model on diurnal variation in solar angle, radiation extinction
and photosynthesis in vegetation types with different and canopy
structure and LAI: grassland, broadleaved forest, coniferous forest

Scale: aggregated stand level (big leaf model)

Methodological focus: model formulation: translating a few equations
into code

Methodological focus: compiling code, running model, reading
input-output

\chapter*{Practical B}\label{practical-b}
\addcontentsline{toc}{chapter}{Practical B}

Group work, report, PC room

Modelling diurnal cycle of carbon and water fluxes for flux tower sites
(Savanna's Sahel)

Scale: aggregated stand level

Methodological focus: model-data comparison (goodness-of-fit), simple
parameter optimisation

\chapter*{Practical C}\label{practical-c}
\addcontentsline{toc}{chapter}{Practical C}

PC-room, supervised exercise

Modelling the size structure of a temperate forest (stand diameter
distribution)

Scale: forest stand

Methodological focus: initial conditions

\chapter*{Practical D}\label{practical-d}
\addcontentsline{toc}{chapter}{Practical D}

Group work, report, PC room

Modelling carbon stocks (above and belowground) and fluxes

Scale: ecosystem

Methodological focus: Spinup and sensitivity analysis (testing which
climate variables have strongest impact on stocks)

\chapter*{Practical E}\label{practical-e}
\addcontentsline{toc}{chapter}{Practical E}

PC-room, supervised exercise

Simulating forest succession, meta-analysis of trait dataset to
prescribe vegetation functional composition (using PEcAn-framework)

Scale: landscape

Methodological focus: parameter meta-analysis (PFT construction), data
assimilation

\chapter*{Practical F}\label{practical-f}
\addcontentsline{toc}{chapter}{Practical F}

PC-room, group work, microteaching

Climate/land use/management scenario analysis

Scale: site/globe? (Pecan framework) each group choses a question and a
model

Methodological focus: sensitivity and uncertainty analysis

\bibliography{book.bib,packages.bib}

\end{document}
