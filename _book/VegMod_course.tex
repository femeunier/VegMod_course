\documentclass[12pt,oneside]{book}
\usepackage{lmodern}
\usepackage{amssymb,amsmath}
\usepackage{ifxetex,ifluatex}
\usepackage{fixltx2e} % provides \textsubscript
\ifnum 0\ifxetex 1\fi\ifluatex 1\fi=0 % if pdftex
  \usepackage[T1]{fontenc}
  \usepackage[utf8]{inputenc}
\else % if luatex or xelatex
  \ifxetex
    \usepackage{mathspec}
  \else
    \usepackage{fontspec}
  \fi
  \defaultfontfeatures{Ligatures=TeX,Scale=MatchLowercase}
\fi
% use upquote if available, for straight quotes in verbatim environments
\IfFileExists{upquote.sty}{\usepackage{upquote}}{}
% use microtype if available
\IfFileExists{microtype.sty}{%
\usepackage[]{microtype}
\UseMicrotypeSet[protrusion]{basicmath} % disable protrusion for tt fonts
}{}
\PassOptionsToPackage{hyphens}{url} % url is loaded by hyperref
\usepackage[unicode=true]{hyperref}
\hypersetup{
            pdftitle={Vegetation modelling},
            pdfauthor={Hans Verbeeck, Félicien Meunier, Marc Peaucelle},
            pdfborder={0 0 0},
            breaklinks=true}
\urlstyle{same}  % don't use monospace font for urls
\usepackage[left=3cm,right=3cm,top=2cm,bottom=2cm]{geometry}
\usepackage{natbib}
\bibliographystyle{apalike}
\usepackage{color}
\usepackage{fancyvrb}
\newcommand{\VerbBar}{|}
\newcommand{\VERB}{\Verb[commandchars=\\\{\}]}
\DefineVerbatimEnvironment{Highlighting}{Verbatim}{commandchars=\\\{\}}
% Add ',fontsize=\small' for more characters per line
\usepackage{framed}
\definecolor{shadecolor}{RGB}{248,248,248}
\newenvironment{Shaded}{\begin{snugshade}}{\end{snugshade}}
\newcommand{\KeywordTok}[1]{\textcolor[rgb]{0.13,0.29,0.53}{\textbf{#1}}}
\newcommand{\DataTypeTok}[1]{\textcolor[rgb]{0.13,0.29,0.53}{#1}}
\newcommand{\DecValTok}[1]{\textcolor[rgb]{0.00,0.00,0.81}{#1}}
\newcommand{\BaseNTok}[1]{\textcolor[rgb]{0.00,0.00,0.81}{#1}}
\newcommand{\FloatTok}[1]{\textcolor[rgb]{0.00,0.00,0.81}{#1}}
\newcommand{\ConstantTok}[1]{\textcolor[rgb]{0.00,0.00,0.00}{#1}}
\newcommand{\CharTok}[1]{\textcolor[rgb]{0.31,0.60,0.02}{#1}}
\newcommand{\SpecialCharTok}[1]{\textcolor[rgb]{0.00,0.00,0.00}{#1}}
\newcommand{\StringTok}[1]{\textcolor[rgb]{0.31,0.60,0.02}{#1}}
\newcommand{\VerbatimStringTok}[1]{\textcolor[rgb]{0.31,0.60,0.02}{#1}}
\newcommand{\SpecialStringTok}[1]{\textcolor[rgb]{0.31,0.60,0.02}{#1}}
\newcommand{\ImportTok}[1]{#1}
\newcommand{\CommentTok}[1]{\textcolor[rgb]{0.56,0.35,0.01}{\textit{#1}}}
\newcommand{\DocumentationTok}[1]{\textcolor[rgb]{0.56,0.35,0.01}{\textbf{\textit{#1}}}}
\newcommand{\AnnotationTok}[1]{\textcolor[rgb]{0.56,0.35,0.01}{\textbf{\textit{#1}}}}
\newcommand{\CommentVarTok}[1]{\textcolor[rgb]{0.56,0.35,0.01}{\textbf{\textit{#1}}}}
\newcommand{\OtherTok}[1]{\textcolor[rgb]{0.56,0.35,0.01}{#1}}
\newcommand{\FunctionTok}[1]{\textcolor[rgb]{0.00,0.00,0.00}{#1}}
\newcommand{\VariableTok}[1]{\textcolor[rgb]{0.00,0.00,0.00}{#1}}
\newcommand{\ControlFlowTok}[1]{\textcolor[rgb]{0.13,0.29,0.53}{\textbf{#1}}}
\newcommand{\OperatorTok}[1]{\textcolor[rgb]{0.81,0.36,0.00}{\textbf{#1}}}
\newcommand{\BuiltInTok}[1]{#1}
\newcommand{\ExtensionTok}[1]{#1}
\newcommand{\PreprocessorTok}[1]{\textcolor[rgb]{0.56,0.35,0.01}{\textit{#1}}}
\newcommand{\AttributeTok}[1]{\textcolor[rgb]{0.77,0.63,0.00}{#1}}
\newcommand{\RegionMarkerTok}[1]{#1}
\newcommand{\InformationTok}[1]{\textcolor[rgb]{0.56,0.35,0.01}{\textbf{\textit{#1}}}}
\newcommand{\WarningTok}[1]{\textcolor[rgb]{0.56,0.35,0.01}{\textbf{\textit{#1}}}}
\newcommand{\AlertTok}[1]{\textcolor[rgb]{0.94,0.16,0.16}{#1}}
\newcommand{\ErrorTok}[1]{\textcolor[rgb]{0.64,0.00,0.00}{\textbf{#1}}}
\newcommand{\NormalTok}[1]{#1}
\usepackage{longtable,booktabs}
% Fix footnotes in tables (requires footnote package)
\IfFileExists{footnote.sty}{\usepackage{footnote}\makesavenoteenv{long table}}{}
\usepackage{graphicx,grffile}
\makeatletter
\def\maxwidth{\ifdim\Gin@nat@width>\linewidth\linewidth\else\Gin@nat@width\fi}
\def\maxheight{\ifdim\Gin@nat@height>\textheight\textheight\else\Gin@nat@height\fi}
\makeatother
% Scale images if necessary, so that they will not overflow the page
% margins by default, and it is still possible to overwrite the defaults
% using explicit options in \includegraphics[width, height, ...]{}
\setkeys{Gin}{width=\maxwidth,height=\maxheight,keepaspectratio}
\IfFileExists{parskip.sty}{%
\usepackage{parskip}
}{% else
\setlength{\parindent}{0pt}
\setlength{\parskip}{6pt plus 2pt minus 1pt}
}
\setlength{\emergencystretch}{3em}  % prevent overfull lines
\providecommand{\tightlist}{%
  \setlength{\itemsep}{0pt}\setlength{\parskip}{0pt}}
\setcounter{secnumdepth}{5}
% Redefines (sub)paragraphs to behave more like sections
\ifx\paragraph\undefined\else
\let\oldparagraph\paragraph
\renewcommand{\paragraph}[1]{\oldparagraph{#1}\mbox{}}
\fi
\ifx\subparagraph\undefined\else
\let\oldsubparagraph\subparagraph
\renewcommand{\subparagraph}[1]{\oldsubparagraph{#1}\mbox{}}
\fi

% set default figure placement to htbp
\makeatletter
\def\fps@figure{htbp}
\makeatother

\usepackage{booktabs}
\usepackage{fancyhdr}

\AtBeginDocument{\let\maketitle\relax} % To relax 

% Remove page number on page parts
\usepackage{etoolbox}
\patchcmd{\part}{\thispagestyle{plain}}{\thispagestyle{empty}}{}{}

% Header and font 
\usepackage{fancyhdr}
\usepackage{float}

\usepackage{amsmath,color}

\pagestyle{fancy}
\fancyhf{} % sets both header and footer to nothing
\renewcommand{\headrulewidth}{0pt} % Remove line

\fancyhead[L,C,R]{} % Empty header
\fancyfoot[C]{\thepage} % Footer, center, page number
\fancyfoot[L,R]{} % Empty footer on left and right

\floatplacement{figure}{H}

\usepackage[makeroom]{cancel}
\usepackage{caption}

\title{Vegetation modelling}
\author{Hans Verbeeck, Félicien Meunier, Marc Peaucelle}
\date{2021-05-04}

\begin{document}
\maketitle

\newcommand{\plogo}{\fbox{$\mathcal{PL}$}} % Generic dummy publisher logo
\frontmatter


\begin{titlepage} % Suppresses headers and footers on the title page

	\centering % Centre everything on the title page
	
	\scshape % Use small caps for all text on the title page
	
	\vspace*{\baselineskip} % White space at the top of the page
	
	%------------------------------------------------
	%	Title
	%------------------------------------------------
	
	\vspace{12\baselineskip}
	
	\rule{\textwidth}{1.6pt}\vspace*{-\baselineskip}\vspace*{2pt} % Thick horizontal rule
	\rule{\textwidth}{0.4pt} % Thin horizontal rule
	
	\vspace{0.75\baselineskip} % Whitespace above the title
	
	{\LARGE Vegetation modelling\\} % Title
	
	\vspace{0.75\baselineskip} % Whitespace below the title
	
	\rule{\textwidth}{0.4pt}\vspace*{-\baselineskip}\vspace{3.2pt} % Thin horizontal rule
	\rule{\textwidth}{1.6pt} % Thick horizontal rule
	
	\vspace{2\baselineskip} % Whitespace after the title block
	
	%------------------------------------------------
	%	Subtitle
	%------------------------------------------------
	
	Syllabus % Subtitle or further description
	
	\vspace*{3\baselineskip} % Whitespace under the subtitle
	
	%------------------------------------------------
	%	Editor(s)
	%------------------------------------------------
	
	Written By
	
	\vspace{0.5\baselineskip} % Whitespace before the editors
	
	{\scshape Hans Verbeeck, Felicien Meunier, Marc Peaucelle \\} % Editor list
	
	\vspace{0.5\baselineskip} % Whitespace below the editor list
	
	\textit{Ghent University \\} % Editor affiliation
	
	\vfill % Whitespace between editor names and publisher logo
	
	%------------------------------------------------
	%	Publisher
	%------------------------------------------------
	
	%\plogo % Publisher logo
	
	\includegraphics[width = 50mm]{figures/UGhent2.png}
	
	\vspace{0.3\baselineskip} % Whitespace under the publisher logo
	
	2021 % Publication year
	
	%{\large publisher} % Publisher

\end{titlepage}

{
\setcounter{tocdepth}{1}
\tableofcontents
}
\mainmatter

\chapter{Introduction}\label{intro}

An ecosystem model is an abstract, usually mathematical, representation
of an ecological system which is studied to better understand the real
system. The scale varies from an individual population to an ecological
community or an entire biome.

A dynamic global vegetation model (DGVM) is a computer program that
simulates shifts in potential vegetation and its associated
biogeochemical and hydrological cycles as a response to shifts in
climate (see 1.5). Vegetation models can be used to conduct virtual
experiments.

\section{The central role of vegetation in the Earth
system}\label{the-central-role-of-vegetation-in-the-earth-system}

Plants and vegetation play an essential role in the earth system. In
Figure \ref{fig:f1}, we have represented the planetary boundaries and
the significant environmental challenges we are facing. For some of
them, we are reaching the system's boundary, such as genetic diversity,
biogeochemical cycling, and climate change. It is crucial to understand
how vegetation reacts to environmental problems (positive or negative
feedback), as vegetation plays a central role in many of these
boundaries. Environmental challenges we are facing include:

\begin{itemize}
\item
  Stratospheric ozone depletion: The stratospheric ozone layer in the
  atmosphere filters out ultraviolet (UV) radiation from the sun. If
  this layer decreases, increasing amounts of UV radiation will reach
  ground level. Because of the actions taken as a result of the Montreal
  Protocol, we appear to be on the path that will allow us to stay
  within this boundary.
\item
  Atmospheric aerosol loading: Through their interaction with water
  vapour, aerosol play a critically important role in the hydrological
  cycle affecting cloud formation and global-scale and regional patterns
  of atmospheric circulation, such as the monsoon systems in tropical
  regions. They also have a direct effect on climate, by changing how
  much solar radiation is reflected or absorbed in the atmosphere.
  Humans change the aerosol loading by emitting atmospheric pollution
  and through land-use change that increases the release of dust and
  smoke into the air. Shifts in climate regimes and monsoon systems have
  already been seen in highly polluted environments, giving a
  quantifiable regional measure for an aerosol boundary.
\item
  Ocean acidification: Around a quarter of the CO2 that humanity emits
  into the atmosphere is ultimately dissolved in the oceans. Here it
  forms carbonic acid, altering ocean chemistry and decreasing the pH of
  the surface water. This increased acidity reduces the amount of
  available carbonate ions, an essential `building block' used by many
  marine species for shell and skeleton formation. Beyond a threshold
  concentration, this rising acidity makes it hard for organisms such as
  corals and some shellfish and plankton species to grow and survive.
  Losses of these species would change the structure and dynamics of
  ocean ecosystems and could potentially lead to drastic reductions in
  fish stocks. Compared to pre-industrial times, surface ocean acidity
  has already increased by 30 percent. The ocean acidification boundary
  has ramifications for the whole planet.
\item
  Biochemical flows: Nitrogen and phosphorus are both essential elements
  for plant growth, so fertilizer production and application is the main
  concern. Human activities now convert more atmospheric nitrogen into
  reactive forms than all of the Earth's terrestrial processes combined.
  Much of this new reactive nitrogen is emitted to the atmosphere in
  various forms rather than taken up by crops. When it is rained out, it
  pollutes waterways and coastal zones or accumulates in the terrestrial
  biosphere. Similarly, a relatively small proportion of phosphorus
  fertilizers applied to food production systems is taken up by plants;
  much of the phosphorus mobilized by humans also ends up in aquatic
  systems. These can become oxygen-starved as bacteria consume the
  blooms of algae that grow in response to the high nutrient supply. A
  significant fraction of the applied nitrogen and phosphorus makes its
  way to the sea, and can push marine and aquatic systems across
  ecological thresholds of their own. One regional-scale example of this
  effect is the decline in the shrimp catch in the Gulf of Mexico's
  `dead zone' caused by fertilizer transported in rivers from the US
  Midwest.
\item
  Freshwater use: Human pressure is now the dominant driving force
  determining the functioning and distribution of global freshwater
  systems. The consequences of human modification of water bodies
  include both global-scale river flow changes and shifts in vapour
  flows arising from land use change. These shifts in the hydrological
  system can be abrupt and irreversible. Water is becoming increasingly
  scarce - by 2050 about half a billion people are likely to be subject
  to water-stress, increasing the pressure to intervene in water
  systems.
\item
  Land-system change: Forests, grasslands, wetlands and other vegetation
  types have primarily been converted to agricultural land. This
  land-use change is one driving force behind the serious reductions in
  biodiversity, and it has impacts on water flows and on the
  biogeochemical cycling of carbon, nitrogen and phosphorus and other
  important elements. While each incident of land cover change occurs on
  a local scale, the aggregated impacts can have consequences for Earth
  system processes on a global scale. Forests play a particularly
  important role in controlling the linked dynamics of land use and
  climate, and is the focus of the boundary for land system change.
\item
  Biosphere integrity: The Millennium Ecosystem Assessment of 2005
  concluded that changes to ecosystems due to human activities were more
  rapid in the past 50 years than at any time in human history,
  increasing the risks of abrupt and irreversible changes. The main
  drivers of change are the demand for food, water, and natural
  resources, causing severe biodiversity loss and leading to changes in
  ecosystem services. The current high rates of ecosystem damage and
  extinction can be slowed by efforts to protect the integrity of living
  systems (the biosphere), enhancing habitat, and improving connectivity
  between ecosystems while maintaining the high agricultural
  productivity that humanity needs.
\item
  Climate change: Recent evidence suggests that the Earth, now passing
  390 ppmv CO2 in the atmosphere, has already transgressed the planetary
  boundary and is approaching several Earth system thresholds. The
  weakening or reversal of terrestrial carbon sinks, for example through
  the on-going destruction of the world's rainforests, is a potential
  tipping point, where climate-carbon cycle feedbacks accelerate Earth's
  warming and intensify the climate impacts.
\item
  Novel entities: Emissions of toxic and long-lived substances such as
  synthetic organic pollutants, heavy metal compounds and radioactive
  materials represent some of the key human-driven changes to the
  planetary environment. Even when the uptake and bioaccumulation of
  chemical pollution is at sub-lethal levels for organisms, the effects
  of reduced fertility and the potential of permanent genetic damage can
  have severe effects on ecosystems far removed from the source of the
  pollution.
\end{itemize}

\begin{figure}

{\centering \includegraphics[width=0.6\linewidth]{figures/chap1/planetary_boundaries} 

}

\caption{The planetary boundaries (www.stockholmresilience.org)}\label{fig:f1}
\end{figure}

A more specific example of the plant-environment interactions is the
global carbon budget. We are currently facing this significant change in
biogeochemical cycling due to the rising fossil fuel emission over the
last 150 years. Figure \ref{fig:f2} represents the balance between
carbon sources (fossil carbon and land-use changes) and sinks (oceans,
land, and atmosphere). The more we emit, the more the Earth system is
capturing. Naturally, the emitted CO2 must go somewhere. Approximately
half of it is taken up by the ocean and the land (soil + vegetation)
sink. As stated earlier, roughly 25\% of the emitted CO2 is dissolved in
the ocean sink. The land sink is highly variable: land and soils are
very heterogeneous and difficult to model. Nevertheless, the land sink
has taken up more emissions in the past 60 years than it did before. The
atmosphere is responsible for most of the uptake. This takes us to a
question, frequently addressed by global vegetation models, about how
long these sinks will continue or not capture our increasing carbon
emissions. Note that there is an imbalance: there was more CO2 emitted
then absorbed. So, where did the surplus go? This is an active area of
research.

\begin{figure}

{\centering \includegraphics[width=0.8\linewidth]{figures/chap1/carbon_budget} 

}

\caption{The global carbon budget (www.globalcarbonproject.org)}\label{fig:f2}
\end{figure}

\textbf{Climate Models} predict how the long-term weather variation and
average weather will evolve. These models include an atmosphere, land
and ocean component (Figure \ref{fig:f3} top). The original climate
models focused on biophysics: energy and water balances, predicting
precipitation, radiation and fluxes between the three components. More
recently, climate models have evolved into \textbf{Earth System Models}
(ESM). ESM have a more complex concept because they represents more
processes (Figure \ref{fig:f3} bottom). Why? If you want to predict the
end of the century climate, we need to consider greenhouse gases and
thus the full carbon cycle. ESM are more complex but also more
realistic.

\begin{figure}

{\centering \includegraphics[width=0.7\linewidth]{figures/chap1/GCM_ESM} 

}

\caption{Scientific scope of (a) climate models and (b) earth system models. (Bonan 2019)}\label{fig:f3}
\end{figure}

Vegetation models are often the land component of an earth system model.
These `terrestrial biosphere models'(TBM) or `land surface models' (LSM)

\begin{itemize}
\tightlist
\item
  simulate \textbf{energy fluxes}: radiation, evapotranspiration and
  sensible heat fluxes between the land and the atmosphere. Depending on
  the vegetation type, the impacts are different.
\item
  simulate the \textbf{hydrology} and the \textbf{carbon cycle}.
\item
  simulate slower processes like \textbf{vegetation dynamics}: the
  succession of forest or \textbf{land use} and \textbf{urbanization}.
\end{itemize}

\begin{figure}

{\centering \includegraphics[width=0.8\linewidth]{figures/chap1/cycles_bonan} 

}

\caption{Scientific scope of terrestrial biosphere model. (Bonan 2019)}\label{fig:f4}
\end{figure}

The coupler is a system that links different models. For example the
terrestrial biosphere can be seen as the coupler between geochemistry
and hydrology. A coupler can be the link between more than two other
systems, so that a kind of satellite system is formed. The land model is
mostly seen as the coupler. This results in the fact that both fast and
slow processes depend on each other. The surface energy flux is an
example of a fast process since it varies over the course of one day
Vegetation dynamics on the other hand, is a slow process. Succession
does not happen overnight.

\section{Why do we need modelling?}\label{why-do-we-need-modelling}

Modelling has proven to be a essential tool:

\begin{itemize}
\tightlist
\item
  For \textbf{understanding}: we need good theoretical foundations
  (understand processes) to generalize knowledge and observations in
  space and time (upscaling). Studying the inaccuraies in models leads
  to the formulation of new hypotheses.
\item
  For \textbf{prediction}, how vegetation responds to expected changes
  (temperature or CO2) to develop management strategies and policies.
\item
  For \textbf{data integration}: a framework to bring together multiple
  data sources and to guide future data collection.
\end{itemize}

\section{Model types}\label{model-types}

How can we look at the different model types that exist (Table
\ref{table:example})? Models are to be placed in a continuum ranging
from empirical to process-based models. \textbf{Empirical models} are
based on data and correlations, not describing precisely the biophysical
processes --- \textbf{process-based models} describing the biophysical
processes and causal relations between the variables (Table
\ref{table:empiricial}). Most existing vegetation models are hybrid.

\begin{center}
\captionof{table}{Continuum of terrestrial biosphere/ecosystem models. (Bonan 2019)}
\label{table:example}

\begin{center}\includegraphics[width=0.8\linewidth]{figures/chap1/table_model_types} \end{center}
\end{center}

\begin{center}
\captionof{table}{Continuum of process-based versus empirical models. (Adams et al. 2013)}
\label{table:empiricial}

\begin{center}\includegraphics[width=0.8\linewidth]{figures/chap1/tables_PB_empirical} \end{center}
\end{center}

The model type depends on: - Purpose: will it be used for management
support, policy support, research. - Question: different people will be
interested in different questions (foresters, ecologists, policy
makers\ldots{}) - Scale: models that are to be applied for local use can
be much more detailed than worldwide models because data gathering is
much more straightforward on a small scale. Also the time-scale is of
importance: will the model be used for research about the past, the
present or the future?

\section{The history of vegetation
models}\label{the-history-of-vegetation-models}

The history of vegetation models is one that parallels that of the
computer. Computers made it possible to calculate much faster and much
more, which made them suitable for modelling. The first vegetation
models have emerged in the 1960s and 1970s.

One of the first models were the \textbf{box models} (1960); these
models describe the flow of mass and energy through boxes. These models
still exist in current biogeochemical models, where arrows represent the
fluxes between the pools. In parallel, \textbf{gap models} had emerged.
Gap models simulate the dynamics of the development of a gap in a forest
and the growth of plants in this gap. Gaps can be created by fallen
trees, by dead trees, \ldots{} . This kind of models are
individual-based and focused on population dynamics and the life cycle
of species: growth, regeneration and mortality while taking
environmental constraints in account. These models are the first models
that were ever used for upscaling: from tree level, to plot-level, to
landscape level. They were developed by forest scientists using forest
inventories to derive growth, regeneration, and mortality in response to
environmental variables. In 1973, the first model (MIAMI model by Lieth
(1973)) was developed to derive global net primary productivity (NPP),
relating NPP in an empirical way to climate variables (temperature) with
vegetation productivity. This was the first attempt to make a global
upscaling of a vegetation process. In the 1980s surged the first
\textbf{land surface models}. Land surface models are the models where
the other models start to integrate into, and evolved as such into --
\textbf{Terrestrial biosphere models} (see Figure \ref{fig:f7}), which
are now the state-of-the-art land components of ESMs. Vegetation
modelling therefore is a very interdisciplinary field because it
involves knowledge of different scientific fields, making it difficult
to find a common terminology. Global EMS are currently still not good to
simulate realistic vegetation dynamics.

\begin{figure}

{\centering \includegraphics[width=0.8\linewidth]{figures/chap1/timeline} 

}

\caption{Timeline showing the parallel development of model types and the integration of model types into land surface models towards terrestrial biosphere models. (Bonan 2019)}\label{fig:f7}
\end{figure}

\section{Components of a model}\label{components-of-a-model}

What is a vegetation model? Two attempts for a definition:

\begin{itemize}
\tightlist
\item
  \textbf{Dynamic global vegetation models} (DGVMs) are powerful tools
  to project past, current and future vegetation patterns and associated
  biogeochemical cycles (Scheiter et al., 2013).
\item
  A \textbf{Dynamic Global Vegetation Model} (DGVM) is a computer
  program that simulates shifts in potential vegetation and its
  associated biogeochemical and hydrological cycles as a response to
  shifts in climate. DGVMs use time series of climate data and, given
  constraints of latitude, topography, and soil characteristics,
  simulate monthly or daily dynamics of ecosystem processes. DGVMs are
  used most often to simulate the effects of future climate change on
  natural vegetation and its carbon and water cycles (Wikipedia 2021).
\end{itemize}

\begin{center}
\captionof{table}{Definition of key model components and examples for a typical TBM}
\label{table:components}

\begin{center}\includegraphics[width=0.9\linewidth]{figures/chap1/table_components} \end{center}
\end{center}

\subsection{Processes}\label{processes}

They are a key component because we are focusing on process-based models
in this course. There is a long list of processes (energy, water,
turbulent transport, canopy scaling, carbon, nitrogen, trace gasses,
demography,\ldots{}) that the models integrate, especially the more
complex ones. These processes will be discussed in detail in the
following theory chapters and we will mainly focus on how to translate
them into equations.

\subsection{Equations}\label{equations}

These are the mathematical representations of the processes. However,
there are important constraints to insert equations into a vegetation
model, such as the specific time scale at which a process operates. For
example, it makes little sense to resolve the equation for forest
composition (succession) on a daily calculation time step. This is a
prolonged process with an extremely low variance between consecutive
days. The solution for the equation for photosynthesis, on the other
hand, varies significantly throughout the day and between consecutive
days (cloudy day vs sunny day).

There are three types of equations within vegetation models: -
\textbf{prognostic equations}: time derivatives of differential
equations -- they calculate the state's change over time -
\textbf{conservation equations}: equations describing the conservation
of mass and energy - \textbf{diagnostic equations}: linking multiple
variables independent from the time.

Often there is no analytical solution of the equations describing
on-linear processes in biological systems; therefore, we must use
numerical methods to solve the equations.

\subsection{Parameters}\label{parameters}

These are the constants in the model. Some parameters are highly
uncertain because we cannot measure them very well at the relevant
scale. For example, we can make reliable measurements of the
photosynthetic capacity of a single leaf. However, upscaling this
parameter so that it is applicable for a forest or multiple PFTs (=
plant functional types) induces uncertainty. The more parameters a model
uses, the more uncertainties that are to be taken into account.

\subsection{Time Steps}\label{time-steps}

Vegetation models run at multiple timescales (combining processes that
are resolved at multiple timescales). Models present fast processes,
which are calculated every hour (e.g.~photosynthesis and energy
balance), intermediate processes calculated daily (e.g.~carbon
allocation and growth) and slow processes in order of years
(e.g.~mortality) (Fig.6).

\begin{figure}

{\centering \includegraphics[width=0.8\linewidth]{figures/chap1/time_steps} 

}

\caption{Structure of a vegatation model indicating the different time steps at which each process is simulated (Williams et al. 2009)}\label{fig:f9}
\end{figure}

\subsection{Spatial structure}\label{spatial-structure}

The division of space in voxels, layers or grid cells and its resolution
determines how many times we repeat our calculations in space. Global
vegetation models have a typical spatial grid of 100km or even more and
divide the landscape into patches. In each patch, they simulate the
vegetation (forest, savannas, grassland\ldots{}). Models also have a
horizontal grid or horizontal layering: some models consider multiple
soil layers. The same is true for above ground layers, where some models
divide the canopy into multiple layers (Figure \ref{fig:f10}). For
example, the Ecosystem Demography Model (ED2.2) divides the forest into
multiple grid cells where the same meteorological conditions apply
within each grid cell. Then within each cell, this model has different
sites with different soils. Each site is divided into multiple patches
(forests with a similar disturbance history). For each patch, the model
simulates multiple cohorts of trees where size and plant functional
types play a role (Figure \ref{fig:f11}).

\begin{figure}

{\centering \includegraphics[width=0.5\linewidth]{figures/chap1/grid_vert_hor} 

}

\caption{Three dimensional grid of a TBM structured in terms of longitude x latitude x level. The number of soil and canopy layers and the geographical resolution is model dependent, (Bonan 2019)}\label{fig:f10}
\end{figure}

\begin{figure}

{\centering \includegraphics[width=0.7\linewidth]{figures/chap1/grid_ED2} 

}

\caption{Example: the spatial multi-level grid structure of of the ED2 vegetation model (Longo et al. 2019)}\label{fig:f11}
\end{figure}

\subsection{Model code, complexity and
uncertainty}\label{model-code-complexity-and-uncertainty}

There is a gap between equations and how the are implemented in the
actual model code. Also, a specific process can be implemented into an
equation in various ways. Usually, large models also contain a
``technical debt'', which means over the years, multiple modelers have
continued working on models and added code lines, but at some point, the
code is so large that none of the developers still knows the entire
code, resulting in persistent bugs or overlooked assumptions.

Models are always a simplification of the real world, but they tend to
become overly complex.

More complex models (adding more processes) become more realistic, but
we also add more sources of uncertainty. Therefore, we should choose our
model carefully based on the research question we want to adress.

\subsection{Data}\label{data}

It is not possible to develop models without data. In general, the more
data (multiple data sources), the better.

\section{Modelling workflow and structure of the
course}\label{modelling-workflow-and-structure-of-the-course}

Vegetation modelling is a multidisciplinary field. This course will
mainly focus on the mathematical formulation of processes and
translating these equations into a working model.

\begin{figure}

{\centering \includegraphics[width=0.9\linewidth]{figures/chap1/course_overview} 

}

\caption{Progression through spatial and temporal scales throughout this course}\label{fig:f12}
\end{figure}

The construction of a model is a continuous process -- a model is never
finished. As Figure 10 shows us, we start by describing our system in
the form of equations, then running the computer program to characterize
the model, perform parameter estimation and interpretation, and then
apply it to other locations and validate against independent data.

\begin{figure}

{\centering \includegraphics[width=0.8\linewidth]{figures/chap1/williams_fusion} 

}

\caption{Model data fusion in every step of the model development cycle (Williams et al. 2009)}\label{fig:f13}
\end{figure}

\begin{figure}

{\centering \includegraphics[width=0.8\linewidth]{figures/chap1/dietze_workflow} 

}

\caption{Methodological workflow of model data fusion (Dietze: Ecological Forecasting)}\label{fig:f14}
\end{figure}

\part{Biophysical and physiological
models}\label{part-biophysical-and-physiological-models}

\chapter{Modelling plant basic
processes}\label{modelling-plant-basic-processes}

\chaptermark{photsynthesis}

\section{Photosynthesis models}\label{photosynthesis-models}

Photosynthesis is a process that takes place in all plants on earth.
There is a huge variability in photosynthesis rates in space and time
because of environmental variations (weather/climate) and properties of
the vegetation (species composition, structure).

\subsection{Refreshing the basic
knowledge}\label{refreshing-the-basic-knowledge}

Leaf processes that are discussed in this lecture occur at the small
spatio-temporal timescale (Figure \ref{fig:f12}), but have an important
impact on ecosystem functioning on the long term, for example:
photosynthesis.

\begin{itemize}
\tightlist
\item
  Carbon assimilation (Photosynthesis) and water loss (Transpiration)
  are regulated by stomatal closure (Figure \ref{fig:f21}).
\item
  Leaf temperature is the result of the energy balance at the leaf level
  which is tightly regulated by transpiration, and thus will impact
  photosynthesis.
\end{itemize}

\begin{figure}

{\centering \includegraphics[width=0.8\linewidth]{figures/chap2/leaf_level_processes} 

}

\caption{Leaf level processes transpiration and photosysnthesis are strongy interlinked and both regulated by stomatal conductance}\label{fig:f21}
\end{figure}

Carbon assimilation follows two main reactions:

\begin{itemize}
\tightlist
\item
  Light harvesting (chlorophyll): the use of light to convert molecules
  into higher energy molecules through the photosystems
\item
  Carbon fixation: converting \(CO_2\) into sugars
\end{itemize}

The RuBisCO (ribulose-1,5-diphosphate carboxylase/oxygenase) enzyme
plays an essential role for the carboxylation and oxidation reactions,
while light harvesting is essential to regenerate the RuBP (ribulose
1,5-bisphosphate) --\textgreater{} see the Calvin cycle

Three different photosynthetic mechanisms:

\begin{itemize}
\tightlist
\item
  C3: PGA (phosphoglycerate) is the first product of \(CO_2\)
  assimilation in the Calvin cycle, the entire process takes place in
  the mesophyll cell
\item
  C4: PEP (phosphoenolpyruvate): make organic acids with a 4C skeleton,
  only 2\% of all plants follow a C4 pathway. It represents only 5 \% of
  the plant biomass on earth, but despite its scarcity, C4 plants
  account for around 23\% of the terrestrial carbon fixation; they have
  the Kranz anatomy, with bundle sheath cells, process is separated
  between mesophyll and bundle sheet cells; The main advantage is that
  C4 plants have a higher \(CO_2\) concentration in their bundle sheath
  cells, in which they also recycle \(CO_2\).
\item
  CAM: will not be discussed in the models, here the \(CO_2\)
  assimilation and \(CO_2\) uptake are separated in time, plants adapted
  to very dry conditions. How to observe photosynthesis? Mainly done by
  leaf gas exchange measurements. This methodology measures the net
  \(CO_2\) exchange of the leaf, which is the sum of the gross
  photosynthesis minus the respiration. The difference between these two
  fluxes is the net photosynthesis.
\end{itemize}

Photosynthesis reacts to different environmental factors (Figure
\ref{fig:f22}).

\begin{enumerate}
\def\labelenumi{\arabic{enumi}.}
\tightlist
\item
  \textbf{Light}: Assimilation increases with increasing light
  availability, but reaches saturation at higher light levels.
\item
  \textbf{Temperature}: Assimilation increases up to an optimum
  temperature. Beyond the optimal temperature there is a limitation of
  photosynthesis.
\item
  \textbf{VPD}: Assimilation is reduced with increasing VPD.
\item
  \textbf{Moisture}: Assimilation decreases with decreasing leaf water
  potential (drought stress), less turgor, stomates close.
\item
  \textbf{\(CO_2\)}: Assimilation increases at elevated atmospheric
  \(CO_2\) concentration, with saturation.
\item
  \textbf{Nutrients}: The more nitrogen a leaf contains (for example at
  the top of the canopy), the more \(CO_2\) can be assimilated. Leaf N
  is a proxy of the amount of rubisco.
\end{enumerate}

\begin{figure}

{\centering \includegraphics[width=0.8\linewidth]{figures/chap2/photosynthesis_obs} 

}

\caption{Net C assimilation in relation to (a) photosynthetically active radiation,(b) temperature, (c) vapor pressure deficit at 25°C and 35°C,(d) foliage water potential, (e) ambient $CO_2$ concentration, and (f) foliage water potential for jack pine trees (Pinus banksiana). Bonan (2019)}\label{fig:f22}
\end{figure}

Different environmental parameters interact with each other (Figure
\ref{fig:f23}).

Left: leaf respiration response is different than the photosynthesis
response. Respiration keeps increasing with T, followed by a collapse
(degrading enzymes), the respiration peak occurs at a higher temperature
than the photosynthesis peak. The temperature response is also different
for different \(CO_2\) levels in the atmosphere.

Right: at lower light levels, the optimal temperature also drops. These
interacting responses, need to be implemented in models!

\begin{figure}

{\centering \includegraphics[width=0.8\linewidth]{figures/chap2/Trespons_interactions} 

}

\caption{Temperature responses of photosynthesis, respiration and net $CO_2$ exchange, interaction with $CO_2$ concentration (A) and light  (B)  Schulze ()}\label{fig:f23}
\end{figure}

\subsection{C3 photosynthesis}\label{c3-photosynthesis}

\subsubsection{Light response curve
models}\label{light-response-curve-models}

This model is based on light only. The slope of the curve (Figure
\ref{fig:f24}) is the quantum yield, and describes the linear part of
the curve.

Pmax: max photosynthesis rate, describes the level of the saturation
plateau. Light compensation point, is that light level at which
respiration equals the gross carbon uptake. Dark respiration: \(CO_2\)
released in the dark by respiration. This curve is described based on
measurements on plants, and formulated as a mathematical equation.

\begin{figure}

{\centering \includegraphics[width=0.8\linewidth]{figures/chap2/LRC} 

}

\caption{Conceptual figure of a leaf-level light reponse curve}\label{fig:f24}
\end{figure}

In a second approach the light response is described as a co-limitation,
a response with two phases: (1) light limitation and (2) light
saturation (linear increase and plateau). Such a co-limitation is
described by the non-rectangulare hyperbola formula below (Figure
\ref{fig:f25}). The formula has two parameters: Amax (max photosynthesis
rate) and E (quantum yield). The theta factor adapts the curve shape.

Disadvantages of these light response curve models: they are measured on
a specific plant in specific conditions, and only applicable under the
same conditions for the same plant, on the same location. They are only
used in empirical studies, not really applicable for prognostic models.

\[
\theta.A^2-(E.I+A_{max})A+E.I-A_{max}=0
\]

with \(\theta\) the curvature of the light response curve, \(A\) the
assimilation rate in \(micromol m^{-2} s^{-1}\), and \(I\) the incident
radiation in \(micromol m^{-2} s^{-1}\).

\begin{figure}

{\centering \includegraphics[width=0.8\linewidth]{figures/chap2/hyperbola} 

}

\caption{Co-limitation illustrated for photosynthetic response to light. The two dashed lines show the rates Amax adn EI The solid lines show the co-limited rate. (Bonan 2019)}\label{fig:f25}
\end{figure}

\subsubsection{Light use efficiency
models}\label{light-use-efficiency-models}

Light use efficiency (LUE) is the slope of the linear part of the light
response curve. These LUE models calculate photosynthesis at a larger
scale (GPP) by multiplying the available amount of light with a light
use efficiency factor(Figure \ref{fig:f26}). Other factors are
introduced to mimic saturation at stressful environmental conditions
(e.g.~temperature, soil water availability, atmospheric drought). These
are empirical models, with its advantages and disadvantages (see above).
Often used for large scale, remote sensing-based simulations of
photosynthesis, as I can be measured by satellites.

\[
GPP=LUE.I.f_1(T).f_2(\theta).f_3(D)
\]

\begin{figure}

{\centering \includegraphics[width=0.8\linewidth]{figures/chap2/MODIS_GPP} 

}

\caption{MODIS based GPP map of the US, based on a LUE model.}\label{fig:f26}
\end{figure}

\subsubsection{The Farquhar, Von Caemmerer and Berry model
(FvCB)}\label{the-farquhar-von-caemmerer-and-berry-model-fvcb}

Photosynthesis is not only limited by light response. Diffusion of
\(CO_2\) from atmosphere into the leaf plays an important role. This is
driven by the \(CO_2\) concentration gradient between the atmosphere and
the substomatal space. Diffusion depends on the boundary layer of the
leaf, the area of the stomatal cavity as well as the resistivity of the
chloroplast membrane (Figure \ref{fig:f27}).

The FvCB model was developed in 1980, and is currenty included in the
large majority of (global) vegetation models. The FvCB model describes
the rubisco kinetics, rubisco regeneration, the carboxylation rate when
rubisco is saturated and the carboxylation rate when rubisco is limited.

\textbf{See Bonan's book, Chapter 11.1}

\begin{figure}

{\centering \includegraphics[width=0.8\linewidth]{figures/chap2/conductance} 

}

\caption{Diffusion of $CO_2$ from free air across the leaf boundary layer and through stomata to the intercellular space. Diffusion to the chloroplast is additionally regulated by mesophyl conductance. (Bonan 2019)}\label{fig:f27}
\end{figure}

The net photosynthesis is described as the carboxylation rate \(V_c\)
minus the photorespiration \(V_o\) and dark leaf respiration \(R_d\). \[
A_n=V_c-0.5V_o-R_d
\] with \({A_n}\), \({V_c}\), \({Vo}\) and \({R_d}\) in
\(micromol m^{-2} s^{-1}\).

\({V_c}\) corresponds to the rate of carboxylation by RUBISCO, which
follows a Michalelis-Menten response function: \[
V_c=\frac{V_{cmax}C_i}{C_i+K_c\left( 1+\frac{O_i}{K_o}\right)}
\] with \(V_{cmax}\) the maximum rate of carboxylation in
\(micromol CO_{2} m^{-2} s^{-1}\), \(C_i\) and \(O_i\) the intercellular
CO\_\{2\} concentration in \(micromol mol^{-1}\), and \(K_c\) and
\(K_o\) are the Michaelis-Menten constants for CO2 and O2 in
\(micromol mol^{-1}\), respectively.

Photorespiration releases 0.5 mole of \(CO_2\) \[
V_o=\frac{V_{omax}O_i}{O_i+K_o\left( 1+\frac{C_i}{K_c}\right)}
\]

It exists a specific \(C_i\) concentration value at which oxygenation
compensate carboxylation and no net CO2 uptake occurs in the absence of
mitrochondrial respiration. We call this value the CO2 compensation
point \(\Gamma^*\). Net assimilation can be reformulated as: \[
A_n=(1-\frac{\Gamma^*}{C_i})V_c-R_d
\] The term \(1-\Gamma^*/C_i\) accounts for CO2 release during
photorespiration.

In the simplest form of the model, carboxylation \(V_c\) is limited by
either the activity of RUBISCO (CO2 limitation), denoted by the rate
\(W_c\), or by the regeneration of RuBP (light limitation), denoted by
the rate \(W_j\), so that: \[
A_n=(1-\frac{\Gamma^*}{C_i})min(W_c,W_j)-R_d
\]

with Wc describing the Michealis-Menten kinetics of carboxylation. \[
W_c=V_{cmax}\frac{C_i}{C_i+K_c(1+O_i/K_o)}
\]

and Wj represents the rate of regeneration of the RuBP by light. It
depends on electron transport rate, as such the photosynthesis light
response is integrated in the Farquhar model.\\
\[
W_j=J\frac{C_i}{4C_i+8\Gamma^*}
\] \(J\) is the electron transport rate (in \(micromol m^{-2} s^{-1}\))
for a given irradiance.

The rate of electron transport is related to absorbed photosynthetically
active radiation (PAR), the maximum electron transport rate and the
amount of light utilized by photosystems. Different expressions can be
found for J. We will use the most common form:

\[
J=\frac{I_{abs}+J_{max}-\sqrt{(I_{abs}+J_{max})^2-4\theta_jI_{abs}J_{max}}}{2\theta_j}
\]

with \(I_{abs}\) the absorbed PAR in \(micromol photon m^{-2} s^{-1}\),
\(\theta_j\) is the curvature of the light response curve and
\(J_{max}\) the maximal electron transport rate in
\(micromol m^{-2} s^{-1}\).

The absorbed incident radiation by the photosystem II (\(I_{abs}\))
depends on leaf absorptance \(\alpha_l\) and the quantum yield of
photosystem II \(\phi_{PSII}\): \[
I_{abs}=\frac{\phi_{PSII}}{2}\alpha_lI_r
\] where \(I_r\) is PAR in \(micromol m^{-2} s^{-1}\), and only one half
of absorbed light reaches the PSII.

In summary, a common form of the FvCB model represents photosynthetic
CO2 assimilation for plants utilizing the C3 photosynthetic pathway as
limited by (i) Ac -- the rate of Rubisco catalyzed carboxylation when
RuBP is saturated (called Rubisco-limited photosynthesis because of its
dependence on maximum Rubisco activity as set by Vcmax); and (ii) Aj --
the rate of RuBP regeneration by light absorption and electron transport
as determined by Jmax (RuBP regeneration-limited, or light-limited,
photosynthesis).

When accounting for the CO2 compensation point, the general Farquhar
model is thus written as follows:

\[
A_n=min(A_c,A_j)-R_d
\] with \[
A_c = V_{cmax}\frac{C_i-\Gamma^*}{C_i+K_c\left(1+\frac{O_i}{K_c}\right)}
\]

\[
A_j = \frac{J}{4}\left(\frac{C_i-\Gamma^*}{C_i+2\Gamma^*}\right)
\]

Figure \ref{fig:f28} illustrates the response of assimilation to
intercellular \(CO_2\) and light. We neglect TPU limitation (\(A_p\)) in
the models we consider here. TPU limitation only occurs at elevated
intercellular \(CO_2\).

\begin{figure}

{\centering \includegraphics[width=0.8\linewidth]{figures/chap2/simulated_responses} 

}

\caption{Simulated responses of C3 photosynthesis in relation to (a) intercellular $CO_2$ (at I↓ = 2000 micromol m–2 s–1) and (b) photosynthetically active radiation (at ci = 266 micromol mol–1). (Bonan 2019)}\label{fig:f28}
\end{figure}

As an alternative formulation a co-limitation analytical model can be
used. In this case we don't use the minimum equation of the original
Farquhar model, but take both limitations into account simultaneously in
the following equation.

\[
\Theta_{A}A^2-(A_c+A_j)A+A_cA_j
\]

\subsection{Parameter and temperature
dependencies}\label{parameter-and-temperature-dependencies}

\textbf{See Bonan's book, Chapter 11.2}

The FvCB model requires 6 physiological parameters: \(K_c\), \(K_o\),
\(\Gamma^*\), \(V_{cmax}\), \(J_{max}\) and \(R_d\). Additionally, the
specification of electron transport requires \(\theta_j\) and
\(\phi_j\). Values for\(K_c\), \(K_o\) and \(\Gamma^*\) are defined by
the biochemistry of Rubisco and are similar among species. On the
opposite, \(V_{cmax}\) is species-dependent.

\(V_{cmax}\) is a key parameter in the FvCB model. It directly
determines the Rubisco-limited rate \(A_c\) and, for C3 plants, also
influences the RuBP regeneration-limited rate \(A_j\) though its
covariation with \(J_max\). The maximum rate of carboxylation has a wide
range among plant species and environments and is tightly linked to leaf
nitrogen content; reported values for 109 species of C3 plants vary from
less than 10 to more than 150 \(micromol m^{–2} s^{–1}\).

Because \(J_{max}\) is well correlated to \(V_cmax\), it can be
approximated to: \[
J_{max} = 1.67V_{cmax}
\]

\(R_d\) is also well correlated to \(V_cmax\) and can be approximated
as: \[
R_d=0.015V_{cmax}
\]

The parameters \(K_c\), \(K_o\), \(\Gamma^*\), \(V_{cmax}\), \(J_{max}\)
and \(R_d\) vary with temperature (Figure \ref{fig:f28bis}). The
instantaneous responses of photosynthesis and respiration to temperature
are driven by their underlying enzymatic responses. When warmed from low
temperature, the enzymes involved in photosynthesis and respiration
increase their activity as described by the Arrhenius function: \[
f(T_l)=exp\left[\frac{\Delta H_a}{298.15R}(1-\frac{298.15}{T_l})\right]
\] where \(T_l\) is leaf temperature (in K), R is the universal gas
constant (8.314 J K\textsuperscript{--1} mol\textsuperscript{--1}), and
\(\Delta H_a\) is the activation energy (J mol\textsuperscript{--1}).
This function is normalized to 25°C (298.15 K). Parameter values
measured at 25°C are multiplied by \(f(T_l)\) to adjust for leaf
temperature.

\begin{figure}

{\centering \includegraphics[width=0.8\linewidth]{figures/chap2/temp_responses} 

}

\caption{Relative temperature responses of the parameters of the Farquhar model (Bonan 2019)}\label{fig:f28bis}
\end{figure}

The parameters \(V_{cmax}\), \(J_{max}\) and \(R_d\) vary with
temperature following the Arrhenius function but have a peaked response
in which enzymatic activity increases up to a temperature optimum beyond
which the reaction rate decreases when temperature is too high because
of enzyme degradation. In this case, parameters such as \(V_{cmax}\) are
writen by: \[
V_{cmax}=V_{cmax25}f(T_l)f_H(T_l)
\] with \(V_{cmax25}\) the value of \(V_{cmax}\) at 25°C. The
\(f_H(T_l)\) function represents the thermal breakdown of biochemical
processes: \[
f_H(T_l)=\frac{1+exp\left(\frac{298.15\Delta S-\Delta H_d}{298.15R}\right)}{1+exp\left(\frac{\Delta ST_l-\Delta H_d}{RT_l}\right)}
\]

To correctly model photosynthesis over time, you should also take into
account climate adaptation (acclimation): the temperature adaptation
curve is different for the same species if they are grown under
different climatic conditions.

Challenges with the Farquhar model: its parameter values are not well
defined, or the values of the parameters are highly variable, depending
on the type of plant or environmental conditions. Vcmax is highly
variable over different leaves in the same tree, and on global scales.
Which value should we use? -\textgreater{} Vcmax is scaling very well to
photosynthesis, so Vcmax is often optimized with data available.
However, by doing so you turn the model into an empirical model, so this
should be avoided. The advantage we have, is that there are covariations
between parameters (Figure \ref{fig:f29}) for example the strong
correlation between Vcmax and Jmax -\textgreater{} you can describe Jmax
as a function of Vcmax, so you can remove one parameter from the model.

\begin{figure}

{\centering \includegraphics[width=0.8\linewidth]{figures/chap2/vcmax_jmax} 

}

\caption{Linear relation between observed Vcmax and Jmax values for Beech (Verbeeck et al. 2008)}\label{fig:f29}
\end{figure}

Vcmax can also be calculated based on leaf nitrogen content. Some
vegetation models model the N content of leaves, and simulated based on
the Vcmax of the leaves. It's a linear relationship with two parameters.
Some models also use more complex relations.

There are also attempts to propose average values per plant functional
type.

\[
V_{cmax25} = i_v + s_v.N_a
\] The temperature functions also have dependence factors.

The Farquhar model has 4 input variables: \(O_i\), \(C_i\), \(PAR\) and
\(T_{leaf}\). Light depends on how the radiation is penetrating the
canopy \textbf{-\textgreater{} see lecture 3}. \(O_i\) is assumed to be
equal to the oxygen concentration in the atmosphere. Leaf temperature
\(T_{leaf}\) is in most models assumed to be equal to air temperature or
can be calculated based on an energy balance. \(C_i\) is calculated
based on gas diffusion through the stomates -\textgreater{} you need an
extra model to fully simulate the behavior of the stomates.

\begin{figure}

{\centering \includegraphics[width=0.8\linewidth]{figures/chap2/full_farquhar} 

}

\caption{Equations of the full Farquhar model}\label{fig:f210}
\end{figure}

\subsection{C4 photosynthesis}\label{c4-photosynthesis}

\textbf{See Bonan's book, Chapter 11.7}

The behaviour of the C4 and C3 models is really different

\begin{figure}

{\centering \includegraphics[width=0.8\linewidth]{figures/chap2/c3_c4} 

}

\caption{Comparison of C3 and C4 photosynthesis in response to (a) photosynthetically active radiation, (b) ambient $CO_2$ concentration, (c) leaf temperature, and (d) vapor pressure deficit. In this figure, stomatal conductance is calculated using the Ball–Berry model and ci is obtained from the diffusion equation}\label{fig:f210b}
\end{figure}

\section{Stomatal models}\label{stomatal-models}

\subsection{Refreshing the basic
knowledge}\label{refreshing-the-basic-knowledge-1}

Guard cells are transporting ions actively and passively into/out of the
cells in order to open or close the stomates by adapting the turgor in
the guard cells. The stomatal conductance represents how easy it is for
a \(CO_2\) molecule to diffuse in or out of the stomates. The higher the
conductance, the easier the transfer. Stomatal models are needed for
transpiration, leaf temperature and \(CO_2\) -\textgreater{} key process
in the model. Stomatal conductance for \(CO_2\) is lower than that of
water as the molecule is a bit bigger. Sometimes expressed as a
resistance in models that use the electric analog concept (Figure
\ref{fig:f211}).

\begin{figure}

{\centering \includegraphics[width=0.8\linewidth]{figures/chap2/conductance} 

}

\caption{Diffusion of $CO_2$ from free air across the leaf boundary layer and through stomata to the intercellular space. Diffusion to the chloroplast is additionally regulated by mesophyl conductance. (Bonan 2019)}\label{fig:f211}
\end{figure}

Plants constantly balance their physical and biochemical limitation, by
adjusting their stomatal conductance.

Photosynthesis is expressed as a diffusion process. Photosynthesis flux
is a range through the circuit of figure \ref{fig:f211}. \(A_n\) can be
described as the flux going from the atmosphere to the internal part of
the leaf, or as the flux going from the internal part of the leaf into
the cell, or a combinations of the two methods.

\[
A_n=\frac{g_{bw}}{1.4}(C_a-C_s) = \frac{g_{sw}}{1.6}(C_s-C_i)=\frac{1}{1.4g_{bw}^{-1}+1.6g_{sw}^{-1}}(C_a-C_i)
\] with \(g_{sw}\) and \(g_{bw}\) the stomatal and boundary layer
conductance for water, respectively, in mol m\textsuperscript{-2}
s\textsuperscript{-1}. the ratio 1.6 correspond to the molecular mass
ration between \(H_2O\) and \(CO_2\) to convert conductance to water to
\(CO_2\). \(C_a\), \(C_s\) and \(C_i\) are the \(CO_2\) concentration
from the atmosphere, at leaf surface and in the stomatal cavity,
respectively.

Stomatal conductance is also dependent on environmental conditions
(Figure \ref{fig:f212}). There is a saturated response to light, an
optimum for temperature, strong response to VPD, stomata close when VPD
is too high, and there is a response to the leaf water potential. Again,
there is interaction between the different environmental conditions.

\begin{figure}

{\centering \includegraphics[width=0.8\linewidth]{figures/chap2/gs_obs} 

}

\caption{Observed responses of stomatal conductance for Pinus banksiana. (Bonan 2019)}\label{fig:f212}
\end{figure}

If the air is dry stomates close to avoid losing too much water

\emph{accounting for all these effects makes it complex }stomatal
conductance is quite linear with photosynthesis

We have three unknowns in the combined photosynthesis-diffusion
equations but only two equations: we need an extra equation to describe
stomatal conductance to be able to solve for the unknowns. 4
Possibilities discussed: see below.

\subsection{Empirical multiplicative
models}\label{empirical-multiplicative-models}

This is an empirical approach. The actual stomatal conductance is
calculated based on the max stomatal conductance (\(G_{smax}\)
multiplied by correction factors for light, temperature, etc. between 0
and 1. It is based on observations. Gsmax is species dependent. There is
no link with the carbon cycle/photosynthesis, this is why it is not used
anymore in the global vegetation models as part of the earth system
models of the IPCC

\[
g_{sw}=g_{smax}f(I^{\downarrow})f(T_l)f(D_l)f(\Phi_l)f(C_a)
\]

\subsection{Semi-empirical photosynthesis-based
models}\label{semi-empirical-photosynthesis-based-models}

Semi-empirical models are still commonly used in vegetation models and
are based on the linear relation between stomatal conductance and
photosynthesis (Figure \ref{fig:f213}).

\begin{figure}

{\centering \includegraphics[width=0.8\linewidth]{figures/chap2/ball_berry} 

}

\caption{Relationship between stomatal conductance and Anhs/cs for soybean.(Bonan 2019). On the x axis, the photosynthesis is depicted, but multiplied by relation humidity and $CO_2$ concentration at the leaf surface.}\label{fig:f213}
\end{figure}

\textbf{Ball-Berry model}: Stomatal conductance is calculated based on
photosynthesis, relative humidity and \(CO_2\) concentration at the leaf
level. This is the third equation needed to solve the problem. The model
needs observations to determine the parameters (i.e.~semi-empirical).
Plants adapt their stomatal opening to maintain a constant Ci/Ca ratio,
or the \(CO_2\) concentration in the leaf and the concentration outside
the leaf (atmosphere). As the model uses the \(CO_2\) concentration at
the surface of the leaf, we also need the boundary layer conductance.

\[
g_{sw}=g_0 + g_1\frac{A_n}{C_s}h_s
\] with \(g_0\) and \(g_1\) empirical model parameters, \(A_n\) the net
assimilation, \(C_s\) the \(CO_2\) concentration at the leaf surface and
\(h_s\) the relative air humidity.

The model becomes a combination of multiple non-linear equations, which
cannot be solved analytically. Some model try to find an analytical
solution by simplifying the equations, this gains calculation speed.
Other models use numerical techniques based on the scheme below (Figure
\ref{fig:f214}): we start with an initial guess of \(C_i\) and do enough
iterations until the guess convergences with the outcome value of the
model. This approach is slower as multiple iterations have to be
calculated, which is time-consuming. This model still needs empirical
input to determine the Ball-Berry parameters. SO it is a theoretical
model that needs empirical constraints.

\begin{figure}

{\centering \includegraphics[width=0.8\linewidth]{figures/chap2/numerical_solution} 

}

\caption{Flow diagram of the iterative procedure to numerically calculate ci.(Bonan 2019)}\label{fig:f214}
\end{figure}

\subsection{WUE models and optimality
theory}\label{wue-models-and-optimality-theory}

Not discussed in detail, but this approach is increasingly implemented
in models as it overcomes the use of empirical parameters. They are
based on ecological theories. Here, the stomatal conductance is
optimized by maximizing the water use efficiency. We assume that plants
try to optimize the benefit of photosynthesis with the cost of water
loss. Stomatal conductance is not described explicitly by an equation,
but it emerges from the model. The concept of calculations is shown in
figure \ref{fig:f215}: we start with an assumed stomatal conductance
value, we go through all calculations and in the end calculate WUE and
do the same for an increased/decreased stomatal conductance, and check
what the gain of WUE is.

\begin{figure}

{\centering \includegraphics[width=0.8\linewidth]{figures/chap2/optimality} 

}

\caption{Flow diagram of leaf flux calculations to numerically solve for stomatal conductance that optimizes water-use efficiency.(Bonan 2019)}\label{fig:f215}
\end{figure}

\subsection{Soil drought stress}\label{soil-drought-stress}

Basic models do not account for soil drying. Many models add an
additional factor that describes the soil wetness. The beta factor
varies between 0 (wilting point) and 1 (field capacity), with different
courses between the two extremes (see figure \ref{fig:f216}). Various
vegetation models differ in the way they account for this factor. Some
models apply this factor on Vcmax other model integrated the factor in
the stomatal equation. Some models simulate leaf figure (Figure
\ref{fig:f216}) or soil (Figure \ref{fig:f217}) water potentials.
Species have a different response to the leaf water potential. First
there is no impact, species have a tolerance, but after threshold, a
decrease in photosynthesis occurs.

\begin{figure}

{\centering \includegraphics[width=0.8\linewidth]{figures/chap2/leafWP} 

}

\caption{Leaf carbon uptake in response to leaf water potential for multiple tree species.}\label{fig:f216}
\end{figure}

\begin{figure}

{\centering \includegraphics[width=0.8\linewidth]{figures/chap2/SWfactor} 

}

\caption{Soil moisture wetness factor in relation to volumetric water content. (Bonan 2019)}\label{fig:f217}
\end{figure}

\subsection{Hydraulic models}\label{hydraulic-models}

These models are originally developed to simulated plant water relations
of individual plants/trees (Figure \ref{fig:f218}). The model simulates
the water transport through the trees. Challenging to apply at the large
scale. Isohydric species: maintain their leaf water potential,
independently from the soil water potentials.

\begin{figure}

{\centering \includegraphics[width=0.8\linewidth]{figures/chap2/hydraulics} 

}

\caption{Flow of water and representative water potentials along the soil–plant–atmosphere continuum. Also shown are conductances along the hydraulic pathway.(Bonan 2019)}\label{fig:f218}
\end{figure}

\textbf{Soil-Plant-Atmosphere (SPA) model}: combines a plant hydraulic
model with photosynthesis model, and they interact with each other. Such
models have a high level of complexity (Figure \ref{fig:f219}).

\begin{figure}

{\centering \includegraphics[width=0.8\linewidth]{figures/chap2/SPA} 

}

\caption{Depiction of (a) plant hydraulics and (b) leaf gas exchange in the Soil–Plant–Atmosphere (SPA) model. SPA is a multilayer canopy model.(Bonan 2019)}\label{fig:f219}
\end{figure}

Different models give different results (Figure \ref{fig:f220}): full
line: Ball-Berry, dashed line: other models. The shapes of the curves
are similar, but there is a difference between them.

\begin{figure}

{\centering \includegraphics[width=0.8\linewidth]{figures/chap2/modelling_approaches} 

}

\caption{Simulated stomatal responses for various modelling approaches. (Bonan 2019)}\label{fig:f220}
\end{figure}

\section{Upscaling from leaf to
canopy}\label{upscaling-from-leaf-to-canopy}

In this chapter we discussed all processes at the leaf level, but in a
forest or other vegetation type, every leaf is different. E.g. the
difference between sun and shade leaves.

Figure \ref{fig:f221}: different responses for shade and sun leaves: big
differences! on the long term there is also genetic adaptation to the
environmental conditions they are living in.

\begin{figure}

{\centering \includegraphics[width=0.8\linewidth]{figures/chap2/sun_shade} 

}

\caption{Leaf microclimate and boundary layer processes in relation to leaf dimension for sun and shade conditions.(Bonan 2019)}\label{fig:f221}
\end{figure}

If you calculated GPP (Figure \ref{fig:f222}), you need to take into
account that leaves are not the same all over the world -\textgreater{}
see coming lectues.

\begin{figure}

{\centering \includegraphics[width=0.8\linewidth]{figures/chap2/GPPcontrols} 

}

\caption{Controlling facors on ecosystem GPP. (Chapin)}\label{fig:f222}
\end{figure}

Thicker arrow means a stronger influence on ecosystem long term GPP.

\section{Case studies}\label{case-studies}

\subsection{Case study 2.1 Ozone impact on global
GPP}\label{case-study-2.1-ozone-impact-on-global-gpp}

\textbf{Sitch et al. 2003: Indirect radiative forcing of climate change
through ozone effects on the land-carbon sink}

The evolution of the Earth's climate during the 21th century depends on
the rate at which \(CO_2\) emissions are removed by the ocean and land
carbon cycles . To asses/simulate this removal, climate-carbon cycle
models are used, but these typically neglect the impacts of changing
atmospheric chemistry (e.g. \(CO_2\) and ozone concentration). Ozone
causes cellular damage inside leaves that adversely affects plant
production and reduces photosynthesis. In this case study, a global land
carbon cycle model - modified to include the effect of ozone and to
account for interactions between ozone and \(CO_2\) - is used to
estimate the impact of projected changes in ozone levels on the
land-carbon sink.

Large scale model. It used a similar leaf gas exchange model as
discussed above, but they added an extra equation that reduces stomatal
conductance depending on the ozone concentrations (higher ozone levels
reduce photosynthesis). The impact of the ozone is highest in tropical
zones, with reductions up to 30\%, which will impact the \(CO_2\)
storage of tropical forests.

\begin{figure}

{\centering \includegraphics[width=0.8\linewidth]{figures/chap2/ozone} 

}

\caption{Simulated global GPP reduction in response to current and future atmospheric ozone concentrations}\label{fig:f223}
\end{figure}

\subsection{Case study 2.2 Drought impact on rainforest
GPP}\label{case-study-2.2-drought-impact-on-rainforest-gpp}

\textbf{Fisher et al. 2007: The response of an Eastern Amazonian rain
forest to drought stress: results and modelling analyses from a
througfall exclusion experiment }

Climate change projections predict harsher droughts in the Amazon rain
forest which could change the amount of \(CO_2\) the forest can absorb,
creating a positive feedback system. A lack in appropriate data and
process-level understanding creates uncertainty in predicting the forest
gas exchange as a response too drought. In this paper these two problems
are addressed: new and better data is collected with a TFE experiment
and a model (SPA) that predicts sap-flow with soil moisture data is
created and fitted to the TFE data. This model is then validated with
other independent data.

\begin{figure}

{\centering \includegraphics[width=0.8\linewidth]{figures/chap2/fisher1} 

}

\caption{Simulated (SPA model) and observed sapflow for a drought experiment in the Amazon; Fisher et al. 2007}\label{fig:f224}
\end{figure}

Figure \ref{fig:f225} shows the simulations og gs and GPP, black:
control plot, grey: dry plot.

\begin{figure}

{\centering \includegraphics[width=0.8\linewidth]{figures/chap2/fisher2} 

}

\caption{Simulated (SPA model) gs and GPP for a drought experiment in the Amazon. Fisher et al. 2007}\label{fig:f225}
\end{figure}

Conclusions: this model worked better than empirical models, which
really predict a collapse of the system. This model is doing better
because it simulates the water transport through the tree. Models need
to take into account deep rooting, trees do have access to water in the
deep soil, even in the dry period.

\chapter{Modelling radiation, vegetation canopies, and energy
balance}\label{modelling-radiation-vegetation-canopies-and-energy-balance}

\chaptermark{Light}

\section{Introduction}\label{introduction}

This chapter deals with upscaling processes from leaf level to canopy
level, but still on a (very) short timescale (\textasciitilde{}hours,
sub-daily).

We want a model that is consistent over different scales. If we measure
parameters at the leaf level and put those parameter values in our
model, we want the results to be as consistent as possible with the data
at the canopy scale, for example data from eddy covariance flux towers.
Also the other way around, if we optimize model parameters based on flux
tower measurements, we want the resulting leaf parameters to be as close
as possible to the parameters observed in the field at leaf level.

We first refresh some basics on radiation. The sun emits shortwave
radiation, the earth emits longwave radiation (invisible). Longwave
radiation has its maximum intensity at a much longer wavelength and
contains a much lower amount of energy (surface under the spectral
intensity curve). The spectrum of wavelengths the sun emits is different
when we measure it on the earth surface (compared to the spectrum at the
top of the atmosphere, as the atmosphere absorbs some of the radiation,
e.g.~in the ultraviolet light and infrared light absorption takes place.
Diffuse radiation (which indirectly reaches the surface) is more shifted
towards the blue light, which is one of the wavelengths used by plants.
On a sunny day, both diffuse and direct radiation reach the plants.
Under full cloud cover, plants only receive diffuse radiation. 50\% of
the shortwave radiation is PAR (Photosynthetic Active radiation). PAR is
typically expressed as a photon flux density (micromols of photons per
m² per s). The lower the zenith angle (higher sun), the higher the PAR
fraction in the diffuse and direct radiation that reaches the earth
surface. The relative PAR fraction is higher in diffuse radiation.

Secondly, we need to refresh some basics on canopy structure parameters.
The leaf area index (LAI) is the total leaf area per unit of ground
area. Sometimes it is also defined as the total double-sided surface of
the leaves (so the two sides of the leaf), but this is not commonly
used. The projected leaf area depends on the orientation of the leaf.
Vertically oriented leaves have a smaller projected leaf area than if
they would be horizontally oriented. Figure \ref{fig:f31} shows a
generalized example of the leaf area density distribution for different
vegetation types. The peak of leaf area density is different depending
on the vegetation type, with a peak higher in the vegetation for forests
and more in the middle for grass. The total LAI is calculated by
integrating the function \(a(z)\) showed in figure \ref{fig:f31} over
the height. If you keep in mind that trees are higher than grass, it is
clear that the total LAI will be bigger for trees than for grass.

\[
total \, LAI = \int_0^{h_c}a(z) \cdot dz
\]

\begin{figure}

{\centering \includegraphics[width=0.8\linewidth]{figures/chap3/f31_LAD} 

}

\caption{Generalized profiles of leaf area density in plant canopies. (Bonan)}\label{fig:f31}
\end{figure}

The cumulative LAI represented in Figure \ref{fig:f32} shows the LAI
above a certain point in the canopy. It is calculated as the integral of
the function between that point and the canopy height. This figure also
shows the cumulative wood area index. If we make the sum of the LAI and
the wood area index (WAI) we get the plant area index PAI, which is the
total area of plant material per unit of ground area.

\[
cumulative \, LAI = \int_z^{h_c}a(z) \cdot dz
\]

\begin{figure}

{\centering \includegraphics[width=0.8\linewidth]{figures/chap3/f32_cLAI} 

}

\caption{Cumulative LAI and WAI in a deciduous oak-hickory forest. (Bonan)}\label{fig:f32}
\end{figure}

The leaf angle (distribution) is a key structural property of a
vegetation canopy. It will largely determine how light will penetrate in
the canopy, and is important for other processes, like precipitation
interception. Figure \ref{fig:f33} illustrates that the leaf angle can
be expressed as the angle between the leaf and the horizontal plane, or
as the angle between the zenith line and the line perpendicular to the
leaf surface.

\begin{figure}

{\centering \includegraphics[width=0.8\linewidth]{figures/chap3/f33_Langle} 

}

\caption{Illustration of a leaf (thick line) oriented at an angle Θℓ to horizontal. (Bonan)}\label{fig:f33}
\end{figure}

Leaf angle distributions describe a probability density function of leaf
angles of a plant or vegetation. Base on these distribution we can
distinguish differences between plants and vegetation types. Planophile
leaf angle distributions: most leaves are horizontally oriented;
erectophile: most leaves are vertically oriented; plagiophile: average
leaf inclination angle is 45°; spherical: mostly used in modelling,
corresponds somehow to erectophile, it is a kind of idealized form. This
distributions are illustrated in Fig. \ref{fig:f34} Subplot b indicates
the cumulative distribution of the leaves.

\begin{figure}

{\centering \includegraphics[width=0.8\linewidth]{figures/chap3/f34_angle_distr} 

}

\caption{Planophile, erectophile, plagiophile, and spherical leaf angle distributions showing (a) the probability density function f(Θℓ) and (b) the cumulative distribution F(Θℓ). (Bonan)}\label{fig:f34}
\end{figure}

Figure \ref{fig:f35} illustrates the impact of leaf orientation in two
crops on light penetration in the canopy of two crops (lower panel). The
complexity of the vegetation plays an important role in where the light
is intercepted and how much is intercepted (upper panel). In an
erectophile leaf orientation, the light interception is more evenly
spread through the vegetation than for a planophile profile, where most
of the interception takes place at the center of the vegetation.

\begin{figure}

{\centering \includegraphics[width=0.8\linewidth]{figures/chap3/f35_architecture} 

}

\caption{Illustration of leaf angle distributions and canopy architecture in general influences radiation attenuation in vegetation canopies.}\label{fig:f35}
\end{figure}

Light extinction follows a typical exponential course when it is
observed at multiple heights in vegetation, it decreases very fast at
the top of the canopy, with almost no light reaching the soil in this
example (Fig. \ref{fig:f36}). We can conclude that radiative transfer in
the canopy depends on:

\begin{itemize}
\tightlist
\item
  the LAI (amount of leaves),
\item
  the optical properties (which depend on the morphology and chemical
  composition) of the leaves
\item
  the organization of the leaves (architecture, angles,..)
\end{itemize}

\begin{figure}

{\centering \includegraphics[width=0.8\linewidth]{figures/chap3/f36_obs_profile} 

}

\caption{Profile of light and foliage in a stand of herbaceous plants approximately 130 cm tall. The horizontal axis shows transmittance as a fraction of incident radiation (top axis) and foliage mass (bottom axis) at various heights in the canopy. (Bonan)}\label{fig:f36}
\end{figure}

Radiative transfer theory aims to integrate the light environment of
individual vegetation elements (leaves, branches, \ldots{}) over the
entire canopy. It must distinguish between the specific behavior of
direct beam radiation and omnidirectional diffuse radiation and it
should account for wavelength.

\section{Radiative transfer
modelling}\label{radiative-transfer-modelling}

Radiative transfer modelling in full 3D is very complex and is a
separate branch of science. The radiative transfer schemes in vegetation
models are usually relatively simple approximations of full 3D radiative
transfer. Such schemes have a lot of assumptions: all leaves are
plane-parallel oriented, leaf layers are horizontally homogeneous. We
mostly use a one dimensional representation of the canopy
(Fig.\ref{fig:f37}), which is often used in vegetation models. Such an
approach neglects the horizontal heterogeneity of vegetation and only
assume a simplified vertical heterogeneity. A more complex model,
subplot b, represents a three dimensional representation of the canopy,
with vertical and horizontal variation.

\begin{figure}

{\centering \includegraphics[width=0.8\linewidth]{figures/chap3/f37_RT_principle} 

}

\caption{Representation of a canopy as (a) one-dimensional with a vertical profile of leaf area (shown by grayscale gradation in which darker shading denotes more leaves) that is horizontally homogenous and (b) threedimensional with vertical and spatial structure determined by crown geometry and spacing. (Bonan)}\label{fig:f37}
\end{figure}

\subsection{Leaf optical properties}\label{leaf-optical-properties}

What happens to light when it reaches a leaf? It can be absorbed,
reflected or transmitted. We express these in relative fractions as
\(\rho\) = reflectance (albedo), \(\tau\) = transmittance, \(\alpha\) =
absorbance. Leaves absorb PAR, with a small negative peak at the green
light, which they reflect more compared to blue and red light. Also in
the infrared wavelength band, radiation is absorbed (mostly longwave
radiation). UV light is absorbed by the cuticula to protect the leaves.
The reflectance and transmittance curves are inverse curves from the
absorptance curve. Here, infrared radiation and green light are
reflected/transmitted. The clear difference between VIS and NIR is used
in remote sensing for vegetation indices (e.g.~NDVI). Transmittance
depends on the thickness of the leave.

\begin{figure}

{\centering \includegraphics[width=0.8\linewidth]{figures/chap3/f37_leaf_optical} 

}

\caption{Spectrum of absorptance, reflectance and transmittance of a typical plant leaf (Jones, 2014)}\label{fig:f37b}
\end{figure}

The albedo of the total canopy is typically lower than that of its
individual leaves, as light that is transmitted or reflected by one
leave can be absorbed by another leaf (multiple scattering) (see Table
below). Reflectance of needles is lower than the reflectance of
broadleaves.

\begin{center}
\captionof{table}{Table showing typical reflectance and absorptance values for leaves and vegetation canopies of different Plant Functional Types (PFT).(Jones, 2014)}
\label{table:reflectance}

\begin{center}\includegraphics[width=0.8\linewidth]{figures/chap3/f38_table_optical} \end{center}
\end{center}

\subsection{Light transmission without
scattering}\label{light-transmission-without-scattering}

The most basic radiative transfer models, do not consider scattering (no
reflection on the leaf). The Beer-Bouguer-Lambert law describes the
attenuation of light through a homogeneous medium with thickness \(dz\)
in absence of scattering. The absorbed radiation is proportional to the
thickness of the medium, the amount of light that comes in and a
constant \(K\).

\[
dI^{\downarrow} = -KI^\downarrow dz
\]

with \(dz\): thickness; \(I^\downarrow\): incoming radiation; \(K\):
constant; \(dI^\downarrow\): absorbed radiation in the medium.

After integration we can calculate the radiation \(I\) leaving the
medium from \(I_0\), the radiation intensity entering the medium.

\[
I^{\downarrow} = I^{\downarrow}_0\cdot e^{-Kz}
\]

For leaves in a canopy, and based on probability theory, Monsi and Saeki
(1953) applied the Beer law for a volume with crown thickness \(dz\),
with randomly distributed horizontal elements (leaves) with density
\(N\) (number of elements per volume, see Fig. \ref{fig:f39}.

\[
dI^{\downarrow} = - I^{\downarrow} a N dz
\]

where the total leaf area in the volume can be calculated as:

\[
L = aNz
\]

with \(N\): density of leaf elements; \(L\): total leaf area; \(a\):
surface of individual leaf perpendicular to the path of light; \(z\) the
total thickness of the canopy.

After integration we get an exponential function that describes the
radiation extinction in the canopy:

\[
I^{\downarrow} = I^{\downarrow}_0\cdot e^{-L}
\]

\begin{figure}

{\centering \includegraphics[width=0.8\linewidth]{figures/chap3/f39_beer} 

}

\caption{Transmission of solar radiation through a homogeneous medium in the absence of scattering. In this example, n non-overlapping opaque particles each with cross-sectional area a oriented perpendicular to the path of light are placed in a medium with cross-sectional area A and thickness dz. The radiation absorbed in the medium is dI.(Bonan)}\label{fig:f39}
\end{figure}

However, in real canopies not all leaves are horizontal, and the light
is not coming in perpendicular as assumed in the original equation of
Monsi and Saeki. In the exponential function, \(L\) should therefore be
adapted to the projected leaf area perpendicular to the light beams, as
the model assumes that light comes in perpendicularly.

\[
I^{\downarrow} = I^{\downarrow}_{sky,b}\cdot e^{-K_bL}
\]

Where \(K_b\): direct beam extinction coefficient, and \(LH\)
(horizontal leaf area)= \(K_b L\). And \(I^{\downarrow}_{sky,b}\) is the
incident radiation at the top of the canopy. \(K_b\) describes how
severe light is attenuated in the canopy depending on the LAI. A higher
\(K_b\) indicates a faster extinction, so a more dark forest. Fig.
\ref{fig:f310} describes the light transmittance with increasing LAI for
different \(K_b\) values.

\begin{figure}

{\centering \includegraphics[width=0.8\linewidth]{figures/chap3/f310_Kb} 

}

\caption{Transmission of direct beam radiation τb in relation to leaf area index for typical values of the extinction coefficient Kb. (Bonan)}\label{fig:f310}
\end{figure}

The extinction coefficient depends on solar zenith angle and leaf
orientation (as illustrated in Fig.\ref{fig:f311}).

\[
K_b = \frac{L_H}{L} = \frac{G(Z)}{cosZ}
\]

With \(Z\): solar angle; \(\Theta_l\): leaf angle; \(G(Z)\): function
depending on the zenith angle (shown in figure \ref{fig:f311}). Divided
by \(cosZ\) because we project to the horizontal plane. \(L_h\): leaf
area projected on the horizontal plane.

\begin{figure}

{\centering \includegraphics[width=0.8\linewidth]{figures/chap3/f311_LLh} 

}

\caption{Extinction coefficient in relation to solar zenith angle Ζ and leaf inclination angle. In each panel, a unit leaf area (L = 1), shown with a thick line, is projected onto a horizontal surface $L_H$ so that $K_b = L_H$. The leaf inclination angle is 0° (bottom panels), 30° (middle panels), and 60° (top panels). In the left and middle columns, the leaf is oriented towards the Sun and the solar zenith angle is 0° (left column) and 15° (middle column). In the right column, Ζ = 15°, but the leaf is oriented away from the Sun. In each panel, the arrows indicate the solar beam (Bonan)}\label{fig:f311}
\end{figure}

The lower the sun is to the horizon (the higher the zenith angle), the
higher the extinction coefficient is, except for horizontal leaves,
which have a constant \(K_b\) of 1.

\begin{figure}

{\centering \includegraphics[width=0.8\linewidth]{figures/chap3/f312_Kb_angle} 

}

\caption{Extinction coefficients for horizontal, spherical, and vertical leaf angle distributions. (a) Direct beam radiation Kb in relation to solar zenith angle. (b) Diffuse radiation Kd in relation to leaf area index(Bonan)}\label{fig:f312}
\end{figure}

The above described formulas do not only allow us to describe the light
extinction in the canopy, but also to calculate the sun and shaded
fraction of leaf are in the canopy. If we divide the canopy into
multiple layers, we can calculate for each layer which part is in the
sun and which part is in the shade. For every canopy layer, the fraction
of sunlit leaves is equal to the fsun(x) function. It represents the
amount of light that comes through the layers above it.

\[
f_{sun} (x) = e^{-K_b x}
\]

Where \(x\) is the cumulative LAI above the layer of interest. Similarly
we can calculate the actual sunlit leaf area within a certain layer:

\[
\Delta L_{sun} = \frac{e^{-K_b x} \left(1 - e^{-K_b \Delta L} \right)}{K_b}
\]

Where \(\Delta L\) is the actual amount of leaves in that layer, this
formula calculates the difference between the sunlit fraction at the top
of the layer minus the ``reduction of sunlit fraction'' within the
layer. If we apply that formula on a full canopy we can calculate the
sunlit leaf area of the full canopy \(L_{sun}\):

\[
L_{sun} = \frac{\left(1 - e^{-K_b L} \right)}{K_b}
\]

\(L_{sun}\): integrated over the whole canopy. Based on this equation
you can calculated the amount of leaves that is sunlit in the canopy.
And also the shade fraction based on the difference with the total LAI.

In the example below (Fig. \ref{fig:f313}) the sunlit and total LAI are
compared between a vegetation with horizontal and with vertical foliage.
The latter has more sunlit leaves, as top leaves don't shade the
underlaying leaves. In the horizontal foliage, more light is absorbed,
so less light is transmitted than in the vertical foliage.

\begin{figure}

{\centering \includegraphics[width=0.8\linewidth]{figures/chap3/f313_sun_shade} 

}

\caption{Radiative transfer and sunlit leaf area index for a canopy of horizontal leaves (top panels) with Kb = 1 and vertical leaves (bottom panels) with Kb = 0.112. The left-hand panels show a canopy consisting of four layers of leaves. Each thick black line represents a leaf area index of 0.1 m2 m–2. The thin lines depict interception or transmission of beam radiation with a zenith angle of 10°. The middle panels show cumulative leaf area index and sunlit leaf area index with depth in the canopy. The right-hand panels show direct beam transmittance with depth in the canopy. (Bonan)}\label{fig:f313}
\end{figure}

When we study the sunlit LAI with increasing cumulative LAI
(Fig.\ref{fig:f314}) for different leaf orientations, the same course of
the curve applies: first, there is a linear increase of the amount
sunlit leave are index with increasing LAI, after which a saturation
occurs when the vegetation becomes more dense (higher LAI). The level of
saturation depends on the leaf angle distribution.

\begin{figure}

{\centering \includegraphics[width=0.8\linewidth]{figures/chap3/f314_sunlit} 

}

\caption{Sunlit leaf area index in relation to total leaf area index for horizontal, spherical, and vertical foliage orientations with solar zenith angle Ζ = 30°. Kb = 1, 0.577, and 0.368 for horizontal, spherical, and vertical foliage. (Bonan)}\label{fig:f314}
\end{figure}

A final complicating factor that we want to highlight here is that
leaves and canopies are not distributed randomly in space, but occur
clumped. Leaf clumping means that leaves appear in groups and that there
are larger gaps in between. Clumping can occur at different scales: at
leaf level, canopy level or landscape level (Fig.\ref{fig:f315}). The
clumping factor (\(\Omega\)) adjusts the amount of light that goes
through the vegetation (more light passes through). The factor is added
in the radiation extinction equations to adjust the extinction
coefficient. Where a clumping factor of 1 represents randomly
distributed leaves, while values lower then 1 represent clumped leaves
(e.g.~0.74 for needleleaf evergreen forest).

\[
I^{\downarrow} = I^{\downarrow}_{sky,b}\cdot e^{-K_b \Omega L}
\]

\[
f_{sun} (x) = e^{-K_b \Omega x}
\]

\begin{figure}

{\centering \includegraphics[width=0.8\linewidth]{figures/chap3/f315_clumping} 

}

\caption{Images illustrating leaf/canopy clumping a various scales: leaf, crown, stand.}\label{fig:f315}
\end{figure}

\subsection{Diffuse transmittance}\label{diffuse-transmittance}

Diffuse radiation is omnidirectional, and therefore it penetrates better
into the canopy. The formula below represents the diffuse transmittance.
The last part of the equation represents the contribution of a certain
sky angle to the total sky irradiation (e.g.~contribution of zenith
angle between Z1 and Z2 in Fig.\ref{fig:f316}). The exponential curve
with the leaf area \(L\) is the light extinction curve, but here it is
also integrated over the zenith angle.

\[
\tau_d = \frac{I^{\downarrow}}{I^{\downarrow}_{sky,d}} \int_0^{\pi/2} exp\left[- \frac{G(Z)}{cosZ}L \right] sinZcosZdz
\]

\begin{figure}

{\centering \includegraphics[width=0.8\linewidth]{figures/chap3/f316_diffuse} 

}

\caption{Illustration of direct beam and diffuse radiation. The sky forms a bowl, or inverted hemisphere, over a horizontal surface. Shown is a cross section of the sky hemisphere. Direct beam (solid line) originates from the  direction of the Sun with zenith angle Ζ. Diffuse radiation (dashed lines) can be treated as independent beams of radiation each with an angle Ζ. The shaded region is the relative contribution between sky angles Ζ1 and Ζ2 to total sky irradiance.(Bonan)}\label{fig:f316}
\end{figure}

The more leaves, the less diffuse light is transmitted through the
canopy. The contribution of the zenith angles is shown on
Fig.\ref{fig:f317}. There is more light transmitted in the 30°-60°
zenith angle range than in the other two groups. Thus perpendicular
diffuse sunlight penetrates less easy in the canopy than sunlight under
a 30°-60° angle.

\begin{figure}

{\centering \includegraphics[width=0.8\linewidth]{figures/chap3/f317_diff_trans} 

}

\caption{Transmittance of diffuse radiation τd in relation to leaf area index for a spherical leaf distribution. Show are the transmittances for sky zones of 0°–30°, 30°–60°, and 60°–90° and also the total transmittance. Fill patterns show the contribution of each sky zone to total transmittance.(Bonan)}\label{fig:f317}
\end{figure}

It is also interesting to compare the diffuse transmission to the direct
transmission. The dashed line (Fig.\ref{fig:f318}) indicates the
transmittance for diffuse light (omnidirectional), the solid lines
indicate the transmittance for direct light at different solar zenith
angles. The transmittance of diffuse light is higher than that of direct
light for solar zenith angles greater than 40-50°, direct light
penetrates less easy through the canopy compared to diffuse light.

It is important to include diffuse radiation into models as it
penetrates deeper into the canopy, it is shifted towards the blue light,
which is used by plants for photosynthesis, and because plants depend on
it in cloudy conditions.

\begin{figure}

{\centering \includegraphics[width=0.8\linewidth]{figures/chap3/f318_diff_dir_trans} 

}

\caption{Transmission of solar radiation through a canopy with spherical leaf distribution in relation to leaf area index. The solid lines show direct beam transmittance τb for solar zenith angles of 0°–80° (in 10° increments).The dashed line shows the diffuse transmittance τd. (Bonan)}\label{fig:f318}
\end{figure}

\subsection{The Norman Model(1979)}\label{the-norman-model1979}

In the Norman model, the canopy is divided into multiple layers of
typically 0.5 LAI per layer. For each layer the diffuse, direct and
longwave radiation balance is calculated: reflection, absorption and
transmission. The model thus accounts for scattering. We do not discuss
the equations in detail, but the large number of equations have to be
solved numerically, which is very computational expensive, especially in
model versions where a sunlit and shade fraction per leaf layer are
considered.

\begin{figure}

{\centering \includegraphics[width=0.8\linewidth]{figures/chap3/f319_Norman} 

}

\caption{Radiative fluxes in a canopy of N leaf layers. The vertical profile is oriented with i = 1 the leaf layer at the bottom of the canopy, leaf layer i + 1 above layer i, and i = N the leaf layer at the top of the canopy. Each layer has a leaf area index ΔL. is the downward diffuse shortwave flux onto layer i, is the upward diffuse shortwave flux above layer i, and is the unscattered direct beam flux onto layer i. and are the corresponding downward and upward fluxes of longwave radiation. These depend on leaf Tℓand ground Tg temperatures. Thick arrows denote boundary conditions of diffuse solar radiation , direct beam solar radiation, and atmospheric longwave radiation at the top of the canopy.(Bonan)}\label{fig:f319}
\end{figure}

\subsection{The Goudriaan and van Laar Model
(1994)}\label{the-goudriaan-and-van-laar-model-1994}

This is an analytical solution of the model, which integrates over a
single calculation step. To be able to do so, one integrated extinction
coefficient for all layers is assumed (K'b, adapted extinction
coefficient). One light extinction coefficient is thus used for diffuse
and direct radiation together, which is the major simplification in the
model. This coefficient also includes a clumping factor. This model is
widely used due to its simplicity and low computational demand. It can
be applied to a single layer or multi-layer canopy. Due to its
assumptions, this model produces more errors, especially for diffuse
light (under cloudy conditions).

\begin{figure}

{\centering \includegraphics[width=0.8\linewidth]{figures/chap3/f320_goudriaan} 

}

\caption{ Derivation of absorbed direct beam solar radiation for a leaf layer with leaf area index ΔL (Goudriaan 1982). $ρ_c$ is the reflectance of the leaf layer.(Bonan)}\label{fig:f320}
\end{figure}

\subsection{The Two-Stream
approximation}\label{the-two-stream-approximation}

This model takes ``the best of both worlds''. It is called ``two
stream''" because it calculates light penetration separately for direct
and diffuse radiation, but it has an analytical solution. It can be used
for multilayer models.

\begin{figure}

{\centering \includegraphics[width=0.8\linewidth]{figures/chap3/f321_two_stream} 

}

\caption{Fluxes for (a) direct beam and (b) diffuse radiation in the twostream approximation for a canopy with leaf area index L.(Bonan)}\label{fig:f321}
\end{figure}

The three approaches (Norman/Goudriaan/two stream) are all based on the
same theory. They are implemented in a different way, but the results
are very similar for direct light. However they show substantial
deviations for diffuse light, shaded leaves and denser canopies.

\subsection{Longwave radiation}\label{longwave-radiation}

Longwave radiation is important for closing the energy balance. To
simulate longwave radiation transfer in canopies, similar approaches as
above are used, but the emission of longwave radiation by the ground,
and leaf surfaces need to be accounted for.

\begin{figure}

{\centering \includegraphics[width=0.8\linewidth]{figures/chap3/f322_LW} 

}

\caption{Longwave radiation fluxes represented for a single leaf layer.(a) Norman’s (1979) numerical model. Shown is the radiative balance for leaf layer i + 1 located above leaf layer i. (b) A simplified model to allow only forward scattering and to permit an analytical solution integrated over a canopy. In both panels, emitted radiation is excluded. Thick lines denote fluxes incident onto the layer. (Bonan)}\label{fig:f322}
\end{figure}

\section{Representing canopy structure in
models}\label{representing-canopy-structure-in-models}

\subsection{Big-leaf models}\label{big-leaf-models}

Big leaf models can have a shaded and a sunlit fraction in the canopy,
or can consider sunlit leaves only. The forest/vegetation is represented
as one big leaf, with a leaf area equal to the total leaf area of the
forest/vegetation. Such models do not account for the vertical structure
of the canopy. These models calculate properties of the big leaf, by
upscaling. For example by caluculating an average stomatal or caopy
conductance. Evapotranspiration can then be calculated by applying the
Penman-Monteith equation and a canopy level conductance. The leaf
boundary layer is replaced by the atmosphere surface layer and gsw is
the average stomatal conductance of the whole forest.

\begin{figure}

{\centering \includegraphics[width=0.8\linewidth]{figures/chap3/f323_bigleaf} 

}

\caption{Scaling of leaf fluxes to the canopy using a big-leaf model. (a) Shown are leaf sensible heat, transpiration, and CO2 fluxes in relation to various conductances. Fluxes are exchanged between the leaf and air around the leaf. Also shown is the total resistance. (b) Shown are big-leaf canopy fluxes in which leaf fluxes are scaled by the average conductance and leaf area index and are further modified by turbulent transport in the atmospheric surface layer. Surface layer processes are commonly omitted for CO2 exchange. Only a single big leaf is shown, but separate sunlit and shaded big leaves can be similarly depicted. (Bonan)}\label{fig:f323}
\end{figure}

It is not an easy task to integrate GPP (e.g.~calculated by the Farquhar
model of Chapter 2) over a canopy represented by a big leaf. We cannot
just give the total light absorbed by the canopy to a photosynthesis
model, as we need to account for the non-linearity in the process. The
task is easier if Amax is assumed to decrease exponentially together
with light availability I the canopy, which is a reasonable assumption,
as we know that leaf nitrogen content is decreasing from the top to the
bottom of the canopy. As such we are assuming a ``photosynthesis
extinction''" instead of a light extinction in the canopy.

\[
GPP = \int_0^LA(x)dx
\]

\[
A_{max}(x) = A_{max0}e^{-K_bx}
\]

\[
GPP = \int_0^{L}A(x)dx = A(0) \left[\frac{1 - e^{-K_b L}}{K_b} \right]
\]

However, this is a rough approach, and it does not account for the
nonlinear canopy responses. A better approach is to include a sunlit and
shade fraction of the leaves. Where the GPP is calculated as the sum for
sunlit and shaded leaves separately. Each fraction has a separate value
for \(g_s\), \(A_m\) etc\ldots{}

\[
GPP = \left[A_{sun} f_{sun} + A_{shade} (1 - f_{sun}) \right]L
\]

Big leaf models that use the Fraquhar approach often assume an
integrated Vcmax value for the canopy, calculated based on an
exponential profile of \(V_{cmax}\).

\[
V_{cmax}(x) = V_{cmax0} e^{-K_n x}
\]

\begin{figure}

{\centering \includegraphics[width=0.8\linewidth]{figures/chap3/f324_vcamx_profile} 

}

\caption{Canopy profiles of relative photosynthetic capacity in relation to cumulative leaf area index. Thin lines show exponential profiles using values of Kn for 16 temperate broadleaf forests and two tropical forests ranging from 0.10 to 0.43 (Lloyd et al. 2010). The two thick lines show observed profiles of Vcmax and Jmax from Niinemets and Tenhunen (1997) obtained for sugar maple (Acer saccharum). (Bonan)}\label{fig:f324}
\end{figure}

\subsection{Multilayer models}\label{multilayer-models}

Account for al vertical heterogeneity, such as light, leaf physiology,
leaf traits and even micrometeorology, e.g.~RH is different at different
leaf layers in the model. All equations (fluxes, energy balance) are
calculated for each layer, and GPP is the sum of the photosynthesis of
all layers. Figure \ref{fig:f325}: each layer is a big leaf to calculate
all leaf level processes. Radiative transfer and leaf fluxes are thus
simulated for each layer. Sometimes, even leaf hydraulics are simulated,
to simulate leaf water potential and its influence on stomatal
conductance. Such models are computationally expensive, especially
because they is solved for each layer a set of equations in an iterative
way (Fig. \ref{fig:f326}). Nevertheless, such multilayer models are very
common in large scale land surface models. Be aware that horizontal
heterogeneity is not considered in these models, see later chapters for
that.

\begin{figure}

{\centering \includegraphics[width=0.8\linewidth]{figures/chap3/f325_multilayer_process} 

}

\caption{Overview of the main processes in a multilayer canopy model.The canopy is represented by N leaf layers with layer i + 1 above layer i. (a) Diffuse and direct beam solar radiation is transmitted or intercepted. The intercepted portion is absorbed or scattered in the forward and backward direction. Longwave radiation is similar to diffuse radiation. (b) Leaf sensible heat, transpiration, and CO2 fluxes depend on absorbed radiation and leaf boundary layer and stomatal conductances. Sensible heat is exchanged from both sides of the leaf. Water vapor and CO2 can be exchanged from one or both sides of the leaf depending on stomata. Leaf temperature is the temperature that balances the energy budget. (c) Stomatal conductance depends on leaf water potential. Plant water uptake for a canopy layer is in relation to belowground soil and root conductance and aboveground stem conductance acting in series and also a capacitance term. (d) Scalar profiles are calculated from a conductance network. Leaf fluxes provide the source or sink of heat, water vapor, and CO2, along with soil fluxes. (e) Sensible heat, latent heat, and heat storage in soil depend on the ground temperature that balances the soil energy budget. (f) The wetted fraction of the canopy layer depends on the portion of precipitation that is intercepted. (Bonan)}\label{fig:f325}
\end{figure}

\begin{figure}

{\centering \includegraphics[width=0.8\linewidth]{figures/chap3/f326_multilayer_solving} 

}

\caption{Flow diagram of processes in a multilayer canopy model. The shaded area denotes leaf processes resolved at each layer in the canopy. This is a generalized diagram of the required calculations for a dry leaf. Specific models differ in how the equation set is solved and the iterative calculations. Evaporation of intercepted water requires additional complexity.(Bonan)}\label{fig:f326}
\end{figure}

\subsection{3D ray tracing models}\label{d-ray-tracing-models}

3D ray tracing models are not (yet) use as part of operational
vegetation models. They rather for a separate group of models, focus on
radiative transfer only (they do not simulate other processes such as
photosynthesis). These models are typically used to simulate what a
satellite (or other remote sensing sensor) would observe in specific
wavelength bands. As such these models are very useful for
interpretation of remote sensing observations. They really simulate how
light, or individual sun rays penetrate the canopy and do that for the
different wavelengths. Based on such models we can also get better
insight in radiative transfer and derive better radiative transfer
schemes for operational vegetation models.

\begin{figure}

{\centering \includegraphics[width=0.8\linewidth]{figures/chap3/f327_DART} 

}

\caption{Example of the PROSPECT leaf optical model and the DART 3D ray tracing model.}\label{fig:f327}
\end{figure}

These models typically consider simple 3D objects with specific optical
properties. However, the newest addition to this research is using
detailed 3D reconstructions of trees from lidar scans as model input and
run the model with the lidar data as scattering objects in the 3D scene.

\begin{figure}

{\centering \includegraphics[width=0.8\linewidth]{figures/chap3/f328_TLS_RT} 

}

\caption{Example of a study that uses terrestrial laser scanning (TLS) to construct a full 3D model of a forest as input for a 3D ray tracing model (Kükenbrink et al. 2020) }\label{fig:f328}
\end{figure}

\section{Ecosystem energy balance}\label{ecosystem-energy-balance}

\subsection{Basic principles}\label{basic-principles}

The energy balance is a fundamental part of vegetation models. The goal
is to calculate how much energy comes in, how much is lost and in what
the resulting temperature is for the leaf or the surface. The energy
balance is governed by non-linear equations. However linearizations
exist (e.g.~Penman Monteith equations to calculate ET).

\subsection{Surface radiation balance}\label{surface-radiation-balance}

The formula below (Fig. \ref{fig:f329}) calculates the radiation balance

\begin{figure}

{\centering \includegraphics[width=0.8\linewidth]{figures/chap3/f329_rad_balance} 

}

\caption{Radiative balance of an opaque gray body receiving downwelling solar S and longwave L radiation.(Bonan)}\label{fig:f329}
\end{figure}

where \(R_n\): net radiation (how much energy is available for system),
\(\rho S^{\downarrow}\): reflected shortwave radiation,
\(\epsilon \sigma T^{4}\) the Stefann-Boltzman law. The equation
essentially makes the balance between how much shortwave net comes in,
and how much longwave is net reflected. The net radiation forms the
major input term of the energy balance.

\subsection{Bulk surface energy
balance}\label{bulk-surface-energy-balance}

This is the central equation of the surface energy balance:

\[
R_n = \lambda E + H + G +S
\]

The net radiation is the input of energy to the ecosystem which is
dissipated via multiple loss terms: latent heat (evapotranspiration),
sensible heat flux, ground heat flux and energy storage (in the biomass
of the ecosystem). Most of the energy goes to the latent heat to
well-watered active vegetation (e.g.~a beach forest in summer).

It is really important to close the energy balance in models. All
factors in the equation can be formulated in terms of input factors of
the model, such as incoming radiation, and in function of the surface
temperature. The equation below is rewriting the above equation in terms
of surface temperature.

\[
\sum_{\Lambda} (1 - \rho_{\Lambda} ) S^{\downarrow}_{\Lambda} + \epsilon L^{\downarrow} = Q_{a} \\ = \epsilon \sigma \theta_s^{4} + c_p ( \theta_s - \theta_{ref} ) g_{ac} + \frac{\lambda \left[ q_{sat} ( \theta_s ) - q_{ref} \right] }{ g_c^{-1} + g_{ac}^{-1}} + \frac{k}{\Delta z} (\theta_s - T_1)
\]

Where \(S\) input energy (shortwave) in a certain wave length band
(\(\Lambda\)), \(Q\) is the total gross incoming energy. At the right
side of the equation we see: the longwave loss term, sensible heat loss,
latent heat (in response to vapor pressure) , and the ground heat flux.
The surface temperature (\(\Theta_s\)) is the big leaf surface
temperature.

The big leaf latent heat (evapotranspiration), in often accounted by the
Penman-Monteith equations (as function of net radiation and vapor
pressure deficit).

\[
\lambda E = \frac{s (R_n - G) + c_p \left(q_{sat}(T_a) - q_a \right) g_{ac}}{s + (1 + \frac{g_{ac}}{g_c}) \frac{c_p}{\lambda}}
\]

\begin{figure}

{\centering \includegraphics[width=0.8\linewidth]{figures/chap3/f330_E_balance} 

}

\caption{Conductance networks for sensible heat flux (top) and latent heat flux (bottom) for various depictions of the land surface. This chapter describes the bulk surface and big-leaf canopies. (Bonan)}\label{fig:f330}
\end{figure}

\subsection{Leaf energy balance}\label{leaf-energy-balance}

Some models make the leaf energy balance, which is essentially using the
same equations as the bulk energy balance, but applying them at the leaf
(layer) level and solving team for leaf temperature. The leaf energy
balance depends on radiative forcing, boundary layer processes and
stomatal physiology. The stomatal conductance is balance photosynthesis,
transpiration and energy fluxes.

\[
c_L \frac{\delta T_l}{\delta t} = Q_a - 2 \epsilon_l T_l^4 - H(T_l) + \lambda E(T_l)
\]

\[
Q_a = \sum_\Lambda S^{\downarrow}_\Lambda (1 + \rho_{g \Lambda})(1 - \rho_{g \Lambda} - \tau_{l \Lambda}) + \epsilon_l (L_{sky}^\downarrow+ L_g^{\uparrow})
\]

The dimension of the leaf plays an important role in the solving of the
leaf energy balance, for example, the boundary layer conductance depends
on the leaf dimensions and the wind speed.

\begin{figure}

{\centering \includegraphics[width=0.8\linewidth]{figures/chap3/f331_leaf_E_balance} 

}

\caption{Biophysics and biochemistry of leaves. (a) The radiative environment consists of solar radiation (left) and longwave radiation (right). (b) Leaf fluxes include CO2, H2O, and heat through the boundary layer. These fluxes are shown as a network of conductances for the adaxial (upper) and abaxial (lower) leaf surfaces. For H2O and CO2, the conductance for each surface is obtained from stomatal and boundary layer conductances acting in series. The total conductance is defined by the adaxial and abaxial surfaces acting in parallel. (c) Stomata open to absorb CO2 for photosynthesis, but, in doing so, water is lost as transpiration. (Bonan)}\label{fig:f331}
\end{figure}

\section{Case studies}\label{case-studies-1}

\subsection{Case study 3.1}\label{case-study-3.1}

How is the amount of diffuse light impacting photosynthesis? (diffuse
radiation fertilization) In normal circumstances, the sun leaves are in
the saturated zone of photosynthesis, while shaded leaves are somewhere
on the linear increase part. If you would change the radiation
composition (e.g.~more clouds), this would result in less light for the
sunlit leaves, which has only limited impact on their photosynthesis.
For the shaded leaves on the other hand, more diffuse radiation has a
big impact on the photosynthesis. So, the hypothesis in this study is
that if you have more diffuse radiation, more photosynthesis takes place
at the canopy level. This hypothesis is tested by using data of two
fluxtower stations and a multilayer vegetation model.

\begin{figure}

{\centering \includegraphics[width=0.8\linewidth]{figures/chap3/f332_knohl1} 

}

\caption{Principle of the effect of increased diffuse raditaion on leaf/canopy photosynthesis. (Knohl et al. 2008)}\label{fig:f332}
\end{figure}

The x axis (Fig. \ref{fig:f333}) represents the diffuse radiation over
the total radiation. Photosynthesis increases with increasing diffuse
radiation an reaches a maximum at 45\% diffuse light. Transpiration
increases also but at a slower pace, and the water use efficiency
increases faster. So, forests have a higher water use efficiency when
the amount of diffuse radiation is higher (again with max but now around
0.65). On cloudy days the Rd/Rs is higher than the optimum. The net
carbon balance increased if you have more diffuse light (in relative
terms), the transpiration remains stable but the WUE is increasing (more
photosynthesis for the same amount of transpiration). However, for the
studied sites, the situation is in 61\% of the time beyond the optimum
(on cloudy days), resulting in a net decreasing impact on the net carbon
uptake.

\begin{figure}

{\centering \includegraphics[width=0.8\linewidth]{figures/chap3/f333_knohl2} 

}

\caption{Resulting impact of changing diffuse fraction on carbon and water fluxes and WUE (Knohl et al. 2008)}\label{fig:f333}
\end{figure}

\subsection{Case study 3.2}\label{case-study-3.2}

Studies have shown that the increase of atmospheric CO2 since the 1850's
caused an enhancement in plant growth, but the magnitude varies. This
enhancement also manifests itself in the enhancement of LAI (leaf area
index), the area of leafs on a specific ground area. In the paper,
Boreal Ecosystem Productivity Simulator (BEPS) was used as process-based
diagnostic model. It initially was used, as the name states, for boreal
ecosystems, but has been adapted for all ecosystems on Earth. It
simulates the impact of drivers (climate, CO2 concentration, and
nitrogen deposition) on the GPP (Gross Primary Production) and as well
carbon pools below GPP and respiration activities, generally, the whole
carbon cycle, this on a daily time interval for each pixel. It makes use
of various satellite data (assimilate LAI) and LUE (light-use
efficiency) models.

Running these simulations on three LAI series, it stated that the
enhancement of LAI contributed 12,4\% to the accumulated carbon sink. In
satellite observations it was observed that the global leaf area index
has increased over the past 40 years, and they wanted to investigate
whether this has had an effect on global carbon sink. The model used in
this study is driven by remote sensing data, it is a diagnostic and not
really a process-based model. The model calculates the GPP based on the
equation (seen today) which used LAIshade and LAIsun based on similar
equation with exponential function, clumping factor and so on. The study
aims to quantify how much the LAI increase is contributing to the
simulated carbon uptake globally. Map shows the trends of the LAI in the
world.

\begin{figure}

{\centering \includegraphics[width=0.8\linewidth]{figures/chap3/f334_chen1} 

}

\caption{Global map of LAI trend between 1981 and 2016 based on remote sensing (Chen et al. 2021).}\label{fig:f334}
\end{figure}

They calculated the different factors which are accounted for in their
model and tested how sensitive their model result was to that different
driving factors. According to the model, global ecosystem productivity
has increased (light blue line in Fig. \ref{fig:f335}). N deposition did
not a net effect. Climate change (T increase, drought stress) has
reduced the sink strength. The LAI increase had a positive impact,
contributes to the increased sink. They observed that the relative
increase of the maximum LAI per year was larger than the relative
increase of the average LAI per year. This indicates that the average
LAI increase is mainly caused by the higher maximum LAI, and not by the
lengthening of the growing season.

\begin{figure}

{\centering \includegraphics[width=0.8\linewidth]{figures/chap3/f335_chen2} 

}

\caption{Simulated impact of different factors contributing to the increased global land C sink since 1981 (Chen et al. 2021) }\label{fig:f335}
\end{figure}

\chapter{Modelling temporal and seasonal
dynamics}\label{modelling-temporal-and-seasonal-dynamics}

\chaptermark{dynamics}

The main topic of this chapter is temporal variations in fluxes and leaf
seasonality. We will study the upscaling in time, going from
instantaneous processes to daily and seasonal dynamics (temporal
dynamics), focusing on phenology.

\section{Introduction on temporal
dynamics}\label{introduction-on-temporal-dynamics}

An example shows how we use flux data to evaluate the diurnal and
seasonal cycles of models (Figure \ref{fig:f41}). The modeled data is
shown in black, the measured data in different colors, for different
variables. The model can predict diurnal (instantaneous) fluxes good,
seasonal cycles are more difficult to predict because the leaf area
changes over time.

\begin{figure}

{\centering \includegraphics[width=0.8\linewidth]{figures/chap4/f41_Krinner} 

}

\caption{Evaluation of the temporal dynamics of fluxes (Rn: Net Radiation, H: sensible heat, LE: latent heat, NEE: net ecosystem exchanges of CO2) simulated by the global model ORCHIDEE. LEFT: measured (color) and modelled "summer" diurnal cycle for each flux and each PFT. RIGHT: measured (color) and modelled seasonal cycle for each flux and each PFT. (Krinner et al. 2005)}\label{fig:f41}
\end{figure}

The temporal dynamics of fluxes respond to multiple factors and
depending on the scales we are looking at, we need to take other
processes into account. If we are interested in \textbf{instantaneous}
responses, we need to modulate the physiological processes (stomata of
leaves). On the other hand, if we look at \textbf{seasonal} responses,
we need to consider both physiology and phenology (the amount of leaves
the vegetation has). It is also important to mention \textbf{long-term}
responses; for example, if you want to simulate carbon fluxes for the
coming decades, more (slower) processes need to be considered
(e.g.~forest succession). The focus of this chapter will be on seasonal
responses and phenology. These are processes that are typically
calculated on an intermediate time step (e.g.~daily or monthly) in
vegetation models.

\section{Phenology: the background}\label{phenology-the-background}

\textbf{Phenology} is the study of periodic events in biological life
cycles and how these are influenced by seasonal and interannual
variations in climate, as well as habitat factors (such as elevation).
There is a link with carbon allocation, which means how much carbon will
be distributed over different plant compartments. The phenological
processes respond to climate variability, the so-called environmental
cue, an environmental trigger. For example, from observational data, the
ecodormancy responds both to high temperatures and longer day length. It
is a complex mechanism with many factors, and we still don't know the
underlying physiological processes.

\subsection{Trends in phenology}\label{trends-in-phenology}

There are important trends in phenology with climate change because it's
not only global average temperatures that are changing, but also spring
temperatures are advancing. Autumn temperatures are coming later in the
year. Figure \ref{fig:f42} shows an example of needle appearance dates
(for larch), where it is possible to identify a trend of needles
appearing earlier. New needles of larch trees appear more early over
time and this shift is accelerating. Shifts are occurring in the range
of weeks. Such phenological trends have an impact on ecosystem
functioning, and those impacts of shifting leaf phenology can manifest
in multiple ways (Fig. \ref{fig:f43}).

\begin{figure}

{\centering \includegraphics[width=0.8\linewidth]{figures/chap4/f42_Defilia} 

}

\caption{Larch needle appearance in Sargans 1958-2002. Dashed line is trend 1958-1999, solid line is trend 1958-2002.  (Defilia and Clot 2005)}\label{fig:f42}
\end{figure}

\begin{figure}

{\centering \includegraphics[width=0.8\linewidth]{figures/chap4/f43_Polgar} 

}

\caption{Concept of various effect that changing phenology can have on ecosystem processes (e.g. productivity or transpiration). (Polgar and Primack 2011)}\label{fig:f43}
\end{figure}

A shift in phenology has an impact on the functioning of ecosystems. In
a normal year the physiological activity (e.g.~photosynthesis) increases
after leaves are formed, reaches a maximum, and after the summer a
decrease occurs because of less efficient leaves and less light input.
Due to changes in phenology, this is influenced (direct or lagged,
positive or negative):

\begin{enumerate}
\def\labelenumi{\Alph{enumi})}
\item
  Full positive effect, faster maximum
\item
  A total increase in amount of leaves, also benefits in summer
\item
  Earlier onset of leaves cause earlier senescence of leaves, causing
  less assimilation in autumn
\item
  Faster ageing leaves: decrease activity after spring
\end{enumerate}

\section{Leaf phenology models}\label{leaf-phenology-models}

Leaf phenology models are essentially models used to update the leaf
area state variable for PFTs in vegetation models.

\textbf{Phenology is typically a PFT dependent process in vegetation
models:}

\begin{itemize}
\tightlist
\item
  Evergreen PFTs typically have a constant background leaf turnover
  rate.
\item
  Summer green and rain green PFTs have a leaf phenology driven by
  environmental factors (temperature, water, light, \ldots{}).
\end{itemize}

\textbf{How to model phenology?} We don't have (bio)physical models,
only empirical models:

\begin{itemize}
\item
  Option 1: prescribed phenology
\item
  Prescribed dates of leaf onset and offset (4 dates to determine carbon
  allocation phases to leaves).
\item
  Or data assimilation of remote sensing data (LAI).
\item
  Option 2: prognostic phenology
\item
  A complete leaf phenology model includes carbon allocation, leaf onset
  and growth, and litterfall as influenced by environmental conditions
  (still empirical).
\end{itemize}

\subsection{Prescribed phenology}\label{prescribed-phenology}

An option to model phenology is using the dates of leaf onset and offset
(typical 4 dates; start leaf formation, maximal activity, start
senescence, all leaves `are gone') .

An alternative is to fit a function to remote sensing data. What is
typically done is fitting a logistic function of time to remote sensing
observations of leaf phenology:

\[
y(t)=\frac{c}{1+e^{a+bt}}+d
\]

If we observed phenology with satellites, it could be done using
vegetation indices (NDVI, EVI, \ldots{}). Based on observation indices
throughout the year, we could define dates to phenological events
(Figure \ref{fig:f44})

\begin{figure}

{\centering \includegraphics[width=0.8\linewidth]{figures/chap4/f44_zhang} 

}

\caption{Sample time series of MODIS EVI data and estimate phenological transition dates for a mixed forest pixel in New England. Diamonds: EVI data, solid line with stars: fitted logistic model. (Zhang et al. 2003)}\label{fig:f44}
\end{figure}

This approach allows plotting maps with the key days for phenology
(Figure \ref{fig:f45}) and seeing the spacial variance in shifts in
phenology. Such datasets can be used to train prognostic models.

\begin{figure}

{\centering \includegraphics[width=0.8\linewidth]{figures/chap4/f45_zhang_map} 

}

\caption{Maps of phenological transition dates for New England. (Zhang et al. 2003)}\label{fig:f45}
\end{figure}

\subsection{Budburst models}\label{budburst-models}

Budburst models are prognostic models and are the most developed ones
(compared to senescence models for example). They simulate when a bud is
becoming a leaf. These models are based on environmental signals (cues),
and the most common approaches are:

\begin{itemize}
\tightlist
\item
  Degree day sums (summation of daily temperature above 5°, for
  example).
\item
  Chilling requirement (to pass the danger of spring frost)
  (accumulation of cold T).
\item
  Photoperiod (daylength).
\item
  Soil moisture (for rain green plants).
\end{itemize}

Some models combine approaches of the above, and others use just one
approach. Most of the time, air temperature is used in these models (and
not bud temperature).

\textbf{Growing degree-day model}

It's the simplest version of the budburst models but also the most
widely used. However, this approach only works well in systems where
temperature is the only limiting factor:

\[
GDD = \sum_{T>T_{base}}(T-T_{base}) \\
budburst:GDD>F^*
\]

Identify some threshold value of GDD \(F^{∗}\) that corresponds to the
metric of interest, which is budburst in this case. Similar models can
be made for fruiting or harvest.

\textbf{Photoperiod}

Used only for some species where there is empirical evidence that
photoperiod plays a dominant role. Typically not used in global models.

\textbf{Models including chilling}

More complicated phenology models use a combination of growing degree
days and a chilling requirement (amount of cold days). There are diffent
ways to implement such a combination:

\begin{itemize}
\item
  \emph{Sequential models}: forcing (GDD) only starts when the chilling
  requirement is met.
\item
  \emph{Parallel models}: chilling and forcing accumulated in parallel
  and critical values then applied to both.
\item
  \emph{Alternating models}: the temperature \(F^∗\) is a decreasing
  function of chilling.
\end{itemize}

\subsection{Snescence models}\label{snescence-models}

Process of leaf ageing. Many models are based on daylength
(photoperiod), but many others use temperature or drought. Some models
are also driven by the negative carbon balance of the leaf.

\subsection{Leaf age}\label{leaf-age}

Leaf age is quite important because if you want to simulate fluxes
throughout the season, it will depend not only on the number of leaves
but also on the physiological activity of leaves related to their
physiological age (e.g.~young leaves perform typically photosynthesis at
a higher). For example, the leaf's functional traits (nutrient content
and water content) will change over time. This effect is especially
important for evergreen species. If we want to have a good simulation
for boreal forests, we should account for leaf age.

Some models track (big) leaf age and link the values of leaf parameters
to leaf age. Figure \ref{fig:f46} shows an example based on data where
Vcmax is linked to leaf age. The photosynthetic capacity is typically
reduced significantly from a certain leaf age threshold.

Some models also use leaf age classes, where it is assumed that a
certain fraction of the leaf area of the system is young and a different
fraction is constituted of middle-aged leaves. Of course, with the
beginning of the growing season, more young leaves are added, but the
leaves aged and change fraction. However, the majority of vegetation
models do not account for leaf age classes.

\begin{figure}

{\centering \includegraphics[width=0.8\linewidth]{figures/chap4/f46_vc_age} 

}

\caption{Relation  between Vcmax and leaf age in the ED2 vegetation model. (Kim et al. 2011)}\label{fig:f46}
\end{figure}

\subsection{Phenology in DGVMs}\label{phenology-in-dgvms}

A few general facts are important to know concerning phenology in global
vegetation models:

\begin{itemize}
\item
  The precise way to model phenology is highly uncertain
\item
  Vegetation models differ significantly in detail for modelling
  phenology
\item
  The phenology models are mainly based on empirical equations and
  parameters
\item
  Phenology is an important process in monitoring, modelling and
  understanding vegetation dynamics and their response to climate
  variations.
\item
  The growing amount of observational data on phenology at various
  scales will allow us to make better phenology models in the future.
\item
  Likely that for some areas at least, species-specific (or slightly
  broader groupings of species) parameterizations of phenology need to
  be considered rather than just broad PFT definitions.
\end{itemize}

\subsection{Phenology in the tropics}\label{phenology-in-the-tropics}

Phenology is very complex if we focus on evergreen tropical forests.
Figure \ref{fig:f47} shows an example of the Yangambi reserve (centre of
the Congo basin) classified as a semideciduous tropical forest. Based on
digitized historical data, we can make an overall summary (Figure
\ref{fig:f48}) showing the phenology complexity of the tropical
rainforests. It is very hard to see an overall pattern. In the
semi-deciduous forest, many trees lose their leaves at (ir)regular
moments. Evergreen species are continuously dropping leaves, whereas
deciduous trees have more specific periods (more clear seasonality).
Most models in evergreen forests assume that the leaf area is (more or
less) constant over time, but this figure shows that in reality the leaf
area can also change over time in evergreen forests. An attempt of
improving phenology modelling in evergreen forests is discussed in case
study 4.2.

\begin{figure}

{\centering \includegraphics[width=0.8\linewidth]{figures/chap4/f49_junglerythms} 

}

\caption{Example of how manual historical phenology observations in Yangambi (DR Congo) are trasnlated in a visual phenology pattern for a single tree species. (junglerythms.org))}\label{fig:f47}
\end{figure}

\begin{figure}

{\centering \includegraphics[width=0.8\linewidth]{figures/chap4/f410_kearsley} 

}

\caption{Overview of species-specific timing of onset of leaf phenophases for evergreen and deciduous species in tropical forest in Yangambi (DR Congo). The median timing of the onset of leaf senescence and turnover is indicated for each species. Species-specific bootstrapped 95\%-confidence intervals are indicated with a line segment. Species are arranged according to the variability in the timing of the phenophase, with species with the lowest uncertainty at the outer edge and continuing towards the center. Species with an annual (full circles) or sub-annual (crosses) fourier-based seasonality are indicated. Grey shaded areas represent the average timing of the long and short dry seasons (LD and SD; monthly precipitation < 150 mm), separated by the long and short wet seasons (LW and SW). (Kearsley et al. 2021))}\label{fig:f48}
\end{figure}

\section{Case studies}\label{case-studies-2}

\subsection{Case study 4.1}\label{case-study-4.1}

\textbf{Richardson, A. D., Anderson, R. S., Arain, M. A., Barr, A. G.,
Bohrer, G., Chen, G., \ldots{} \& Xue, Y. (2012). Terrestrial biosphere
models need better representation of vegetation phenology: results from
the N orth A merican C arbon P rogram S ite S ynthesis. Global Change
Biology, 18(2), 566-584.}

The study of Richardson et al.(2012) is part of the North American
Carbon Program, an initiative that is stuying the North American land
carbon cycle use various data platforms, including multiple flux tower
stations. In this specific project, 14 vegetation models are compared
and ran for the same flux towers with a very detailed protocol to make
results comparable. The study is focused on how good models are
predicting seasonal cycles. Figure \ref{fig:f49} represents the
simulated leaf area index (LAI) for five sites on deciduous forests. The
overall growing season was predicted too long by models (too early
spring and too late autumn), leading to an overestimation of GPP.

\begin{figure}

{\centering \includegraphics[width=0.8\linewidth]{figures/chap4/f47_LAI_richardson} 

}

\caption{Simulated and observed LAI for 5 deciduous forest sites and 14 vegetation models participating to the NACP model intercomparison project. (Richardson et al. 2012)}\label{fig:f49}
\end{figure}

Large biases are present for deciduous forests, evergreen forests are
better predicted. If we want to be able to perform better C-fluxes
simulations in the future, better understanding of phenology is
necessary.

\begin{figure}

{\centering \includegraphics[width=0.8\linewidth]{figures/chap4/f48_bias_richardson} 

}

\caption{Bias in modeled gross ecosystem photosynthesis (GEP=GPP) for deciduous broadleaf (top) and evergreen needleleaf (bottom) forests. Left panels show bias, by model. Right panels show the frequency distribution of these spring and autumn biases in re-scaled model GEP, across all models, sites, and years of data, for each forest type. The sign convention is that positive bias means that modeled GEP > tower GEP. (Richardson et al. 2012)}\label{fig:f410}
\end{figure}

The conclusion of this study is the models are not very good at
predicting the timing of seasonality.

\subsection{Case study 4.2}\label{case-study-4.2}

\textbf{Chen, X., Maignan, F., Viovy, N., Bastos, A., Goll, D., Wu, J.,
\ldots{} \& Ciais, P. (2020). Novel representation of leaf phenology
improves simulation of Amazonian evergreen forest photosynthesis in a
land surface model. Journal of Advances in Modeling Earth Systems,
12(1), e2018MS001565.}

The study of Chen et al. (2018) is an attempt to improve phenology
modelling in evergreen forests. This study uses a leaf age -- leaf
functioning relation and shows clearly the \textbf{links between
phenology and photosynthetic activity}.

\begin{figure}

{\centering \includegraphics[width=0.8\linewidth]{figures/chap4/f411_chen1} 

}

\caption{Observed and assumed relation between Vcmax and leaf age in the ORCHIDEE global model, for the tropical evergreen PFT. (Chen et al. 2018)}\label{fig:f411}
\end{figure}

Improved GPP simulation if phenology and leaf age is better represented.

\begin{figure}

{\centering \includegraphics[width=0.8\linewidth]{figures/chap4/f412_chen2} 

}

\caption{Comparison of litterfall data with two new and the old leaf turnover schemes in the ORCHIDEE model for 4 sites in the Amazon. (Chen et al. 2018)}\label{fig:f412}
\end{figure}

Improved GPP simulation if phenology and leaf age is explicitly
discussed.

\begin{figure}

{\centering \includegraphics[width=0.8\linewidth]{figures/chap4/f413_chen3} 

}

\caption{Comparison of GPP fluxtower data with two new and the old leaf turnover schemes in the ORCHIDEE model for 4 sites in the Amazon. (Chen et al. 2018)}\label{fig:f413}
\end{figure}

\part{Modelling vegetation
dynamics}\label{part-modelling-vegetation-dynamics}

\chapter{Modelling C-allocation and biogeochemical
cycles}\label{modelling-c-allocation-and-biogeochemical-cycles}

\chaptermark{Grotwh}

\section{Introduction}\label{introduction-1}

In Part I of the course we focused on modelling fluxes, but ecosystem
dynamics involve other aspects (biogeochemistry, demography,
biodiversity) that determine its functioning. The aspects, discussed in
Part II, are typically important at larger temporal and spatial scales
than the processes that determine fluxes. In this chapter we will focus
on the biogeochemical perspective, studying the full cycle of carbon or
nutrients, accounting for pools and fluxes. The biogeochemical
perspective is illustrated by the Figure \ref{fig:f51}. Bauters et al.
(2019) studied a landscape with forests of different age on the same
soil and with the same climate (a chronosequence). Figure \ref{fig:f51}b
shows the classical description of these forests in terms of stem
density, basal area and tree height. While Figure \ref{fig:f51}a shows
the situation from a biogeochemical perspective, focusing on aboveground
an below ground carbon pools (not accounting for the number of trees
etc.).

\begin{figure}

{\centering \includegraphics[width=0.8\linewidth]{figures/chap5/f51_bauters} 

}

\caption{Forest succession. (a) the biogeochemical perspective: variation of aboveground carbon (AGC) and soil organic carbon (SOC) along the forest succession in the Maringa-Lopori-Wamba landscape (DR Congo. Young: recently abandoned farmland, 5–25 years; SYoung: second growth young forest, 25–30 years; SOld: second growth old forest, approx. 150–300 years; Old-Mix: old growth mixed forest, 1700 years; Old-Gil: old growth Gilbertiodendron forest.BA stands for basal area. Error bars represent standard deviations. Percentages are stock values relative to the value of Old-Mix forest. The dashed horizontal line indicates the regional mean AGC of old growth forests across Central Africa (Lewis et al., 2013). (b) the forest structure perspective: Variation of four forest structure plot variables along the successional stages at our study site. Bars represent mean plot values per successional stage. Error bars represent standard deviations. BA stands for basal area. Error bars represent standard deviations. No standard deviations were calculated for Old-Mix because of its single plot replicate. Significant differences across forest types are indicated by different letters per type (p < 0.05)(Bauters et al. 2019)}\label{fig:f51}
\end{figure}

\section{Carbon cycle models: stocks and
fluxes}\label{carbon-cycle-models-stocks-and-fluxes}

In this section we refresh your knowledge on the carbon cycle and
translate that into a basic carbon cycle model. We typically study
carbon pools and fluxes (Figure \ref{fig:f52}). The fluxes are the flow
of carbon between pools, e.g litter fall, respiration,\ldots{} The gross
primary productivity (GPP) is the photosynthesis at the ecosystem level
and is the largest gross flux. The carbon taken up by photosynthesis is
allocated to different plant parts (roots, stems, leaves). This is a key
process, as the pool where the carbon is ending up will determine the
residence time and the potential long-term carbon storage in the system.
In addition to GPP and allocation, the carbon balance is controlled by
autotrophic and heterotrophic respiration and carbon turnover (liter
fall, mortality and disturbances). As we will discuss later in this
chapter, in parallel with carbon fluxes we can also study nutrient
fluxes between the different plant an soil pools.

\begin{figure}

{\centering \includegraphics[width=0.8\linewidth]{figures/chap5/f52_stocks_fluxes} 

}

\caption{Forest carbon cycle: stocks and fluxes.}\label{fig:f52}
\end{figure}

The carbon cycle of ecosystems is typically studied in terms of
productivity (Figure \ref{fig:f53}). The net primary productivity (NPP)
is the net production of the vegetation calculated as the difference
between the GPP and the autotrophic respiration. A positive NPP
corresponds to a net carbon uptake (growth) of the vegetation. If we
subtract from the NPP the heterotrophic respiration we get the net
ecosystem productivity (NEP), which is the full carbon balance of the
ecosystem (vegetation + soil). A positive value of NEP means carbon
uptake by the ecosystem from the atmosphere. Atmospheric scientists are
estimating the NEP by using fluxtowers. In that case the measure the net
ecosystem exchange (NEE) as estimate of NEP. Remark: for fluxtower
measurements an opposite sign convention is used, where a negative NEE
means a net carbon uptake, corresponding to a positive NEP.

When ``non-respiratory'' components are taken into account you get the
net biome productivity (NBP):

\[ NEP = GPP - R_a - R_h = NPP - R_h \]

\begin{figure}

{\centering \includegraphics[width=0.8\linewidth]{figures/chap5/f53_GPP_NBP} 

}

\caption{Definitions of productivity terms in the carbon cycle (Schulze et al. 2000)}\label{fig:f53}
\end{figure}

When considering all carbon stocks and fluxes in an ecosystem (Figure
\ref{fig:f52}), it is striking that usually only a few components are
(or can be) measured. In most monitoring plots only carbon stocks in
wood (inventories) and soil (soil sampling) are observed. In terms of
fluxes, woody productivity and litterfall are the most commonly
observed. In some cases NEE is observed by eddy covariance fluxtowers
and sometimes soil respiration is observed via chamber measurements. All
other stocks and fluxes are hard to measure (e.g.~root growth,
\ldots{}). In that respect vegetation models are very usefull to
estimate those stocks and fluxes that cannot be measured and to close
the full carbon cycle of the ecosystem.

Biogeochemical models for the carbon cycle are relatively simple models,
essentially doing the bookkeeping of the carbon stocks. These models
follow a few key principles:

\begin{itemize}
\item
  the net carbon input of the ecosystem is calculated by substracting
  the autotrophic respiration (\(R_a\)) from the GPP:
  \(NPP = GPP - R_a\)
\item
  Within the ecosystem, carbon flows (fluxes) between donor to receiver
  pools are depending on: Donor pool size, Chemical quality, and the
  Environment
\item
  Mass balance must be maintained (no carbon can be lost, except via
  respiration to the atmosphere)
\item
  Decay of litter and soil organic matter releases CO2 heterotrophic
  respiration (\(R_h\)).
\end{itemize}

These principles result in in a system of first order linear
differential equations (see CASA CNP model below). The above-mentioned
principles are common to any type of ecosystem, which has facilitated
the wide use of biogeochemical models in global vegetation models.

The GPP is calculated based on the various photosynthesis models that
were presented in chapter 2. Some models calculate \(R_a\) as a constant
fraction of GPP (example: \(R_a=0.5 \cdot GPP\)), but most
biogeochemical models consider the autotrophic respiration as the sum of
maintenance and growth respiration.

\[ R_a = R_m + R_g\]

The maintenance respiration (\(R_m\)) is the cost to support living
biomass (including the repair of membranes and proteins). \(R_m\) is
temperature dependent and therefore typically Q10 or Arrhenius functions
are used. In addition, \(R_m\) is proportional to the amount of biomass
of specific organs and the type of tissue. Leaves and fine roots are
fore example more costly, depending on their protein content.

The growth respiration (\(R_g\)) is the cost of synthesis of new
biomass, and is independent of temperature. Some models assume a fixed
fraction of NPP (\(R_g= 0.25 \cdot NPP\)). Growth has a double carbon
cost: the cost to of C to incorporate into the biomass and the energy
(respiration) needed to drive the biosynthetic reactions. If you know
the biochemical constitution of the plant tissue (leaf, wood, etc.), we
can calculate the cost of transforming basic sugars into the complex
molecules (e.g.~lignin) out of which the plant/tree is build. To
illustrate the functioning of biogeochemical model, we will study in
detail the CASA CNP model (Figure \ref{fig:f54}). This model has 9
pools: 3 plants pools, 3 litter pools and 3 soil organic matter pools.
The three litter pools are: structural litter (not yet decomposed),
metabolic litter (easy decomposable), CWD (coarse woody debris, dead
branches lying on the forest floor). The model essentially consists of 9
equations that calculate the change of each carbon pool over time, which
is calculated by summing up the incoming carbon and subtracting the
outging caron of each pool. The change of carbon content of each plant
pool is calculated according to the equations below, where the input is
determined by the NPP which is the carbon available for growth:

\[\frac{dc_1}{dt} = b_1 NPP - \xi_1 k_1 c_1 \]
\[\frac{dc_2}{dt} = b_2 NPP - k_2 c_2 \]

\[\frac{dc_3}{dt} = b_3 NPP - k_3 c_3 \]

Where \(b_i\) are allocations factors to leaves, roots and wood
respectively. These allocation factors may vary over time with phenology
and are specific for their pool, the CASA-CNP model uses constant
fractions.

\(k_i\) are turnover rates for plant and organic matter pools in litter
and soil. The example values in the figures are given for broadleaved
forest. These turnover rates can be modified for environmental
conditions, for examples for leaf turnover by the factor \(\xi_1\). If a
turnover rate (e.g.~for leaves) is higher than 1, it means that leaves
are replaced more than one time per year. The change of carbon content
of each litter and soil pool is calculated by the following equations:

\[\frac{dc_4}{dt} = a_{41} \xi_1 k_1 c_1 + a_{42} k_2 c_2 - \xi_4 k_4 c_4 \]

\[\frac{dc_7}{dt} = a_{74} \xi_4 k_4 c_4 + a_{75} \xi_5 k_5 c_5 + a_{76} \xi_6 k_6 c_6 - \xi_7 k_7 c_7 \]
Where the turnover rates \(k_i\) are modified for soil temperature and
moisture using the \(\xi_i\) factors and where the \(a\) factors depend
on lignin content and C/N ratio of the respective pools. The a factors
can be calculated as:

\[a_{ij} = (1 - r_{ij}) f_{ij} \] with:

\(f_{ij}\) = fraction that is allocated from j to i, and \(r_ij\) =
fraction lost as heterotrophic respiration \(R_h\) (represented by the
small arrows in Figure \ref{fig:f54}).

Biogeochemical models like CASA-CNP can work on various timesteps. Some
models simulate the allocation and biogeochemistry on a daily
calculation step, others use a monthly or annual time step.

The equations of the CASA-CNP model, and any similar biogeochemical
model can also be written in generalized matrix form, with PFT specific
parameters, to calculate the biogeochemical pools and fluxes in a rather
efficient way.

Other biogeochemical models work in a very similar way, but the
implementation can differ strongly (e.g.~the number of pools considered,
the environmental factors accounted for, etc.).

\begin{figure}

{\centering \includegraphics[width=0.6\linewidth]{figures/chap5/f54_casa_cnp} 

}

\caption{Structure of the nine-pool CASA-CNP biogeochemical model (Wang et al. 2010). Plant carbon mass consists of leaf c1, fine root c2, and wood c3. Net primary production is allocated to plant material in proportion to bi. Plant residue becomes metabolic litter c4, structural litter c5, or coarse woody debris c6. These pools decompose to fast c7, slow c8, and passive c9 soil organic matter. Pools differ in base turnover rate ki. Lines indicate carbon pathways, with aij the fraction of carbon turnover from pool j that enters pool i after heterotrophic respiration loss. Curved arrows denote heterotrophic respiration fluxes for each pathway. Shown are representative allocation and turnover rates for evergreen broadleaf forest. (Bonan)}\label{fig:f54}
\end{figure}

Figure \ref{fig:f55} illustrates the result of a simulation over 1000
years by the CASA-CNP model. It illustrated that long time scales are
needed to bring carbon pools at equilibrium. Plant pools reach there
maximum carbon stocks relatively quickly, while SOM pools needs more
time, illustrating that old forest keep on sequestering CO2 for
centuries in this simulation. The broadleaved evergreen forest simulated
here, needs more than 3000 years to reach equilibrium. The residence
time of carbon in the respective carbon pools is also very different.

\begin{figure}

{\centering \includegraphics[width=0.8\linewidth]{figures/chap5/f55_casa_result} 

}

\caption{Carbon dynamics of evergreen broadleaf forest using parameters described in (Bonan Table 17.3 and Table 17.4) with net primary production equal to 1000 g C m–2 y–1. The thick black line denotes soil organic matter at equilibrium.Definitions of productivity terms in the carbon cycle. (Bonan)}\label{fig:f55}
\end{figure}

The carbon turnover rate is the inverse of the residence time. Figure
\ref{fig:f56} illustrates that long term carbon storage depends a lot on
allocation towards pools with various turnover times (in the boxes in
Figure \ref{fig:f56}). Leaf level respiration fluxes result in very
short turnover times, while carbon allocated to wood has a much longer
turnover in the system.

\begin{figure}

{\centering \includegraphics[width=0.8\linewidth]{figures/chap5/f56_turnover} 

}

\caption{Turnover of carbon in terrestrial ecosystems (Schulze 2000).}\label{fig:f56}
\end{figure}

\section{Carbon allocation models}\label{carbon-allocation-models}

As indicated at the end of the previous section, carbon allocation is a
key process for the ecosystem carbon cycle. It is the process which
determines the carbon accumulation in various plant components (leaves,
wood, roots), and it affects the structural characteristics of the plant
(leaf area, rooting profile, plant height) which determine resource
acquisition (e.g.~higher leaf area results in higher photosynthesis of
the plant or ecosystem). However, the process-based knowledge on carbon
allocation is limited. It is uncertain how carbon allocation varies with
environmental factors. We know that plants allocate more to roots when
soils are dry or nutrient poor and we know that plants allocate more to
leaves when light is limiting, but the exact process are unknown or
unquantified. It is clear that plants are constantly adjusting their
allocation to prevent an overwhelming limitation by one resource, which
tends to make plants limited by multiple resources. Allocation is
usually studied/observed very roughly, by studying
aboveground/belowground biomass ratios or sapwood/leaf area ratio.

Due to our limited knowledge of the underlying processes, growth of
plants and the allocation processes can only be showed by means of a
schematic model (e.g.~Figure \ref{fig:f57}) because there exists no
fully physiological model that describes the distribution of assimilates
over the different organs. In the example of Figure \ref{fig:f57} also
the interaction with the nutrient cycle is accounted for.

\begin{figure}

{\centering \includegraphics[width=0.8\linewidth]{figures/chap5/f57_schulze_alloc} 

}

\caption{Schematic model of internal carbon, water and nutrient pools and fluxes within the plant (full lines) as well as important feedbacks and regulation (dashed lines). regylating factors are climate (a), stomata and impact of water balance on carbon allocation (b, c, d, e) and nutrient impacts (f) (Schulze and Chapin 1987).}\label{fig:f57}
\end{figure}

Current carbon allocation models make use of one of the following
approaches:

\begin{enumerate}
\def\labelenumi{\arabic{enumi}.}
\tightlist
\item
  Time-invariant partitioning coefficients (per PFT or climate type)
\item
  Empirical algorithms with variable allocation in response to
  environment
\item
  Functional relations affected by resource limitations (e.g.~balance in
  sapwood and leaf area) (scaling relationships)
\item
  Optimality approaches that balance multiple resource limitations
  (e.g.~optimize allocation to maximize NPP, or to balance water (root)
  and light (leaves) acquisition and water transport (stem))
\item
  More mechanistic approaches (for single plants) (e.g.~explicitly
  simulating phloem transport, etc\ldots{})
\end{enumerate}

The example of Figure \ref{fig:f58} illustrates the dependency of
allocation factors of the CTEM model to environmental factors (light
availability) for dry and wet soils.

\begin{figure}

{\centering \includegraphics[width=0.8\linewidth]{figures/chap5/f58_alloc_factors} 

}

\caption{Biomass allocation to leaf, root, and wood in relation to relative light availability f1 for (a) wet (f2 = 1) and (b) dry (f2 = 0.5) soil as in Arora and Boer (2005).}\label{fig:f58}
\end{figure}

We know that plants (trees) have a storage pool of (labile) carbon (NSC:
non-structural carbohydrates), which they can use to buffer variation in
resource supply. Many models account for such a storage pool, example to
allow for initial leaf growth in spring for deciduous vegetation. Such a
buffer allows plants to acquire most resources when they are abundant
and use accumulated carbon when needed for growth. Plants as such follow
`economic rules' similar to those of business firms. The actual
implementation of a storage pool differs from model to model (as
illustrated in Figure \ref{fig:f59}).

\begin{figure}

{\centering \includegraphics[width=0.8\linewidth]{figures/chap5/f59_storage_pool} 

}

\caption{Two different representations of plant growth. (a) Autotrophic respiration Ra is subtracted from gross primary production (GPP) and the remaining carbon is allocated to the growth of leaves, wood, and roots. (b) GPP first enters a storage pool from which maintenance respiration Rm is subtracted. The remaining carbon is allocated to growth after accounting for growth respiration Rg.(Bonan)}\label{fig:f59}
\end{figure}

It is important to mention here that the allocation process is tightly
linked with phenology (as discussed in chapter 4). Phenology will
determine how some allocation factors will vary over time, for example
determined by the timing of budburst and senescence.

\section{Case Study 5.1}\label{case-study-5.1}

In this case study the Thetys-Chloris model, with detailed carbon
allocation, was tested for a FACE experiment in Switzerland. A FACE
(free air CO2 enrichment) is an experimental setup on open air where and
ecosystem is exposed to elevated \(CO_2\) as global change experiment
(Figure \ref{fig:f510}). In the Swiss setup a tubing system in the tree
crowns is used to continuously spray CO2 in the experimental plot. The
site also has a `canopy-crane' used for direct in-situ measurement on
the leaves of the trees. Multiple supporting measurements are executed
(sap flow, stomatal conductance, litterfall, stem growth) for model-data
comparison.

\begin{figure}

{\centering \includegraphics[width=0.8\linewidth]{figures/chap5/f510_face} 

}

\caption{The Swiss Canopy Crane (SCC) free air CO2 enrichment (FACE) experiment conducted in a mature mixed deciduous forest near Hofstetten, 15 km south of Basel, Switzerland. Fatichi et al.2013. }\label{fig:f510}
\end{figure}

The study illustrate that the model works well to simulated carbon,
water and energy fluxes (Figure \ref{fig:f511}), however testing of the
model remains difficult for carbon allocation (Figure \ref{fig:f512}).

\begin{figure}

{\centering \includegraphics[width=0.8\linewidth]{figures/chap5/f511_fatichi_sf} 

}

\caption{A comparison between observed sapflow and simulated transpiration in relative units for the Swiss Crane forest stand in ambient (AMB) and elevated (ELE) CO2 conditions during the growing season of 2004 (subplots a and c) and 2005 (subplots b and d). Scatter plots of daily sums (subplots a and b) and time series for two representative five day periods (subplots c and d) are shown. Fatichi et al. 2013.}\label{fig:f511}
\end{figure}

\begin{figure}

{\centering \includegraphics[width=0.8\linewidth]{figures/chap5/f512_fatichi_growth} 

}

\caption{A comparison between observations [OBS. (1)] (solid lines) with uncertainty bounds (colored area) and model simulations (SIM., triangles) of stem growth (species averaged) per unit surface area for the period of 2000 through 2010 in ambient (AMB.) and elevated (ELE.) CO2 conditions. The stand carbon increments using only the Acrown derived for trees equipped with dendrometers in ambient and elevated conditions [OBS. (2)] in the upscaling are also reported (dashed lines). Fatichi et al. 2013.}\label{fig:f512}
\end{figure}

Figure \ref{fig:f513} illustrates the strength of this models that
allows to simulated the dynamics of all carbon pools at the species
level. It shows that beech is more drought-sensitive than oak at this
site, with a clear impact of the 2003 drought on the reserves and
sapwood pools for beech.

\begin{figure}

{\centering \includegraphics[width=0.8\linewidth]{figures/chap5/f513_fatichi_pools} 

}

\caption{The simulated dynamics of green aboveground, living sapwood, fine roots and carbohydrate reserves carbon pools of (a) Fagus sylvatica (Fs) and (b) Quercus petraea (Qp) under ambient (dashed lines) and elevated (solid lines) CO2 conditions for the period of 2001 through 2010.Fatichi et al. 2013}\label{fig:f513}
\end{figure}

It is important that in many detailed site level studies, such as the
one illustrated here, parameter optimalisation is applied to get an as
good as possible match between simulated results and measured data. Such
model tuning can be achieved in an automated way, for example by
algorithms that minimize the mismatch between model output and data, or
by manual tuning of specific parameters to achieve a better fit.

\section{Soil biogeochemistry models}\label{soil-biogeochemistry-models}

In this section we take a closer look to the soil component of
biogeochemical models. We are not only considering the carbon cycle
here, but also nutrients cycling (for which plants also play a key role
of course). It is important to be aware that carbon cycling
(decomposition) is a standard process in most vegetation models, while
the nitrogen cycle has only been included in vegetation models during
the last decade. The P cycle is currently only included in a handful of
models, will other nutrients (Mg, K, ..) are usually not accounted for
at all.

\subsection{Decomposition models}\label{decomposition-models}

Decomposition is the physical and chemical breakdown of dead organic
matter. It is an important process as it (1) provides energy for
microbial growth, (2) releases nutrients for plant uptake and (3)
influences ecosystem carbon storage (and therefore the global climate).
Decomposition basically consists of 3 processes: (1) leaching of soluble
materials by water, (2) fragmentation of organic matter by soil fauna
(which increases the surface for microbial attack), (3) chemical
alteration of the organic material. As such the decomposition process
changes the chemical composition of the detritus, breaks down the
organic matter to CO2 and nutrients and forms recalcitrant compounds.

Decomposition models assume that the decay rate of organic matter over
time is proportional to the amount of material in the litter pool and a
decay rate k.

\[\frac{dm}{dt} = -k m \]

When integrating this relation over time we can calculate the litter
mass in function of time following an exponential decline:

\[m(t) = m_0 e^{-kt} \] where the decay rate \(k\) depends on substrate
quantity and quality (Figure \ref{fig:f514}), the properties of the
microbial community and the physical environment. To account for the
physical environment, many models correct the decay rate for soil
temperature, pH and/or soil water content. Figure \ref{fig:f514bis}
gives an overview of the various factors controlling decomposition.

\begin{figure}

{\centering \includegraphics[width=0.8\linewidth]{figures/chap5/f514_leaf_decomp} 

}

\caption{Mass decline over time during decomposition of different plant materials.}\label{fig:f514}
\end{figure}

\begin{figure}

{\centering \includegraphics[width=0.8\linewidth]{figures/chap5/f514_chapin_decomp_controls} 

}

\caption{Controling factors on the decomposition process. The thickness of the arrows are proportional to the level of control. (CHapin).}\label{fig:f514bis}
\end{figure}

Some models use specific decay rates for each type of SOC depending on
its composition (C/N ratio, lignin content (Figure
\ref{fig:f516}),\ldots{}). This results in models with multiple SOC and
litter pools with specific decay rates (as illustrated earlier with the
CASA-CNP model). A decomposition model with three pools will then have
the following form:

\[m(t) = m_1 e^{-k_1t} + m_2 e^{-k_2t} + m_3 e^{-k_3t} \]

Based on such a model, one can simulate the different phases of
decomposition depending on the composition of the remaining litter
(Figure \ref{fig:f515}). In phase 1 the leaching dominates. In phase two
we still have a relatively high value of \(k\), because the labile
substrates are broken down. In phase 3 a low value of \(k\), reflects
the recalcitrant substrates that predominate. Figure \ref{fig:f515} also
illustrates the fact that the time scale of decomposition heavily
depends on the environment, with a much longer time scale in the arctic
compared to the tropics where conditions are much more favorable (warm
and humid) for decomposition.

\begin{figure}

{\centering \includegraphics[width=0.8\linewidth]{figures/chap5/f515_decomp_phases} 

}

\caption{Phases of decomposition over time depending on the constitution of the remaing litter.}\label{fig:f515}
\end{figure}

\begin{figure}

{\centering \includegraphics[width=0.8\linewidth]{figures/chap5/f516_decay_lignin} 

}

\caption{Relation between the decomposition constant and the lignin:nitrogen ration.}\label{fig:f516}
\end{figure}

Some recent vegetation models use a `litter cohort model' in their soil
biogeochemical component (Figure \ref{fig:f517}). These models track
each litter type at different stages of decomposition (litter cohorts).
Such models can assign a specific decay rate to each litter cohort and
can simulate transfer of organic material between cohorts. Within a
cohort the C/N ration can change over time, impact the decay rate.

\begin{figure}

{\centering \includegraphics[width=0.8\linewidth]{figures/chap5/f517_litter_cohort} 

}

\caption{Decomposition as represented by a litter cohort model. Foliage, twig, root, and wood litter form individual cohorts with an initial carbon, nitrogen, and lignin mass. Each box represents an individual cohort for a particular year. Foliage litter can vary in initial quality, represented by multiple litter cohorts. The cohorts decompose over time, immobilize nitrogen, and transfer to humus upon reaching a critical nitrogen concentration. Fresh wood first passes through a well-decayed wood pool (WDW) before becoming humus. Humus decomposes and mineralizes nitrogen.(Bonan)}\label{fig:f517}
\end{figure}

Figure \ref{fig:f518} illustrates the structure of the DAYCENT model,
which is a recent full biogeochemical soil model (based on the original
CENTURY model, which is reference in the field). It shows the complex
structure of this very detailed model, which considers cohorts in
different stages of decay with different k values, tracking lignin
contents, accounting for the impact of temperature, soil water pH,
aerobic conditions, texture, et cetera. It is important to mention here
that most of the global vegetation models do not have such a detailed
biogeochemical soil component. Current developments in soil
biogeochemical models are focusing on implementing multiple soil layers,
the specific conditions of permafrost soils and impact of the
composition of the soil microbial community.

\begin{figure}

{\centering \includegraphics[width=0.8\linewidth]{figures/chap5/f518_daycent} 

}

\caption{Example of a full soil biogeochemical model DAYCENT. Litter, soil organic matter, and coarse woody debris pools and fluxes represented. The model has leaf, fine root, and three coarse woody debris litter flux inputs (u1–u5) and twelve carbon pools (c1–c12). Shown are litter flux partitioning parameters bij, base decomposition rates kii (per year), fractional carbon transfer fij, and respiration fraction rij. Solid lines indicate decomposition pathways, with curved arrows denoting heterotrophic respiration fluxes for each pathway. DAYCENT allows for photodegradation from solar radiation, but that was not included in Bonan et al. (2013). Nor was leaching loss included. (a) Leaves decompose as surface material represented by two litter pools and two organic matter pools (shown on the left). Fine roots decompose as belowground material represented by two litter pools and three organic matter pools (shown on the right). The actual decomposition rate varies with soil temperature T, soil moisture θ, and pH. Belowground decomposition additionally varies with anaerobic conditions (O2), cultivation, and soil texture. Structural litter decomposition also depends on lignin fraction flig. The total turnover of the surface slow pool depends on decomposition and mixing, with a fraction to the belowground slow pool and the remainder to the surface active pool. The C/N of organic matter differs among pools and varies with soil mineral nitrogen. Shown is the minimum and maximum value for each pool and (in parentheses) the soil mineral nitrogen (g N m–2) for the minimum C/N. (b) Coarse woody debris decomposes to the active and slow pools. Fine-branch and large-wood debris flows to surface pools, and coarse root flows to belowground pools.(Bonan)}\label{fig:f518}
\end{figure}

\subsection{Nutrient cycle models}\label{nutrient-cycle-models}

We will only discuss nutrient cycles (and the associated models)
superficially here. Many other (soil) courses in the program focus on
nutrient cycles. Implementation of nutrient cycles in vegetation/soil
models is pretty similar as the carbon biogeochemistry models discussed
above. When there is a carbon model into place, a nitrogen model can be
implemented by tracking for each carbon pool or flux an associated N
pool or flux. However, the soil nitrogen cycle is more complex than the
soil carbon cycle (Figure \ref{fig:f519}) because we need to account
for:

\begin{itemize}
\tightlist
\item
  Reabsorption of N from litter fall and reallocation of N from leaves
  during senescence
\item
  N has gas forms: N2, NH3, N2O, NO (losses)
\item
  Plants use inorganic ions nitrate (\(NO_3^{-}\) and ammonium
  (\(NH_4^{+}\))
\item
  Multiple input processes: biological fixation, atmospheric deposition,
  fertilizer
\item
  Specific role of micro-organisms
\end{itemize}

\begin{figure}

{\centering \includegraphics[width=0.8\linewidth]{figures/chap5/f519_N_cycle} 

}

\caption{Depiction of the nitrogen cycle. Circles indicate various pools (solid lines) or gaseous losses (dashed lines). Boxes denote processes. Also shown are natural inputs from biological nitrogen fixation and anthropogenic inputs from nitrogen deposition, fertilizer, and manure.(Bonan).}\label{fig:f519}
\end{figure}

Research in the past decade has shown the important impact of the
nitrogen cycle on carbon cycle simulations (e.g.~a decrease of NPP when
soil N is limiting). Figure \ref{fig:f520} illustrates that recent earth
system models (with and without accounting for N) simulated a global
carbon sink which is smaller when accounting for nitrogen limitation.
This shows that up until last IPCC report (where models did not account
for the N cycle) future potential carbon stocks have been over
estimated. So, the importance of implementing the N cycle in global
models in widely accepted. However, a variety of approaches exists,
where models differ largely in the way they represent: - N impact on
photosynthesis - N uptake processes - Constant or variant C:N ratios -
Competition for soil mineral N by plants and microbial community

\begin{figure}

{\centering \includegraphics[width=0.8\linewidth]{figures/chap5/f520_zhaele} 

}

\caption{Cumulative C sequestration from the CMIP5 models and plausible range of C sequestration considering N constraints (CMIP5-N) for the HadGEM2-ES and IPSL-CM5A-LR models. Zhaele et al. 2015}\label{fig:f520}
\end{figure}

\section{Case study 5.2}\label{case-study-5.2}

In the second case study of this chapter, we take a closer look to one
of the first global models (JSBACH) that accounted for the P cycle. The
study presents how the P cycle was implemented and tested the impact on
the global land carbon cycle. Figure \ref{fig:f521} illustrates how the
N and P cycle are implemented in JSBACH and how these nutrient cycles
are interacting with the C cycle. The model allowed for global simulated
maps of phosphorus in vegetation and soil, which illustrate the low P
content in tropical areas.

\begin{figure}

{\centering \includegraphics[width=0.8\linewidth]{figures/chap5/f521_goll_jsbach} 

}

\caption{Schematic representation of pools and fluxes in JSBACH. Solid arrows indicate carbon fluxes and dashed arrows nutrient fluxes. The plant compartment consists of the three C pools: active (leaves and non-lignified tissue), wood (stem and branches) and reserve (sugar and starches). The litter compartment consists of non-lignified litter, and woody litter (lignified litter and fast-decomposing soil organic matter).The soil compartment consists of one pool (slow) representing slow-decomposing organic matter. All carbon pools except the reserve pool have a corresponding nutrient pool with variable C :N: P ratio (slow, non-lignified litter) or fixed C :N: P ratio (rest). There is one mobile plant pool representing mobile nutrients stored internally in plants. Soil mineral nitrogen is represented by a single pool (soil mineral pool), while mineral P is represented by labile (available) pool and sorbed pool. The opposing triangles indicated that the flux is controlled by phosphorus (red triangles), nitrogen (blue triangles) or both (mixed triangles).Goll et al. 2012}\label{fig:f521}
\end{figure}

\begin{figure}

{\centering \includegraphics[width=0.8\linewidth]{figures/chap5/f522_goll_Pmaps} 

}

\caption{The simulated P in vegetation (top left), sorbed P (top right), P in soil organic matter and litter (bottom left), and the annual P uptake by vegetation (bottom right) for present day as simulated by JSBACH. Goll et al. 2012.}\label{fig:f522}
\end{figure}

The study compared simulation of the accumulated land carbon uptake
between 1860 and 2100 with and without the nutrient cycles activated in
the model. The results show a simulated global land carbon sink which
was 13\% lower when the N cycle was accounted for (mainly contributed by
effects in the high latitudes). When the P cycle was accounted for the
sink was even 16\% lower (mainly contributed by tropical ecosystems).
The combined N and P cycle resulted in a 25\% lower land sink (Figures
\ref{fig:f523} and \ref{fig:f524}).

\begin{figure}

{\centering \includegraphics[width=0.8\linewidth]{figures/chap5/f523_goll_sink} 

}

\caption{The simulated change in land carbon storage under the SRES A1B scenario. Shown are the 10-yr mean of soil temperature (a), the CO2 concentration as used in the forcing simulation (b), the resulting change in total land C storage (c), and the changes in the two main land compartments vegetation (d) and soil (e). Goll et al. 2012.}\label{fig:f523}
\end{figure}

\begin{figure}

{\centering \includegraphics[width=0.8\linewidth]{figures/chap5/f524_gool_sink_map} 

}

\caption{The reduction in C storage (kgm−2) by nutrient limitation at the end of the 21st century. Shown is the difference in the mean C storage (2070–2099) between the CN simulation and the C-only simulation (upper panel), and between the CP simulation and the C-only simulation (lower panel). The latitudinal means over land points are shown on the right side.Goll et al. 2012.}\label{fig:f524}
\end{figure}

\section{Water balance models}\label{water-balance-models}

We will not focus on water balance models in this course. As these are
discussed in detail in hydrology courses. However, it is important to
mention the key role of water balance models here, which are usually an
integral part of vegetation models. The water balance is essential to
simulate the energy balance of the system via evapotranspiration (latent
heat flux). Moreover, reliable soil water balance models (see soil
physics course) are essential to simulate water availability. It is also
important to mention that the C allocation model will influence the
water cycle. More allocation to leaves will result in high transpiration
and a higher rainfall interception.

\chapter{Modelling Vegetation Dynamics and
Demography}\label{modelling-vegetation-dynamics-and-demography}

\chaptermark{Biodiversity}

\section{Introduction}\label{introduction-2}

\textbf{Biogeography} is the study of the distribution of species and
ecosystems in geographic space and through geological time. A group of
models that studies the distribution of species geographically is called
\textbf{``species distribution models''} (SDM) (e.g.~Fig.
\ref{fig:f61}). These models are typically empirical models, as they
link species distribution to other parameters, e.g.~climate and soil
parameters, by using statistical approaches. We just mention these
models briefly here (as they are not the focus of this course, but they
are still important as biogeographical model outputs are used as an
input to the vegetation models, to inform the vegetation model where a
species occurs or can occur. SDMs link the current distribution of a
species, to a range of environmental properties, typically climate and
soil properties, by using statistical approaches. The models typically
use climate envelopes for this, where a specific species occurs within a
range of a certain parameter. These data can then be used to predict
where a species will occur in the future, or where it occurred in the
past, neglecting the possible barriers of species migration. A second
group is a more mechanistic type of model that uses plant physiology and
plant traits to predict the distribution of a species. These models
should be able to account for climate adaptation and acclimation.

\begin{figure}

{\centering \includegraphics[width=0.8\linewidth]{figures/chap6/f61_svenning_SDM} 

}

\caption{Example of a species distribution model (SDM), used to simulate current and past species distribution (Svenning et al. 2011). LGM: Last Glacial Maximum. Such models can also be used to predict future species distributions.}\label{fig:f61}
\end{figure}

The study of \textbf{ecosystem demography} tries to understand and
quantify vegetation dynamics. Temporal dynamics in vegetation, community
composition and ecosystem structure are typically driven by demographic
processes: recruitment and establishment, growth, and mortality (Fig.
\ref{fig:f62}).

\begin{figure}

{\centering \includegraphics[width=0.8\linewidth]{figures/chap6/f62_demography} 

}

\caption{Illustration of succesional forest dynamics.}\label{fig:f62}
\end{figure}

\section{Gap models, area and cohort based
models}\label{gap-models-area-and-cohort-based-models}

From the perspective of vegetation dynamics we can consider three types
of vegetation models (Fig. \ref{fig:f63}). \textbf{Gap models} were
developed in parallel with biogeochemical models. They simulate
demographic processes for each individual tree. The landscape consists
of different patches, which can be in different stages of succession.
The model output is typically an output per tree. The second group of
models are the \textbf{area-based models}, most dynamic global
vegetation models (DVGM) belong to this model type. This group is on the
other side of the spectrum, as this model does not calculate values for
individual plants, or the number of individuals, but it simulates
allocation at ecosystem scales. In these models, the landscape is
divided in different patches, but every patch is only occupied by one
PFT. Every patch is then represented by an average tree, after which the
output is calculated at landscape model. The third type of model falls
in the middle of the spectrum between two previous model types.
\textbf{Cohort based models} do not track each individual tree in the
landscape, but group trees from similar size and function in cohorts.
Each cohort is represented by an average tree, for which the
calculations are made. Contrary to the area models, the number of
individuals is tracked. Gap models/individual based models and
cohort-based models are both called \textbf{demographic models}, as they
are both able to model demography and the account for demographic
processes.

What are plant functional types (PFT)? They represent broad groupings of
plant species that share similar characteristics (e.g.~growth form) and
roles (e.g.~photosynthetic pathway) in ecosystem function. However, PFTs
have a different meaning in demographic models and area-based models. In
the latter, they represent an ecosystem functional type, whereas in
cohort/individual based models they represent a specific plant type.

\begin{figure}

{\centering \includegraphics[width=0.8\linewidth]{figures/chap6/f63_3types} 

}

\caption{Representation of vegetation patches in models. (a) Forest gap models are individual-based models and represent a landscape as patches that differ in stage of post-disturbance development. Shown are two patches, each comprised of multiple trees that differ in size and species. (b) Area-based dynamic global vegetation models (DGVMs) represent vegetation as discrete patches of plant functional types (PFTs) that differ in area. In this example, the model grid cell has six plant functional types (distinguished by canopy shape and shading) that differ in biomass (size of plant) and patch area (size of subgrid tile). (c) Ecosystem demography models define patches based on time elapsed since disturbance. Multiple plant functional types can exist within a patch and are represented as cohorts defined by plant type and size. Shown are six patches with different subgrid area and cohort assemblages. (Bonan)}\label{fig:f63}
\end{figure}

\subsection{Area based models}\label{area-based-models}

Area based models typically focus on biogeochemical processes. In
area-based models, a region is subdivided in pixels, and every pixel is
occupied by one or multiple plant functional types (PFT). If more than
one PFT is present, the pixels are subdivided in different patches,
where the area of the patch relates to the abundance of the PFT. Plant
functional types are not mixed, e.g.~a savanna pixel is represented by
two patches, a grass patch and a forest patch. The size of the patches
can change time if the conditions are more in favor of a specific PFT.
The calculation of the surface of the patches is based on simplistic
rules, based on productivity, mortality, establishment, fire, bioclimate
tolerance,\ldots{} Example: if the upper soil dries, grasses die, but
trees have access to deeper water so they survive and the size of the
tree-patch will increase. In this perspective, area-based models are
also dynamic models.

Within each pixel (or grid-cell), the same homogeneous climatic
conditions apply. PFTs are determined based on growth form, leaf
longevity, leaf type, photosynthetic pathway and bioclimate in
area-based models. This typically results in 12 up to 14 functional
types to cover the entire world. As only one average plant is simulated,
the light availability within the PFT (patch) is homogenous, and no
canopy dynamics are included. As the model is deterministic, depending
on the climatic parameters you set, a certain area will be occupied by a
certain PFT. Co-existence (e.g.~two competing trees in a mixed forest
living together) is very difficult to simulate with such models, because
if the parameter of one PFT is a bit more advantageous, this PFT will
eventually expel the other and dominate the model.

Despite the above described strong assumptions, these models are
nowadays mostly used in the current earth system models, because they
are computational efficient, and they are good in simulating broad scale
landscape patterns. Examples of vegetation models that are using an
area-based approach are ORCHIDEE (French model), JSBACH (German model),
JULES (UK model). Sometimes the models are used with prescribed
vegetation, e.g.~a vegetation map, and are then executed without the
dynamic vegetation. This is mostly done when simulations are run over a
shorter time period, e.g.~20 years.

The LPJ-model is a very widely used model and the model occurs in many
of the examples in this syllabus. LPJ-GUESS is the demographic variant
of the LPJ-model. It is a particular type area-based model because it
simulates the average plant, but it also simulates the number of plants
in each patch. The overall scheme of the LPJ model is shown below (Fig.
\ref{fig:f64}). The model contains the biogeophysics (see first
lectures), biogeochemistry (e.g.~Carbon pools), and had vegetation
dynamics, which is calculated at a yearly basis. It has a background
mortality of 1\% and mortality caused by negative NPP, heat stress,
fire, \ldots{}

\begin{figure}

{\centering \includegraphics[width=0.8\linewidth]{figures/chap6/f64_LPJ_fllowchart} 

}

\caption{Coupling of a dynamic global vegetation model with a land surface model. Shown are linkages among the biogeophysics, biogeochemistry, and vegetation dynamics components of the model. The lightly shaded biogeophysical processes represent the traditional hydrometeorological scope of land surface models. The darker boxes represent the greening of land surface models with the introduction of dynamic vegetation and the carbon cycle. (Bonan)}\label{fig:f64}
\end{figure}

The figure below (Fig. \ref{fig:f65}) shows some dynamic simulation
results of the original LPJ-model. The color of the grid cells shows the
most dominant vegetation type simulated in that cell. The resulting
world map of potential vegetation is very realistic.

\begin{figure}

{\centering \includegraphics[width=0.8\linewidth]{figures/chap6/f65_LPJ_PFTmap} 

}

\caption{Map of LPJ simulated dominant PFTs (Sitch et al. 2003)}\label{fig:f65}
\end{figure}

In Figure \ref{fig:f66} LPJ simulations are compared with satellite
data. The comparison shows that the model does a reasonable job, but for
some regions, the model simulations are not good, e.g.~Spain: there are
no dominating evergreen species, while the model predicts this. Also,
tundra vegetation is not simulated well.

\begin{figure}

{\centering \includegraphics[width=0.8\linewidth]{figures/chap6/f66_LPJ_comparison_sitch} 

}

\caption{Comparison of LPJ-simulated distributions of woody vegetation with satellite-based maps, for percentages of tree cover, partitioned according to phenology (evergreen vs. deciduous), and leaf morphology (broadleaf vs. needleleaf). (Sitch et al. 2003)}\label{fig:f66}
\end{figure}

These models are also able to simulate successional vegetation dynamics:
see figure \ref{fig:f67}. Within one grid cell, the relative cover of
different pft is calculated, and in this way, the forest succession is
simulated. The start is from bare ground, and the model then dynamically
simulates that it is first covered by grasses, which are gradually
replaced by trees, with first a peak in broadleaved trees, which are
eventually replaced by needleleaf trees in this example. The simulated
competition is a kind of `prescribed' as each PFT covers a separate
landscape patch. Problems with area-based models: They cannot be linked
to real inventory observations because the model does not simulate a
number if individuals or a size of individuals. They also neglect
important aspects of the successional dynamics: for example, height,
size distribution and species diversity are not simulated, but they are
very important as they affect the dynamics and habitats of forests. The
model is also not capable of simulating disturbances. For example, if
you want to take thinning into account, you can only remove a certain
amount of biomass, not a specific number of trees. Therefore, we need
other models to simulate demographic processes.

\begin{figure}

{\centering \includegraphics[width=0.8\linewidth]{figures/chap6/f67_DGM_boreal_succession} 

}

\caption{Boreal forest dynamics in terms of (a) percentage cover and (b) plant carbon pools as simulated by a dynamic global vegetation model. The simulation is from initially bare ground for a single model grid cell in the boreal forest over 1000 years in the absence of fire. Percentage cover is the annual extent of plant functional types in the grid cell.(Bonan)}\label{fig:f67}
\end{figure}

The Triffid model: is the predecessor of JULES mode, competition is
based on lotka-volterra competition model, a general equation originally
developed to predict predator prey interactions. A well know study (Cox
et al. 2004) predicted the amazon dieback by using this area-based
model, coupled to a climate model. It predicted a drying of the Amazon
forest, and following the drying pattern, the forest would be replaced
by grasses, creating a savanna (Fig \ref{fig:f68}). However in follow-up
studies, it turned out that the Amazon forest is more resilient than in
the original model simulations by Triffid, if you consider that patches
are interacting with each other, which was not simulated in this model
(see examples later in this syllabus).

\begin{figure}

{\centering \includegraphics[width=0.8\linewidth]{figures/chap6/f68_amazon_dieback} 

}

\caption{Evolution of the vegetation cover in the Amazon from a coupled climate-carbon cycle simulation with the TRIFFID DGVM and the HadCM3LC climate models. (Cox et al. 2004)}\label{fig:f68}
\end{figure}

\subsection{Gap models}\label{gap-models}

Gap models (or IBMs individual based models) were developed in parallel
with area-based models in the 1970s. However, both model types were
developed from a different perspective. Area based models are driven by
an interest to simulate global carbon cycle, and global energy balance
and water cycles, and vegetation dynamics were only a secondary element
of these models. Gap models are originally developed by foresters. They
were interested in the dynamics of forests as such, in individual trees,
but not in the energy balance or the carbon cycle. The models are based
on studies of gap dynamics, and therefore, they have patches of circa
100-800 m², which is the size of the gap created by one big tree that
dies.

Within a gap, trees are competing for resources, and growth, mortality
and regeneration are simulated (Fig \ref{fig:f69} and Fig
\ref{fig:f610}) . The main mechanism that drives the dynamics in gap
models is the competition for light between individuals. Growth,
diameter, age and height are simulated in response to environmental
variables. Mortality is simulated in response to environmental
constraints such as stress (climate), fire and insects. Regeneration is
calculated based on seed availability, sprouting, which in turn depend
on environmental constraints. All these variables are calculated for
every individual, typically at annual timesteps.

\begin{figure}

{\centering \includegraphics[width=0.8\linewidth]{figures/chap6/f69_gap_dynamics} 

}

\caption{Cyclic growth and thinning of trees in a forest patch during gap dynamics.(Bonan)}\label{fig:f69}
\end{figure}

\begin{figure}

{\centering \includegraphics[width=0.8\linewidth]{figures/chap6/f610_gap_model} 

}

\caption{Depiction of a boreal forest gap model. The growth of an individual tree depends on its diameter, age, and height as modified by environmental constraints. Mortality depends on the age of the tree as modified by stress, wildfire, and insects. Regeneration depends on seed availability, the ability to sprout or layer, and site conditions. (Bonan)}\label{fig:f610}
\end{figure}

By simulating demographic processes for each tree, the model can track
stem density over time, also the size distribution and the species
composition. Gap models are stochastic: as each simulated patch is
different and can evolve in a different way, depending on stochastic
elements. However, all patches combined results in a landscape with
stable properties.

In these models, biomass is not in the center, but can be calculated as
the model simulates the growth of all individual trees. This also
applies to the community composition and the biogeochemical cycles.
These are all emergent outcomes, as they are not described explicitly in
the model, but emerge from its output.

In gap models, landscapes are simulated by different patches, in which
each individual is modelled by the model. There are different
possibilities to put the patches together in the landscape (Fig
\ref{fig:f611}): first, a monte Carlo simulation is run to select random
places, and for theses places, the growth is simulated, after which it
is upscaled (averaged out) to the landscape. A second possibility is to
use a grid of patches and average out, while a third method simulates
all processes for adjacent plots. Finally, you can start a simulation
from existing inventories. If you have inventories over different parts
of the landscape, you can average simulations of these inventories to
come to a landscape average.

Gap models are computationally very intensive, as they simulate each
individual tree, which makes it problematic to implement them in earth
system models. They also don't simulate short time scale leaf
physiology, but simulate growth based on specific growth equations, that
describe growth directly as a response to the available light or other
factors.

\begin{figure}

{\centering \includegraphics[width=0.8\linewidth]{figures/chap6/f611_gap_upscaling_shurgart} 

}

\caption{General functioning of a gapmodel. As one moves to the right to left, spatial scale increases from an individual tree to a small plot to a landscape. The tree-level response shown here is the elementary growth equation from the FORET (Shugart and West 1977) model. The magnitude of the tree-mortality probablity of each tree is also determined at the tree-level depending on tree growth as an index of vigor, species longevities and other conditions. The form of the growth equation with no constraints is shown at the top and the decrement to this optimal growth equation is found below according to the particular controlling environmental factors (available light, soil moisture, etc). At the plot level, the vertical profile of light, available soil moisture, and other environmental and biogeochemical factors are calculated and tree to tree interactions are computed. Conditions for potential new seedlings for each year are determined factors such as the environmental conditions and seed sources. At the landscape model, a basic gap model can be used to represent the landscape as: (a) the summation of a Monte Carlo collection of independent random points; (b) gridded points at some spacing, (c) a tessellation of adjacent plots; (d) a spatially explicit landscape simulation with a spatial map of trees that is ‘windowed’ or updated for tree birth, growth and death by dropping a gap-model computational window onto the tree-stem map to solve for a subset of a new map. This is repeated to produce the new map. The size of this subset determines the resolution of the spatial map. (Shugart et al. 2018)}\label{fig:f611}
\end{figure}

Examples: ForClim model Fig \ref{fig:f612} shows simulations for forests
in the alps, where different species are competing. The model is
initiated from bared ground, and also includes some disturbances with a
stochastic character: e.g.~wind or management impact. The figure shown
is a typical figure for this kind of models, where calculations are made
on an annual basis, and simulations are run for centuries or even
millennia. In this graph, the abundance of a species is expressed as its
total biomass.

\begin{figure}

{\centering \includegraphics[width=0.8\linewidth]{figures/chap6/f612_forclim_succession} 

}

\caption{Comparison of the behavior of two versions of the gap model ForClim under current climate (years 0–800) and under conditions of climatic change (years 800–900) and under a hypothetical future constant climate (years 900–1500) at two sites in the European Alps, Bever (a, b) and Davos (c,d). Model version 2.4 (a, c) has a parabolic temperature response function for tree growth (like JABOWA model), whereas model version 2.9 (b, d) features an asymptotic response function. (Bugmann 2001)}\label{fig:f612}
\end{figure}

Gap models are mostly applied locally but can sometimes be upscaled to
larger areas (Fig \ref{fig:f613} and Fig \ref{fig:f614}). However, for
such large scale applications they don't simulate every single tree over
the large area but simulate representative areas and extrapolate that
data over the entire study area. Sometimes a model data fusion approach
is followed, and the data is matched with satellite data, to have a more
realistic representation of reality (as in example Fig \ref{fig:f613}).
In this example for the Amazon they used satellite data to parameterize
the FORMIND model. The model gives a very detailed biomass map at a very
high resolution. This is only possible with this kind of model, with an
area-based model for example, you would have very homogenous biomass
patterns over the amazon.

\begin{figure}

{\centering \includegraphics[width=0.8\linewidth]{figures/chap6/f613_formind_amazon} 

}

\caption{High-resolution biomass map for the Amazon rainforest at 1000 m resolution (left) and relative frequency distributions at 40 m resolution (right) derived by linking FORMIND (IBM) simulation results with remote sensing data. (Rödig et al. 2017)}\label{fig:f613}
\end{figure}

In the example of Fig \ref{fig:f614} projections for the change in
carbon biomass in Russia are shown over the period from now to the end
of the century, by using two different climate scenarios. Red areas
indicate a large decrease in carbon biomass, green areas indicate a
growing carbon biomass. For example, northern Russia is now too cold for
tree growth, but could be warmer towards the end of the century,
allowing tree growth and leading to a higher carbon biomass.

\begin{figure}

{\centering \includegraphics[width=0.8\linewidth]{figures/chap6/f614_uvafme_russia} 

}

\caption{Difference in UVAFME simulated total carbon biomass (tonnes of carbon per hectare (t C·ha–1) for a mature forest between year 2000 and year 2095 following 95 years of temperature and precipitation as represented by NCAR CCSM A1B and A2 scenarios.(Shuman et al 2015)}\label{fig:f614}
\end{figure}

\subsection{Cohort based models}\label{cohort-based-models}

Cohort based models (also called `ecosystem demography models') are an
approximation of IBM's. They group all similar plants into cohorts and
simulated one average plant for each cohort, enormously reducing the
computational demand. They bridge the gap between IBM and area-based
models. They do simulate very short time-scale processes such as
photosynthesis and simulate demographic processes. They are not
stochastic, i.e.~they don't simulate the mortality of each individual
tree but have a more deterministic description of mortality. For
example, every year, a fraction of x\% of the trees is removed. The
cohort approach retains the dynamics of IBMs, with reduced computational
cost, but removes stochastic processes that can enhance the
representation of functional diversity. Just like in IBMs, the model has
emergent outcomes for community composition, stand biomass and
productivity and biogeochemical cycles, as these are not explicitly
simulated by the model.

We will focus on one example: the ED model (ecosystem demography model).
In ED the landscape is divided into multiple grid cells (Fig
\ref{fig:f615}). In each grid cell, the same climatological conditions
apply. Within each grid cell, multiple sites exist, based on abiotic
conditions, such as soil texture. The sites and grid cells belong to the
static part of the model, these characteristics don't change over time.
Each site has two dynamic levels, i.e.~patches and cohorts. A patch
represents a fraction of the landscape grouping all pieces of land that
have the same disturbance history. The size of the patches and the
number of patches is dynamic, and changes during the simulation. E.g. if
there is a lot of mortality, a large patch of young, new forest will be
created. Or if there is almost no mortality, the size of the patch of
old growth forest will increase. Each patch is covered by a vegetation
community, built by different cohorts. A cohort represents a number of
individual trees belonging to the same size class and the same plant
functional type. Cohorts are also dynamic and change over time (number
of plants can increase or decrease), and cohorts can merge (e.g.~a
cohort of young beech trees that grew enough can become part of the
cohort of old large beech trees).

\begin{figure}

{\centering \includegraphics[width=0.8\linewidth]{figures/chap6/f615_ED_structure} 

}

\caption{Schematic representation of the multiple hierarchical levels in the ED-2.2 model, organized by increasing level of detail from top to bottom. Static levels (grid, polygons, and sites) are assigned during the model initialization and remain constant throughout the simulation. Dynamic levels (patches and cohorts) may change during the simulation according to the dynamics of the ecosystem.(Longo et al 2019)}\label{fig:f615}
\end{figure}

The ED2.2 model is an advance version of ED that has the full biophysics
and biogeochemistry cycle and energy balance included (Fig
\ref{fig:f616}, Fig \ref{fig:f617}). For each cohort, all these
variables and processes are calculated, and the different cohorts are
competing within the patch. This implies that it will be more difficult
for a young tree to develop in this model, a struggle for life which is
not represented in an area-based model, as here, the tree stands
underneath another cohort, which reduces the light and nutrients
available for the young tree. The advantage of this model compared to
gap models and area-based models is that we can simulate fast time
processes and vegetation dynamics with one model.

\begin{figure}

{\centering \includegraphics[width=0.8\linewidth]{figures/chap6/f616_ED_biophysics} 

}

\caption{ED2.2 biophysics. Schematic of the fluxes that are solved in ED-2.2 for a single patch (thermodynamic envelope). In this example, the patch has NT cohorts, NG soil layers, and NS=1 temporary surface water. Both NG and the maximum NS are specified by the user; NT is dynamically defined by ED-2.2. Letters near the arrows are the subscripts associated with fluxes, although the flux variables have been omitted here for clarity. Solid red arrows represent heat flux with no exchange of mass, and dashed yellow arrows represent exchange of mass and associated enthalpy. Arrows that point to a single direction represent fluxes that can only go in one (non-negative) direction, and arrows pointing to both directions represent fluxes that can be positive, negative, or zero.(Longo et al 2019)}\label{fig:f616}
\end{figure}

\begin{figure}

{\centering \includegraphics[width=0.8\linewidth]{figures/chap6/f617_ED_biogeochemistry} 

}

\caption{ ED2.2 biogeochemistry. Schematic of the patch-level carbon cycle solved in ED-2.2 for a patch containing NT cohorts. Like Fig. 6.15, letters near the arrows are the subscripts associated with fluxes. Fluxes shown in solid yellow lines are part of the CO2 cycle, and dashed red lines are part of the carbon cycle but do not directly affect the CO2 flux.(Longo et al 2019)}\label{fig:f617}
\end{figure}

Fig \ref{fig:f618} illustrates some outputs of the ED2 model for a
temperate forest. Fig\ref{fig:f619} shows the output of the model for
tropical ecosystems. The visualization in these figures (6.18a, 6.19a
and b) are produced based on the outputs of ED2.2, they are not
simulated directly by the model, as the model is not spatially explicit
(it is not tracking the location of cohorts). The model simulates carbon
fluxes, but also size distribution, which gives a lot of opportunities,
as it can be compared with many different datasets: fluxtowerdata,
inventory data,

\begin{figure}

{\centering \includegraphics[width=0.8\linewidth]{figures/chap6/f618_ED_harvard} 

}

\caption{ Visualization of the ecosystem composition in the Harvard Forest flux tower footprint as simulated by ED2. Midsuccessional hardwoods (red) dominate, but the footprint also contains early successional hardwoods (green), pines (blue), late successional conifers (magenta) and late successional hardwoods (gray). (b) Distribution of basal area of the different plant functional types across tree diameter classes.(Medvigy et al. 2009)}\label{fig:f618}
\end{figure}

\begin{figure}

{\centering \includegraphics[width=0.8\linewidth]{figures/chap6/f619_ED_amazon} 

}

\caption{ Examples of size, age, and functional structure simulated by ED-2.2, after 500 years of simulation using local meteorological forcing and active fires. (a, b) Individual realization of simulated stands for sites (a) Paracou (GYF, tropical forest); (b) Brasília (BSB, woody savanna). The number of individuals shown is proportional to the simulated stem density, the distribution in local communities is proportional to the patch area, the crown size and stem height are proportional to the cohort size, and the crown color indicates the functional group. (c, d) Distribution of cohorts as a function of size (DBH and height) and age since last disturbance (patch age) for sites (c) GYF and (d) BSB. Crown sizes are proportional to the logarithm of the stem density within each patch.(Longo et al 2019)}\label{fig:f619}
\end{figure}

\section{Growth and allometric relations in demographic
models}\label{growth-and-allometric-relations-in-demographic-models}

Allometric relations are a critical driver for growth in demographic
models. Most variables are scaled to the diameter at breast height. The
allometries are like the allometries for biogeochemical models but are
applied to individual (in IBM) and cohorts (in cohort-based models).
Further, these models also use growth specific equations, as can be seen
below where we will typical examples of growth and allometric equations.

The \textbf{diameter height allometry} of the JOBOWA IBM has three
parameters, b1 is breast height (1.5m) and the other parameters are
species specific and can be derived from the maximum height and diameter
that specific species can reach, using empirical equations (Fig.
\ref{fig:f620}).

\[
h = b_1 + b_2D - b_3D^2
\] with \(h\) the tree height in m, \(D\) the tree diameter in m and
\(b_1\) is breast height (1.5m), \(b_2\) and \(b_3\) empirical parameter
following: \[
b_2=2\left(\frac{h_{max}-1.37}{D_{max}}\right)
\] and \[
b_3 = \frac{h_{max}-1.37}{D_{max}^2}
\]

\begin{figure}

{\centering \includegraphics[width=0.8\linewidth]{figures/chap6/f620_HD_allom} 

}

\caption{Height-diameter allometry in the JABOWA individual based model.(Buggman 2001)}\label{fig:f620}
\end{figure}

Allometric equations have also been developed for \textbf{biomass},
where an exponential relationship is used. ai and bi are species
specific parameters. This leads to different curves for different
species (see Fig \ref{fig:f621}). For the tropics, no such equations are
available, and the use of pan-tropical equations or average equations
based on destructive measurements are used.

\[
M_i=a_iD^{b_i}
\] with \(M_i\) the mass of individual \(i\), \(D_i\) its diameter and
\(a_i\) and \(b_i\) empirical coefficients.

\begin{figure}

{\centering \includegraphics[width=0.8\linewidth]{figures/chap6/f621_BD_allom} 

}

\caption{Relationships between stem diameter and (a) aboveground biomass and (b) height for sugar maple, yellow birch, beech, and red spruce trees at the Hubbard Brook Experimental Forest, New Hampshire using allometric relationships. (Bonan)}\label{fig:f621}
\end{figure}

The \textbf{leaf biomass allometry} is also an exponential relation
between the diameter and leaf biomass, which can then be recalculated to
LAI using the specific leaf area (SLA). This allometry is not often used
in forestry, but is interesting from a modelling point of view, as the
size and area of the crown determines the competition for light with
other trees. \[
M_{leaf}=cD^2
\] with \(M_{leaf}\) the mass of leaves, \(D\) the diameter and \(c\) an
empirical coefficient.

\begin{figure}

{\centering \includegraphics[width=0.8\linewidth]{figures/chap6/f622_LD_allom} 

}

\caption{Leaf area - diameter allometry in the JABOWA individual based model. (Buggman 2001)}\label{fig:f622}
\end{figure}

The \textbf{growth equation} used in the JABOWA model is shown in Fig
\ref{fig:f623}. This is a key equation in the model to calculate the
diameter increment, as the other variables are calculated based on the
diameter. IBM's apply growth functions every year, whereas cohort-based
models typically apply them at a shorter calculation step, mostly every
month. The equation of JABOWA is an empirical equation, based on
inventories, and the diameter is the only state variable. The equation
calculates the maximum obtainable growth (first 4 terms) and multiplies
this with a correction factor (values between 0 and 1), which is based
on environmental conditions. This results in the fact that the obtained
growth in the model is different for every individual, based on the
competition it has with neighboring trees for light and other resources.

\[
\frac{\Delta D}{\Delta t}=GD(1-\frac{DH}{D_{max}H_{max}})\frac{1}{b(D)}f(e)
\] with \(D\) the tree diameter, \(H\) its height, \(t\) the time and
\(b\) and \(f(e)\) empirical coefficients.

\begin{figure}

{\centering \includegraphics[width=0.8\linewidth]{figures/chap6/f623_jabowa_growth} 

}

\caption{The JABOWA equation of maximum tree growth plotted for a tree with Hmax = 40 m, Dmax = 285 cm, and G = 143 cm/yr. (Buggman 2001)}\label{fig:f623}
\end{figure}

The figure \ref{fig:f624} illustrates the allometries and growth factor
of two tree species in an IBM, subplots a and b represent the height-D
allometry and the leaf area -D allometry for two species. Subplot c
shows the cumulative leaf area index vs the fraction of full sunlight
and subplot d indicates the effect of light availability on the growth
factor. Exceptionally, the growth factor can be higher than one, for the
shade tolerant species in full sunlight. This indicated that the maximum
obtained growth can be higher that the modeled maximum under certain
circumstances.

\begin{figure}

{\centering \includegraphics[width=0.8\linewidth]{figures/chap6/f624_growthfactor} 

}

\caption{Tree height and light competition in a forest gap model. (a) Height in relation to stem diameter for two species of trees. (b) Leaf area in relation to stem diameter. (c) Light profile in relation to cumulative leaf area index. (d) Light growth factors for shade tolerant and intolerant species. Height is shown for basswood (Tilia americana) and blackjack oak (Quercus marilandica) with parameters from a model of forests in eastern North America (Pastor and Post 1985). Leaf area and light extinction are from FORET (Shugart 1984). (Figure from Bonan)}\label{fig:f624}
\end{figure}

Figure \ref{fig:f625} illustrates other dependencies of the growth
factor on environmental conditions, for the dependency on growing days,
drought and nitrogen availability. Some models also link the growth
factor to the basal area of the total stand, where a high basal area
lowers the growth factor, because of the high competition.

\begin{figure}

{\centering \includegraphics[width=0.8\linewidth]{figures/chap6/f625_growth_factorbis} 

}

\caption{Growth factors for (a) temperature (growing degree-days), (b) soil moisture, and (c) soil nitrogen illustrated for basswood and blackjack oak (Pastor and Post 1985). Basswood grows at more northern locations than blackjack oak, on wetter soils, and is intolerant of low soil nitrogen. (Bonan)}\label{fig:f625}
\end{figure}

Figure \ref{fig:f626} shows the growth functions of the two species
discussed above. Basswood is more a pioneer species as its peak annual
growth falls early, whereas oak is more a climax tree, as its peak falls
later and is more spread over time. The maximum growth shown on these
curves is then corrected by the growth factor which is shown in the
figures above.

\begin{figure}

{\centering \includegraphics[width=0.8\linewidth]{figures/chap6/f626_growth strategies} 

}

\caption{Stem diameter growth in relation to age as represented in gap models. Relationships are shown for basswood (Dmax = 100 cm, hmax = 30 m, G0 = 188.7 cm y–1, maximum age 140 years) and blackjack oak (Dmax = 50 cm, hmax = 15 m, G0 = 34.0 cm y–1, maximum age 400 years). Parameter values are from Pastor and Post (1985). (Bonan)}\label{fig:f626}
\end{figure}

\section{Competition for light in demographic
models}\label{competition-for-light-in-demographic-models}

In this section we focus on how is competition for light is implemented
in demographic models. This is important because it largely determines
the dynamics of the forests.

Cohort based models are usually not specially explicit. This implies
that plants have no specific place in the patch, so it is impossible to
know which individual has an influence on another, e.g.~which
individuals shades another individual. To be able to apply the
exponential decay of light in the canopy, we need to have a vertical
canopy profile. This needs to be build with the cohorts that grow in the
patch and can be done in different ways (Fig \ref{fig:f627}), depending
on the assumption we make on the shape of the crowns and the spatial
(vertical) organization of these crowns within the patch. The easiest
approach is the approach of ``flat-topped crowns''. In this approach,
each cohort has all its leaf area at its maximum height and the largest
tree in the patch is shading all the other trees within the patch. A
second approach is the perfect plasticity approximation (PPA). This
approximation assumes that the available horizontal space in the patch
is first filled by the largest trees, followed by the lower trees, until
a layer of LAI = 1 is reached. The remaining trees are then considered
as understory trees. In this approach, smaller trees that are still in
the top layer are not shaded by the highest tree, which is more
realistic than the first approximation. This approach assumes that
crowns are not interfering, so crowns don't grow into each other
(i.e.~are perfectly plastic). It further results in a critical height
that represents the height that is needed to be in the overstory, and
thus to be fully sunlit (Fig \ref{fig:f628}).

Among various demographic models, there are different approaches. Some
models assume multiple layers within one cohort, so only the top layer
can be sunlit. There are attempts to model in three dimensions, but this
will not be discussed in this course.

Both methods have advantages and disadvantages: e.g.~shading by
neighboring trees is not accounted for in the PPA model, while it is in
some way included in the flat crown approach.

\begin{figure}

{\centering \includegraphics[width=0.8\linewidth]{figures/chap6/f627_light_comp} 

}

\caption{Representation of plant canopies in vegetation dynamics models.(a) Actual canopies have complex three-dimensional structure determined by the spatial location of trees and their crown geometry. (b) Gap models simplify the canopy so that the crown of an individual tree is a thin, flat layer of leaves at the top of the tree. Shown are six trees with different heights (white vertical lines). The leaf area of each tree (black horizontal lines) spreads over the area of the patch so that the tallest tree shades all others, and so forth, through the canopy. Horizontal positioning is for illustration only. The models do not represent spatial location, only the vertical dimension. Darker shading denotes progressively less light deeper in the canopy. The ecosystem demography model (ED) uses a similar concept, but applied to cohorts rather than individual trees. (c) The perfect plasticity approximation (PPA) organizes canopies into layers. All cohorts in the overstory (white boxes) receive identical light for that layer. The understory (black boxes) receives less light. Spatial location is for illustration only. (Bonan)}\label{fig:f627}
\end{figure}

\begin{figure}

{\centering \includegraphics[width=0.8\linewidth]{figures/chap6/f628_ppa} 

}

\caption{Canopy organization in the perfect plasticity approximation. Shown are 10 cohorts arranged from tallest to shortest. The cumulative crown area of the six tallest cohorts (C1–C6) sums to one, and these form the overstory. The remaining cohorts (C7–C10) form the understory. The height z∗ separates the two canopy layers. (Bonan)}\label{fig:f628}
\end{figure}

We further some critical issues of these approximations. If we consider
an open canopy (no canopy closure yet), the flat top crown will simulate
shading, while there is none. Also, the impact of a height advantage is
magnified in the flat top crown approach, i.e.~the highest tree has a
huge advantage over all the other trees. This leads to monodominance,
which makes it difficult to simulate co-existence of multiple trees.

In the PPA approximation (Fig \ref{fig:f629}), it is assumed that gaps
are filled, which removes magnification of small height advantages, but
it assumes that all trees in the top layer are fully sunlit, which is
likely not the case in reality. Also, all understory trees have all the
same light environment in this model, which is different from reality.

\begin{figure}

{\centering \includegraphics[width=0.8\linewidth]{figures/chap6/f629_ppa_purves} 

}

\caption{Crown shapes modelled by the Ideal Tree Distribution model (ITD) following the perfect plasticity approximation (Purves et al. 2007)}\label{fig:f629}
\end{figure}

The simulated vegetation dynamics (growth, mortality, recruitment) can
have an impact on the light regime within the patch and the light
availability of the respective cohorts. Figure \ref{fig:f630} shows how
growth can lead to demotion (opposite of promotion) of certain cohorts.
Subplot a: the growth of the largest cohorts leads to an increased LAI
of these cohorts. Because of that, they become too large to fit as one
lai layer in the patch, so a part of the smallest cohort will be demoted
to the shaded canopy, creating a new cohort. For mortality, there are
two options: in subplot b, mortality leads to the death of a certain
fraction of the crown of each cohort, resulting in a more open space,
and thus place for new cohorts to reach the top of the canopy. The new
open space is filled starting with the highest cohort of the shaded
cohorts, until the top layer is filled again. In subplot c, the
mortality leads to new open space which is used to create new patches of
newborn vegetation. Both approaches lead to a different forest dynamic.

\begin{figure}

{\centering \includegraphics[width=0.8\linewidth]{figures/chap6/f630_canopy_dynamics} 

}

\caption{Depiction of cohort and patch dynamics. (a) Growth of cohorts leads to crown expansion, resulting in splitting of a cohort and demotion to the understory. Shown are four overstory cohorts (C1–C4). Part of the shortest cohort (C4) is demoted to the understory to form a new cohort. (b) Mortality leads to open canopy space that can be filled by promotion of understory cohorts to the overstory. In this example, the two understory cohorts form new cohorts (C5, C6) in the overstory. Patch area is unchanged. (c) The open canopy space upon mortality can also be used to create a new patch that is filled by new cohorts. The area of the old patch decreases, and a new patch is formed from the open canopy area.(Bonan)}\label{fig:f630}
\end{figure}

\section{Competition for water and nutrients in demographic
models}\label{competition-for-water-and-nutrients-in-demographic-models}

The implementation of competition for water and nutrients depends on how
detailed the soil is represented in the model. The soil representation
varies between models, ranging from a one-bucket model (i.e.~one soil
water reservoir with inputs and outputs), to multiple soil layers with
an explicit rooting depth or profile. In demographic models (i.e.~models
with multiple PFTs), plants can compete with each other. In these
models, the rooting profile is often the key to determine competition
for water and nutrients; e.g.~a plant with superficial roots will be
more vulnerable to drought than a plant with roots that reach the ground
water table.

Models also differ in the scale at which they represent the competition
for resources. Some models consider that all patches in the same
gridcell have access to the same resource pool (fig \ref{fig:f631} b),
while other models track a resource pool for each patch separately (fig
6.31 a). In the former case, water uptake by a tree in for example patch
3, has a direct impact on the water availability for trees in patch 1.

The below ground competition in models is generally less well developed
than the above ground competition for resources. This is mainly because
there is less information and data available on variables that determine
the below ground competition, such as root distribution, root biomass in
different soil layers, \ldots{}

\begin{figure}

{\centering \includegraphics[width=0.8\linewidth]{figures/chap6/f631_root_competition} 

}

\caption{Illustration of resource partitioning in vegetation demographic models. (a, b) show two alternative depictions of resource partitioning in an age-since-disturbance resolving (ED-type) model. In (a) resources (water/nutrients) are resolved for each age-since-disturbance patch, meaning that different extraction levels can affect resource availability over the successional gradient, a situation made more likely by large spatial-scale disturbances. In (b) all patches share a common pool, a situation more relevant to smaller (individual) scale disturbances. (Fischer et al. 2018)}\label{fig:f631}
\end{figure}

\section{Seed dispersal and
recruitment}\label{seed-dispersal-and-recruitment}

In most models, recruitment and dispersal are stochastic processes, and
the model implementations are much less process-based compared to for
example photosynthesis. Most models assume that seeds of all PFTs that
potentially grow in the patch are available everywhere in the patch,
i.e.~there is a constant available seedbank for all PFTs. By
consequence, most models don't simulate seed arrival, germination and
establishment, but only how many new young trees establish. However,
some models are more spatially explicit, and only allow establishment of
certain PFTs, only those PFTs that grow in the neighborhood of the
establishment site (seed input of neighboring patches allows for
gap-colonization by specific PFTs).

Establishment of new seedlings further depends on light availability and
other climatic conditions, such as water availability or temperature.
The initial dimensions of new seedlings/saplings are highly model
dependent. Some models assume a new tree to be 0.5 m high from the
start, without explicitly simulating the growth of that young tree,
while other models assume a variation in the size of new young trees.

\section{Mortality}\label{mortality}

Mortality is typically a stochastic process, which consists of three
different components. First, there is a \textbf{background mortality}.
This is a constant probability for a tree/plant to die each year
throughout the tree life, which typically ensures that only a limited
number (a few percent) of trees reach the maximum age (maximum age is a
PFT dependent parameter in most models). In area-based models, this
component is included by removing a certain fraction of the wood biomass
each year, whereas in demographic models, a certain number of
individuals dies each year.

The second component of mortality is the \textbf{stress related
mortality}. Typical examples of this are drought stress or heat stress.
Some models link these stresses to actual physiological processes
simulated by the model, e.g.~if actual drought stress is simulated on
photosynthesis, the model also increases the mortality because of the
drought (process based mortality). Other models, such as the JABOWA
model (shown in figure \ref{fig:f632}), link mortality empirically to
the diameter growth of trees (higher mortality rates for low growth
rates). In this case, trees that only grow very little according to
their prescribed growth curve, have a higher mortality. Figure
\ref{fig:f632} shows how mortality over the life cycle of a tree is
included in the JABOWA model. The full line is the background mortality.
On the left hand, the stress related mortality for young trees is shown.
Young trees have a higher mortality because they have only a limited
diameter growth, as they are mostly shaded and heavily competing in
their early life stages. On the right hand of the graph, the increased
stress related mortality for old trees is shown. Again, the mortality
becomes higher as older trees show less diameter growth. The total
probability of mortality over the lifetime of a tree is shown by the
dark black line and is the result of the sum of all individual
probabilities.

Third, there is \textbf{disturbance related mortality}. This mortality
is related to individual events, such as wind or fire. The simplest
models assume a constant annual mortality related to disturbance, while
more complicated models link this mortality to, for example, wind speed
or have an explicit fire module. Mortality is, just like recruitment and
dispersal, not fully developed, because our knowledge on these processes
is rather limited.

\begin{figure}

{\centering \includegraphics[width=0.8\linewidth]{figures/chap6/f632_mortality_jabowa} 

}

\caption{Changes of mortality patterns with tree age and their approximation in the JABOWA model. AIM – age-independent ‘background’ mortality; SM – stress-related mortality.(Buggman 2001)}\label{fig:f632}
\end{figure}

\section{Conclusion}\label{conclusion}

Demographic models (individual based or cohort based) simulate forest
dynamics as an outcome of the life history traits (such as maximum
height, maximum age and climate tolerance), attributed to the species or
the PFT. In these models, the different PFTs compete with each other,
limited by their life history traits, resulting in a vegetation
composition. When we provide other input values for the life history
traits, the simulated vegetation community (PFT composition, size
distribution) will also be different. For this reason, these models are
also called `trait filtering models': properties of the
individuals/cohorts determine which species can become dominant during
succession.

From that perspective, these models could be used to replace standard
species distribution models, which need species abundance data, climate
data and soil data to predict where a certain species can occur. In a
well-working demographic model, only species that suit to a certain
place and climate are able to grow. For example, in Belgium, no tropical
trees would survive, but temperate tree species would, and will dominate
the vegetation in the model after some time. However, more data on more
traits should be accounted for to allow such model to predict species
distribution reliably. In the meantime, demographic models are often run
with prescribed potential species or PFTs that can grow in a certain
region, which results in figures as Figure \ref{fig:f633}.

Figure \ref{fig:f633} b shows the stem density at a certain point in
time during the simulation. It shows a typical pattern, namely that of
multiple small diameter trees, and only a few large diameter trees.
Subplot c shows the biomass distribution over the diameter classes. Here
we see that most of the biomass is concentrated in the trees with an
average diameter, and less in trees with a small, or large diameter. The
results of this model can be related to inventory data to validate the
model output.

\begin{figure}

{\centering \includegraphics[width=0.8\linewidth]{figures/chap6/f633_LM3_ppa_succession} 

}

\caption{Forest dynamics simulated by LM3-PPA. Shown is (a) basal area of aspen (Populus tremuloides), red maple (Acer rubrum), and sugar maple (Acer saccharum) in relation to stand age. Also shown for year 80 are (b) stem diameter distribution and (c) biomass distribution. (Bonan))}\label{fig:f633}
\end{figure}

Area-based models are the models that are now standard in global models,
but cohort-based models will likely take over this task in the future.
The table below gives a general overview of the main properties of area-
based models, cohort-based models and gap models. Please note that not
every model can fit exactly into one of these groups, some models
combine properties of different groups.

Models are not only used to make simulations or predict what will happen
in the future, but they can also be used as a diagnostic tool. Combing
the model with available data (model-data fusion) leads to better
estimates of parameters that are more difficult to derive from the
original dataset that was used to feed the model. This allows people to
extract more information from the data.

\begin{figure}

{\centering \includegraphics[width=0.8\linewidth]{figures/chap6/f634_table} 

}

\caption{Summary of the differences between area based, cohort based and individual based vegetation models.}\label{fig:f634}
\end{figure}

\section{Case study 6.1}\label{case-study-6.1}

\textbf{Bohn, F. J., May, F., \& Huth, A. (2018). Species composition
and forest structure explain the temperature sensitivity patterns of
productivity in temperate forests. Biogeosciences, 15(6), 1795-1813.}

The FORMIND model is an individual-based model that was used in this
study to simulate how the forest structure and composition affects the
temperature sensitivity of woody productivity. The model showed that
structure and species composition are both relevant for temperature
sensitivity. The results were diverging for young and old forests. In
young forests, low diversity and low heterogeneity resulted in a
positive impact of the temperature on the NPP, whereas in old forest,
the opposite is true. If you plant a new forest, the authors therefore
advice to start with a closed, even aged pioneer canopy, with a climax
species rich understory.

\begin{figure}

{\centering \includegraphics[width=0.8\linewidth]{figures/chap6/f635_bohn1} 

}

\caption{Overview of drivers influencing wood production. External variables in this study are temperature, radiation and precipitation. Forest properties are divided into two groups: species composition properties (e.g. Rao’s Q as a measure of functional diversity and species distribution index OAWP) and forest structure properties (e.g. forest height, leaf area index and tree height heterogeneity). (Bohn et al. 2018)}\label{fig:f635}
\end{figure}

\begin{figure}

{\centering \includegraphics[width=0.8\linewidth]{figures/chap6/f636_bohn2} 

}

\caption{Overview of forest properties and resulting temperature sensitivity of AWP of three exemplary forests: (a) old even-aged spruce forest; (b) mature deciduous forest; (c) a quite young mixed species forest. The middle (panels d, e and f) shows the corresponding stem size distributions and provides information on the highest tree in the forest (Hforest) and species distribution index OAWP (which quantifies the suitability of a species distributed within the forest structure with regard to AWP). Each forest is treated with 320 climate time series; the last column (panels g, h and i) shows the AWP as a function of mean annual temperature (MAT). The colours indicate different inter-annual temperature amplitudes (Q95) of the used time series. (The coloured bands show the standard deviation due to the variability of the five different time series that exist for each combination of mean annual temperature and intra-annual temperature amplitude.(Bohn et al. 2018).}\label{fig:f636}
\end{figure}

\section{Case study 6.2}\label{case-study-6.2}

\textbf{Levine, N. M., Zhang, K., Longo, M., Baccini, A., Phillips, O.
L., Lewis, S. L., \ldots{} \& Moorcroft, P. R. (2016). Ecosystem
heterogeneity determines the ecological resilience of the Amazon to
climate change. Proceedings of the National Academy of Sciences, 113(3),
793-797.}

This study used a demographic model (ED2) to find out whether or not
there would still be a dieback of the Amazon forest if demographic
processes are included in the model (compared to the dieback simulated
by the area-based model in Fig \ref{fig:f68}). To test the effect of
demography, they used the ED2 model, and a big leaf version of that
model, which has no demographic processes. Figure \ref{fig:f637} shows
some of the results of the study. The left side of the figure shows the
result of the model with demography, the left side shows the results of
the big leaf model. Different colors indicate different dry season
lengths. Subplot A and C show the AGB vs clay fraction. The demographic
model shows a smooth relation, while the big leaf model simulated big
jumps in the driest patches. The same results were obtained when AGB was
plotted vs plant water stress. In the demographic model, a smooth
relationship appears, whereas the big leaf model suddenly drops to zero
and predicts a dieback.

\begin{figure}

{\centering \includegraphics[width=0.8\linewidth]{figures/chap6/f637_levine1} 

}

\caption{Impact of changes in soil clay fraction (A and B) and plant water stress (C and D) on AGB in the ED2 (A and C) and ED2-BL (B and D) model simulations. Four climatological conditions are shown, a 2-month dry season, a 4-month dry season, a 6-month dry season, and an 8-month dry season.(Levine et al. 2016)}\label{fig:f637}
\end{figure}

The next Figure \ref{fig:f638} shows the same pattern, in A, the green
dots show a gradual decrease of AGB with increasing dry season length,
while the purple dots (big leaf model) remain at the same level until a
threshold is reached and the system goes to an AGB of zero. The black
dots show what is observed via satellites, and correspond more to the
green dots.

\begin{figure}

{\centering \includegraphics[width=0.8\linewidth]{figures/chap6/f638_levine2} 

}

\caption{(A) Change in AGB with DSL (dry season length) for remote sensing-based estimates (black and gray circles), ground-based plot measurements (blue triangles), ED2 model output (green circles), and ED2-BL model output (purple circles).(B) Distribution of AGB in the observations and the two models. (C) Change in the percentage of biomass variability, with the coefficient of variation (CV) defined as 1σ/mean. Results are for undisturbed primary vegetation forests. (Levine et al. 2016) }\label{fig:f638}
\end{figure}

\chapter{Representing functional biodiversity in vegetation
models}\label{representing-functional-biodiversity-in-vegetation-models}

\chaptermark{Dynamics}

\section{Introduction}\label{introduction-3}

This chapter wil focus on the biodiversity perspective of vegetation
models. The first question to address is: why do we want to represent
biodiversity in vegetation models? Some initial reasons why biodiversity
is key: - From a conservation perspective it is hey to have the ability
to simulate species abundance - Ecosystem resilience is depending on its
diversity - Species composition is impacting biogeochemical feedbacks to
the atmosphere.

In this chapter, the main question to address is: \textbf{How to
represent biodiversity in vegetation models?} This question can be
translated into two subquestions: - How do we parameterize PFTs? (in the
different types of models described in chapter 6) -- Wich parameters can
we add to PFTs to represent different types of biodiversity. - Do we
really need these PFTs? Are there any alternatives?

\section{Functional diversity}\label{functional-diversity}

Diversity can be studied from multiple perspectives. In this chapter,
the focus will be on tree species diversity (as trees are the dominant
growth form in forests). For example, for many temperate forests in
Europe, the dominant PFT are beech trees; for most model applications it
is unnecessary to consider the small understory plants in beech forests.
However, for highly diverse systems, like tropical forests, it is much
more complicated to represent tree species diversity in a model.

There are at least 400.000 plant species in the world. Figure
\ref{fig:f71} shows the species number per ecoregion. It is possible to
identify biodiversity ``hot spots'' in tropical and mountain areas on
the map. It is estimated that 50.000 tree species exist globally in
tropical forests. However, the diversity is quite different between the
continents, with fewer tree species in Africa and a very limited amount
of tree species overlapping between the tropical continents. Region
specific parameterization for global vegetation models in the tropics is
a topic is ongoing research in modelling, because most current
vegetation models assume the same parameterization and PFTs in the
Amazon and Congo Basin for example.

\begin{figure}

{\centering \includegraphics[width=0.8\linewidth]{figures/chap7/f71_species_map_Kier} 

}

\caption{Estimated vascular plant species richness per ecoregion (Kier et al. 2005)}\label{fig:f71}
\end{figure}

It is clear that is not possible to simulate a specific parameterization
for every single tree species on Earth. Therefore, we focus on the
\textbf{dominant species} (trees) and we abandon the taxonomic framework
and pass to a functional framework -- grouping species into functional
groups. In such a functional approach we are not interested in the
number of species but in how functionally diverse the forest is. In a
funct framework we describe trees not by their taxonomy but by their
functional properties, the so-called functional traits.

In other words, \textbf{functional diversity} refers to those components
of biodiversity that influence how an ecosystem operates or functions.
The biological diversity, or biodiversity, of a habitat is much broader
and includes all the species living in a site, all of the genotypic and
phenotypic variation within each species, and all the spatial and
temporal variability in the communities and ecosystems that these
species form. Functional diversity, which is a subset of this, is
measured by the values and range in the values, for the species present
in an ecosystem, of those \textbf{organismal traits that influence one
or more aspects of the functioning of an ecosystem}. Functional
diversity is of ecological importance because it, by definition, is the
component of diversity that influences ecosystem dynamics, stability,
productivity, nutrient balance, and other aspects of ecosystem
functioning (Tilman 2001).

A \textbf{functional trait} is any morphological, physiological or
phenological feature measurable at the individual level, from the cell
to the whole-organism level, without reference to the environment or any
other level of organization. It is functional if it affects fitness
indirectly via its effects on growth, reproduction and survival (Figure
\ref{fig:f72}).

\begin{figure}

{\centering \includegraphics[width=0.8\linewidth]{figures/chap7/f72_violle} 

}

\caption{Arnold’s (1983) framework revisited in a plant ecology perspective. Morpho-physio-phenological (M-P-P) traits (from 1 to k) modulate one or all three performance traits (vegetative biomass, reproductive output and plant survival) which determine plant performance and, in fine, its individual fitness. M-P-P traits may be inter-related (dashed double-arrows). For clarity, interrelations among performance traits and feedbacks between performance and M-P-P traits are not represented. (Violle et al. 2007)}\label{fig:f72}
\end{figure}

Important to note that not all plant traits can be defined as
functional. Plant proprieties can be measured at different scales:
individual, population, community, and ecosystem-level (Figure
\ref{fig:f73}). This course mainly focuses on individual traits,
measured in individual plants, which can impact higher scale levels.

\begin{figure}

{\centering \includegraphics[width=0.8\linewidth]{figures/chap7/f73_violle2} 

}

\caption{Pathways linking the challenge of interest of different organizational levels, through their related inherent components, to some examples of traits found in the literature. Without trait-based information, scaling-up to higher organizational levels needs complex integration information (I). Thus fitness components of an individual determine the components of the finite rate of increase (lambda) of the population (Ii-p). Occurrence and frequency of species at the community level encompass components of lambda through complex integration (e.g. biotic interactions) (Ip-c). Finally, scaling-up to ecosystem properties can be done by combining functional property of each species of the community (Ic-e). Using traits as proxies of a process at a particular organizational level can sometimes be done without such integration function. For example, at the ecosystem level, ecosystem productivity (one component of ecosystem functioning) shows a strong positive relationship with plant height (an effect trait) (Saugier et al. 2001; Violle et al. 2007)}\label{fig:f73}
\end{figure}

\subsection{Trait correlations reflect plant
strategy}\label{trait-correlations-reflect-plant-strategy}

Certain trait combinations have proven to be successful during
evolution. For example, leaves with high nitrogen content are mostly
leaves with a high specific leaf area (and the opposite is true as
well). In contrast, very tick leaves with high nitrogen content are very
uncommon; this combination was not successful during evolution. At the
leaf level, we call this spectrum of treai combinations the \textbf{leaf
economics spectrum (LES)} (Figures \ref{fig:f74} and \ref{fig:f75}) -- a
spectrum of trade combinations that have proven successful and represent
a plant strategy. - At one site of the spectrum we find leaves with high
photosynthetic rates, high nitrogen content and very thin leaves --
short lifetime. - At the other end of the spectrum we find leaves that
invested more in the structure but with lower photosynthetic rates and
nutrients content -- longer lifetime.

\begin{figure}

{\centering \includegraphics[width=0.8\linewidth]{figures/chap7/f74_LES} 

}

\caption{Illustration of the leaf economics spectrum.}\label{fig:f74}
\end{figure}

The LES can also be related to growth and plant strategies: for example
nitrogen conservative versus nitrogen spending plants. If we want to
simulate different groups of species the LES offers an interesting
framework, we can consider pioneer trees that grow fast and have leaves
situated at the fast return on investment side, and we can have late
successionial trees with leaves situated at the slow return on
investment side of the LES.

\begin{figure}

{\centering \includegraphics[width=0.6\linewidth]{figures/chap7/f75_LES_wright} 

}

\caption{The leaf economic spectrum. Three-way trait relationships among six leaf traits with reference to LMA (leaf mass per area), one of the key traits in the leaf economics spectrum. The direction of the data cloud in three-dimensional space can be ascertained from the shadows projected on the floor and walls of the three-dimensional space. Sample sizes for three-way relationships are necessarily a subset of those for each of the bivariate relationships. a) Amass, LMA and Nmass; 706 species. b) LL, Rmass and LMA; 217 species. c) Nmass, Pmass and LMA; 733 species. d) Aarea, LMA and Narea; 706 species. (Wright et al. 2004)}\label{fig:f75}
\end{figure}

In parallel to the LES a similar spectrum was developed for wood trait
correlations (Chave et al. 2009), the \textbf{wood economics spectrum
(WES)} -- representing correlations between wood proprieties and plant
strategies (Figure \ref{fig:f76}). Her trees with low wood density are
typically fast growing pioneer trees and and trees with high wood
density are typically slow growing late successional trees. Years after,
Diaz et al. (2015) integrated various plant traits (including leaf,
wood, seed and whole plat traits) into the \textbf{plant economics
spectrum (PES)} -- woody and not woody species can be clearly separated
in this analysis (Figure \ref{fig:f77}). The correlations described by
these economic spectra are very useful to vegetation modellers because
it helps to define plant functional types in a more constrained way.

\begin{figure}

{\centering \includegraphics[width=0.8\linewidth]{figures/chap7/f76_WES} 

}

\caption{Illustration of the wood economics spectrum. Relationship between wood density and relative growth rate (log-transformed, a), and mortality rate (log-transformed, b), for two tropical forest sites (Barro Colorado Island, Panama, white circles, and Pasoh, Malaysia, black circles). All correlations were highly significant (P < 0.001), and the correlation coefficients ranged between r2 = 0.13 and 0.19. Demographic data were collected from saplings 1–5 cm in diameter. (Chave et al. 2009).}\label{fig:f76}
\end{figure}

\begin{figure}

{\centering \includegraphics[width=0.8\linewidth]{figures/chap7/f77_PES} 

}

\caption{The global spectrum of plant form and function. a, Projection of global vascular plant species (dots) on the plane defined by principal component axes (PC) 1 and 2 (based on PCA of gloabl trait data). Solid arrows indicate direction and weighing of vectors representing the six traits considered; icons illustrate low and high extremes of each trait vector. Circled numbers indicate approximate position of extreme poles of whole-plant specialization, illustrated by typical species. The colour gradient indicates regions of highest (red) to lowest (white) occurrence probability of species in the trait space defined by PC1 and PC2, with contour lines indicating 0.5, 0.95 and 0.99 quantiles. Red regions falling within the limits of the 0.50 occurrence probability correspond to the functional hotspots. b, c, location of different growth-forms (b) and major taxa (c) in the global spectrum. (Diaz et al. 2015).}\label{fig:f77}
\end{figure}

\subsection{Environmental (trait)
filtering}\label{environmental-trait-filtering}

The principle of Environmental filtering describes how species abundance
is determined by the environment and is illustrated in Figure
\ref{fig:f78}. In relation to other community assembly processes species
abundance changes across an environmental gradient, because certain
environmental factors will constrain for some plants with certain
proprieties/traits. In the end certain traits combinantions are
`filtered' by a combination of environmental factors (climate, soil,
disturbance, topography and biotic interactions). I other words, there
are different possible reasons why a plant (species) does not occur at a
specific location: the abiotic site conditions, distance from the seed
source, local competitors, \ldots{}

\begin{figure}

{\centering \includegraphics[width=0.8\linewidth]{figures/chap7/f78_kraft_filtering} 

}

\caption{Environmental filtering in relation to other community assembly processes in the context of species abundance changes across an environmental gradient. Firstly, a species may be absent from a focal site on the gradient because of dispersal limitation. Next, environmental filtering (sensu stricto) occurs when a species arrives at a focal site but fails to establish or persist with neighbours removed. Competitive exclusion occurs when a species arrives and can persist in the absence of neighbours but not in their presence. Finally, at a different focal site, within‐site abiotic heterogeneity (not typically defined as environmental filtering) can contribute to the ability of community members to persist locally. Note that in this hypothetical example, the observed pattern of species abundance shifts across the gradient emerges from the combined action of all four processes. (Kraft et al. 2015).}\label{fig:f78}
\end{figure}

\subsection{TRY global plant trait
database}\label{try-global-plant-trait-database}

To be able to parameterize PFTs globally, plant trait data are
essential. Various efforts have been done in the past two decades to
synthesize global plant trait databases. The most extended and most
important one is the TRY database. Thousands of scientists around the
world have been contributing millions of trait data values to TRY in the
past years. Figure \ref{fig:f79} shows gives an overview of the
geographical representativeness of the current TRY database 5.0 (2020).
It shows that that some regions, for example the Congo Basin, still have
a considerable gap in the proportion of species covered by TRY 5.0
represented in that area, while others (e.g.~Europe) have a really good
coverage.

\begin{figure}

{\centering \includegraphics[width=0.6\linewidth]{figures/chap7/f79_try_kier} 

}

\caption{Geographic representativeness of the TRY database: (a) the number of species with at least one trait measurement in an ecoregion in TRY version 5; (b) number of species per ecoregion estimated by Kier et al. (2005); (c) fraction of species represented in TRY version 5 versus number of species per ecoregion estimated by Kier et al. (2005). (Kattge et al. 2020)}\label{fig:f79}
\end{figure}

From the TRY database it is possible to derive trait frequency
distributions (Figure \ref{fig:f710}). Which are key to parameterize
vegetation models and their PFTs. The Figure \ref{fig:f710}c shows for
example the specific leaf area (SLA) distribution for Pinus sylvestris
compared to all needlleleaves trees and to all plants growing in natural
environments. While Figure \ref{fig:f711} shows the SLA distrubutions
for various PFTs compared to the actual values used in a number of
global vegetation models (in red). The analysis of the TRY database has
also shown that some of for some traits the global variation can be well
explained by the PFT factor (illustrating the strength of the PFT
approach), while for other traits the PFT factor only explains a very
limited part of the global variation (illustrating the limitations of a
PFT approach).

\begin{figure}

{\centering \includegraphics[width=0.8\linewidth]{figures/chap7/f710_try_distr} 

}

\caption{Examples of trait frequency distributions for four ecologically relevant traits in the first version of the TRY database. Upper panels: (a) seed mass and (b) plant height for all data and three major plant growth forms (white, all database entries; light grey, herbs/grasses; dark grey, trees; black, shrubs). Rug-plots provide data ranges hidden by overlapping histograms. Lower panels: (c) Specific leaf area (SLA) and (d) leaf nitrogen content per dry mass [Nm, white, all database entries excluding outliers (including experimental conditions); light grey, database entries from natural environment (excluding experimental conditions); medium grey, growth form trees; dark grey, PFT needle-leaved evergreen; black, Pinus sylvestris]. (Kattge et al. 2011)}\label{fig:f710}
\end{figure}

\begin{figure}

{\centering \includegraphics[width=0.8\linewidth]{figures/chap7/f711_try_models} 

}

\caption{Frequency distributions of specific leaf area (SLA, mm2 mg-1) values (grey histograms) compiled in the TRY database and parameter values for SLA (red dashes) published in the context of the following global vegetation models: Frankfurt Biosphere Model (Ludeke et al., 1994; Kohlmaier et al., 1997), SCM (Friend and Cox, 1995), HRBM (Kaduk and Heimann, 1996), IBIS (Foley et al., 1996; Kucharik et al., 2000), Hybrid (Friend et al., 1997), BIOME-BGC (White et al., 2000), ED (Moorcroft et al., 2001), LPJ-GUESS (Smith et al., 2001), LPJDGVM (Sitch et al., 2003), LSM (Bonan et al., 2003), SEIB–DGVM (Sato et al., 2007). n, number of SLA data in the TRY database version 1 per PFT.(Kattge et al. 2011)}\label{fig:f711}
\end{figure}

\section{Representing 400.000 plant species in a single model: the PFT
approach}\label{representing-400.000-plant-species-in-a-single-model-the-pft-approach}

To represent the dominant plant species in a vegetation model, the
scientific community had soon adopted the PFT approach based on the
functional framework described above. Both area-based and demographic
models group species into broader functional groups, but the actual
meaning of a PFT differs amonf model types. Moreover the PFT has evolved
over time both conceptually and in the way they are paramerized (Figure
\ref{fig:f712}). PFTs have evolved from fixed parameterizations
(constant parameter values) to more variable (in space and time)
parameterizations.

\begin{figure}

{\centering \includegraphics[width=0.8\linewidth]{figures/chap7/f712_pft_wullschleger} 

}

\caption{A conceptual diagram showing critical aspects of PFT classification, remote sensing, trait databases and methods of PFT parameterization that will be important as DGVMs develop into the future.(Wullschleger et al. 2014)}\label{fig:f712}
\end{figure}

\subsection{PFTs in area-based models}\label{pfts-in-area-based-models}

Area-based models group plant (ecosystem) types by responses to
resources and climate. They are essentially a simplification based on
biome description and plant functioning at the ecosystem level. These
models typically consider 12-14 PFTs to cover the entire globe (see
Figure \ref{fig:f713} for the PFT adopted by the LPJ model). Due to this
simplification global modelling is possible with a limited amount of
parameters needed. To be able to global modeling the adopted PFTs are
required to (i) represent the world's most important plant types; (ii)
characterize them through their functional behaviour; (iii) provide
complete, geographically representative coverage of the world's land
areas. Dynamic vegetation models can simulated the PFT map (based on
bioclimatic limits) or PFT mapping can be done based on remote sensing,
and such maps can then be used as input for vegetation models. The first
use of PFTs managed to reproduce well the large scale observed gradients
at the global scale, but for current applications, it is insufficient
for some approaches, especially to make future predictions.

The LPJ model is a specific example of an area-based model. LPJ model
has 10 PFTs, and it is mainly distinguished based on plant physiology
(C3/C4 plants), phenology (evergreen/deciduous), physiognomy
(woody/herbaceous) and bioclimate (cold/heat tolerance). Figure
\ref{fig:f713} describes the bioclimatic limits of each PFT that will
determine the occurence of PFTs. The problem with such an approach of
bioclimatic limits is that the entire simulated ecosystem is depending
on a single parameter, which makes it very hard to simulate resilience
to climate variability.

\begin{figure}

{\centering \includegraphics[width=1\linewidth]{figures/chap7/f713_lpj_bioclim_table} 

}

\caption{PFT bioclimatic limits in the LPJ model. Tcmin: minimum coldest temperature for survical; Tcmax: maximum coldest-month temperature for establishment; GDDmin: minimum degree-day sum (5°C base) for establishment; Twcmin: minimum warmests minus coldest month temperature range. (Sitch et al. 2003)}\label{fig:f713}
\end{figure}

In addition, these PFTs are described by a handful of fixed parameter
values, as shown in Figure \ref{fig:f714}, again for the LPJ model. This
illustrating that with only a few values, an entire biome is defined.
Such an approach inherently has limitations. Spatial variation within a
biome will only be driven by climate variation, while in reality we know
that there is a lot of biotic variation within a biome (e.g.~within the
Amazon basin forest). Also here the faith of an entire biome depends on
a few fixed parameters while we know that the hy functional diversity
within the biome (e.g.~the Amazon rainforest) makes the system much more
resilient than we will probable simulate by a single PFT.

\begin{figure}

{\centering \includegraphics[width=1\linewidth]{figures/chap7/f714_LPJ_pft_table} 

}

\caption{PFT parameter values in the original LPJ model. z1 and z2 are the fraction of fine roots in the upper and lower soil layers, respectively; gmin is the minimum canopy conductance; rfire is the fire resistance; aleaf is the leaf longevity; fleaf, fsapwood, froot are the leaf, sapwood and fine root turnover times, respectively; tmort,min is the temperature base in the heat damage mortality function and Sgdd is th egrowing degree day requirement to grow full leaf coverage? (Sitch et al. 2003)}\label{fig:f714}
\end{figure}

\subsection{PFT parameterization in demographic
models}\label{pft-parameterization-in-demographic-models}

Where in area-based models PFTs are describing `ecosystem types', in
demographic models, PFTs are representing the function group to which an
individual orcohort belongs. Within one patch multiple PFTs can compete
and a functionally diverse plant community can be simulated. Figure
\ref{fig:f715} demostrates how temperate forest species are devided into
5 functional groups (PFTs) in the ED2 model. Also here fixed parameter
values are asigend to each PFT (Figure \ref{fig:f716}).

\begin{figure}

{\centering \includegraphics[width=1\linewidth]{figures/chap7/f715_ed2_pft_table} 

}

\caption{Summary of PFTs in the ED2 model for temperate forest. (Medvigy  et al. 2009)}\label{fig:f715}
\end{figure}

\begin{figure}

{\centering \includegraphics[width=1\linewidth]{figures/chap7/f716_ED2_pft_values_table} 

}

\caption{Eco-Physiological, Life-History, and Allometric Parameters for the Plant Functional Types in the ED2 model for temperate forest. (Medvigy et al. 2009)}\label{fig:f716}
\end{figure}

\subsection{Methods to determine PFT
parameters}\label{methods-to-determine-pft-parameters}

A large part of the work of a vegetation modellers entails finding an
appropriate parameterization of the PFTs. There are multiple possible
approaches to determine PFT parameter values and usually a combination
of these approaches is used:

\begin{enumerate}
\def\labelenumi{\arabic{enumi}.}
\tightlist
\item
  PFT parameter values based on literature data
\item
  PFT parameter values based on plant trait databases (e.g.~TRY)
\item
  Calibration of PFT parameters, which is optimizing parameters for
  specific sites or regions to get a best possible fit between model
  outputs and observations.
\item
  Data assimilation. Here PFT parameters are updated during the
  simulation when new data become available.
\end{enumerate}

The parameterisation of PFTs and vegetation models in general is a
challenging task, because there are multiple bottlenecks:

\begin{itemize}
\tightlist
\item
  Model-specific parameters. Many model parameters are specific to the
  equiations adopted in a specific model, and are no necessarily
  measurable in the field or at the scale of interest.
\item
  Some processes are shared by all PFTs, while some processes require
  PFT specific equations (for example phenology of grasses versus
  evergreen trees versus deciduous trees).
\item
  Parameter equifinality. Multiple parameter combinations lead to the
  same (good or bad) model outputs. When we try to optmise parameter
  values, the information content of the data might not be enough to
  constrain parameters. Adding additional data sources can solve that
  problem in some cases.
\item
  Interaction between PFTs. Especially in demographic models multiple
  PFTs are interacting which complicates parameterization.
\end{itemize}

\section{Limitations of the PFT
concept}\label{limitations-of-the-pft-concept}

The PFT concept is widely adopted, but it clearly suffers from some
limitations, especially if we aim to build models that are widemy
applicable, also for future projections. The limitations of the PFT
concept are related to the model type.

Area-based models suffer from multiple limitations. In the first place,
there is a considerable uncertainty in the land cover information we
have available from remote sensing data. Secondly different PFT mapping
methods (to convert a land cover map into a PFT map) give different
results, so these mapping methods clearly add up to the uncertainty.
Thirdly, area-based models inherently do not account for diversity with
the PFT and they described the entire global plant/ecosystem diversity
by only about 10 types. The latter makes the risk for dieback
unrealistically large, because survical of a PFT depends on single
thresholds. Moreover, landscape heterogeneity is not or poorly accounted
for and the functional heterogeneity within a PFT is not represented (we
know that not all boreal forest or all tropical forest are the same in
reality).

Demographic models resolve some of the above described problems and even
alow to make species-specific simulations if the necessary data is
available for parameterization. Howver, for species-rich system, like a
tropical rainforest, this is noyt possible. Using fixed parameter values
for PFTs in demographic models means that intra-species or intra-PFT
trait variability is neglected, while we know that such trait plasticity
is very important for ecosystem resilience.

To overcome these limitations, vegetation modellers are using multiple
strategies. A first option is to introduce more PFTs in the models or to
construct `flexible' PFTs that can be assigned variable trait values.
Such an approach could account for plant strategies (LES, PES),
trait-trait correlations and/or trait environment relationships. It is
clear that all these efoorts require a large amount of trait
observations in space and time. Luckly more and more of such data exist
and become available to a wide research community due to initiatives
like TRY. In the next section we describe a few (recent) approaches to
account for trait variability.

\section{Alternative approaches to account for trait
variability}\label{alternative-approaches-to-account-for-trait-variability}

In the endeavor to account for trait variability we can see three
different approaches in recent years.

A first approach is to account for \textbf{plastic intra-specific trait
variability}. In such an approach we account for physiological response
of plants to there environment and introduce equations that `correct'
the standerd PFT parameter values for changing environmental conditions.
Such approaches are already widely adopted in models for example to
account for the temperature acclimation of photosynthesis (e.g.~VCmax
temperature dependency functions discussed in chapter 2). The advantage
of this approach is that is relatively easily applicable in standard
PFT-based models.

In a second approach we need a different type of models:
\textbf{trait-based models}. In this approach we abandon the fixed trait
values for each PFT. But individuals or cohorts are assigned trait
values by sampling from \textbf{prescribed trait distributions}. For
example all the simulated individuals for a boreal forest are assigned
trait values from the available trait distributions for boreal forest in
the TRY database. Such an approach can and should account for the
trade-offs described in the LES and the PES to avoid the emergence of
individuals with unrealistic trait combinations. Figure \ref{fig:f717}
shows a conceptual example of an individual trait-based model. Due to
the principle of assigning trait values to individuals sampled from
trait distributions, traits are flexible in space and time. In this way
we approximate plant adaptation to environmental conditions rather
quickly. Compared to standard demograhic models with fixed PFTs, this
approach avoids a demographic timel lag in adaptation. These types of
models allow having a more realistic and diverse system compared to the
fixed PFT-base approaches. The example of Figure \ref{fig:f717} adopts a
few principles from genetics because the model assumes a community trait
pool and also mutation and crossover by creating different plant trait
combinations. The model simulates a seed bank that can sample from the
available trait combinations to define the new plants to grow. Such
models offer a lot of potential but are so far not yet not implemented
in operational global vegetation models.

\begin{figure}

{\centering \includegraphics[width=0.8\linewidth]{figures/chap7/f717_aDGVM} 

}

\caption{Conceptual modelling framework for a next-generation dynamic global vegetation model (DGVM). Individuals are characterized by their traits that influence their carbon (C) status and phenotype. All individuals at a site form the community, which influences resources, environmental conditions and disturbances via engineering and modulating impacts. These conditions interact to influence growth of the individuals. Individuals, through reproduction, can add their traits to the community trait pool. Crossover and mutation of the community trait pool yield the community seed bank. PDF, probability density function. (Sheiter et al. 2013)}\label{fig:f717}
\end{figure}

A third and new approach was suggested in the past two years and is
based on eco-evolutionairy principles (e.g.~Figure \ref{fig:f718}).
These models aim to be less depending on prescribed parameter values and
more on theoretical principles. These \textbf{eco-evolutionary models}
are still in the conceptual phase but offer a lot of potential. These
models try to optimize the processes instead of using prescribed values.
They assume different \textbf{eco-evolutionary} principles (Figure
\ref{fig:f719}). For ecophysiological processes they rely on the
\textbf{optimality principle} (as briefly touched upon in chapter 2). To
simulate the community composition they rely on the principle of
\textbf{natural selection}. And to simulate the optimal vegetation
structure they assume \textbf{self-organisation} as driving principle
(cfr. the perfect plasticity approximation in chapter 6). It is however
still an open question if a unifying theory exists for such models and
if enough data is available to make these models operational.

\begin{figure}

{\centering \includegraphics[width=0.8\linewidth]{figures/chap7/f718_berzaghi} 

}

\caption{Physio–demo–genetic (PDG) models integrate physiological, demographic, and evolutionary processes. They have been developed to better understand the interplay among plasticity and genetic adaptation and the effects of both processes on tree population dynamics under global change. The advantage of PDG models is their ability to account for the variability in functional traits due to both standing genetic variation and evolutionary change in response to changing local environmental conditions. This figure shows the conceptual framework of PDGmodels. PDGmodels couple: (i) a biophysical module to simulate carbon and water fluxes at the tree level using climate observations; (ii) a forest dynamics module to calculate demographic rates for adult trees (growth, mortality, and reproduction) based on carbohydrate reserves, and to simulate ecological processes across the life cycle; and, (iii) a quantitative genetics module relating genotype to the phenotype of one or more functional traits. (Berzaghi et al. 2019)}\label{fig:f718}
\end{figure}

\begin{figure}

{\centering \includegraphics[width=0.8\linewidth]{figures/chap7/f719_franklin} 

}

\caption{Framework for the use of organizing principles in vegetation modelling. The application of organizing principles (circles) helps predict (arrows) vegetation properties (boxes). Natural selection drives the evolution of species (or plant types) and their heritable functional traits, modelled as emergent evolutionairy stable strategies. Natural selection is also the reason that phenotypic plasticity in response to environmental variation is predictable through fitness-proxy maximization (optimality). At the community level, collective self-organization among many plants results in predictable patterns of spatial structure at the stand level (for example, due to plasticity of stem angles in the perfect plasticity approximation). Self-organization influences the biotic environment, which, together with the abiotic (external) environment, feeds back on plant reproduction and survival—that is, the natural selection of community composition. Many different community compositions may be possible, and the most likely can be predicted by MaxEnt. The external environment includes abiotic factors and all other external drivers, including disturbance regimes. (Franklin et al. 2020)}\label{fig:f719}
\end{figure}

Some condluding thoughts. Based on what we described I chapter 6 and 7,
we can cocluded that demographic models and trait based models in
principle should allow to simulate environmental filtering via simulated
feedbacks between traits -- productivity- demography- survival --
competition. And that the simulated vegetation community with such
models will largely impact the simulated biogeochemical cycles. Such
model are therefore able to simulated important feedbacks between
climate, vegetation dynamics and biogeochemistry, which is essential to
use vegetation models in a climate change context.

It is important to mention here that are mainly focusing on natural
vegetation in this chapter. Simulating crops and managed systems is on
one hand `simpler' than natural systems, because we don't have to
address the challenge of simulating diversity we studying crops or
monospecific forest stands. However these systems pose oter challenges
for example accounting for management practices, crop rotation,
fertilizer use, etc\ldots{}

\section{Case study 7.1}\label{case-study-7.1}

The study of Sakschewski et al. (2016) tests for the Amazon forest if
adding more functional diversity will increase the resilience of the
simulated ecosystem (in comparison the the models discussed in chapter 6
for the Amazon dieback). Here, they use the LPJml-FIT model, a
trait-based version of LPJ, which adds variability in plant proprieties
and is tested for future scenarios.

The high diversity model (Figure \ref{fig:f720}, blue) still simulates a
strong collapse of the Amazon forest in response to the climate
scenario, but also a strong recovery. This allows concluding that if we
have a more diverse system, the system can more easily recover after a
severe disturbance.

\begin{figure}

{\centering \includegraphics[width=0.8\linewidth]{figures/chap7/f720_lpjML_1} 

}

\caption{Simulated rainforest biomass under climate change and different plant trait diversity. Annual biomass over 800 simulation years for 400 ha of Ecuadorian rainforest (longitude: 77.75 W; latitude: 1.25 S) from three different versions of the vegetation model LPJmL under a severe climate change scenario (RCP 8.5 HadGEM2).DeltaT: annual temperature difference to the mean temperature of pre-impact time (1971–2000) in K. (Sakschewski et al. 2016)}\label{fig:f720}
\end{figure}

\begin{figure}

{\centering \includegraphics[width=0.6\linewidth]{figures/chap7/f721_lpjML_2} 

}

\caption{Forest height structure recovers with biomass. a, Mean biomass contribution of tree height classes for pre-, mid- and post-impact time. b, Visualization of model output showing 0.5 ha of the 400 ha of Ecuadorian rainforest in a selected year during pre-, mid-, and post-impact time, respectively (top to bottom). Different crown (stem) colours denote different SLA (WD) values of individual trees. Crown size, stem diameter and tree height are scaled by model output. Green squares indicate tree gaps covered by herbaceous plants. (Sakschewski et al. 2016)}\label{fig:f721}
\end{figure}

\section{Case study 7.2}\label{case-study-7.2}

This study present the first vegetation model that accounts for lianas
in tropical rainforest. The model allows includes a liana PFT allowing
to study the liana impact on the forest carbon cycle. The authors assess
the contribution of lianas to various components of carbon balance in
South America. Some key results:

\begin{itemize}
\tightlist
\item
  Figure \ref{fig:f722} shows in percent the relative contribution of
  lianas to above-ground biomass and photosynthesis, where it was found
  that their contribution to biomass is relatively low (1.1\%) but to
  photosynthesis is significant (8.9\%) in old-gowth rainforest.
\item
  Figure \ref{fig:f723} simulates the impact of lianas on forest size
  distribution. In long term simulations, lianas have a clear impact on
  forest demography.
\item
  Lianas are important in the first phases of succession (Figure
  \ref{fig:f724}) When estimate the role of lianas on the carbon cycle
  (Figure \ref{fig:f725}, it is clear that the impact is much larger in
  a young forest in Panama, compared to an old-growth forest in French
  Guiana.
\end{itemize}

\begin{figure}

{\centering \includegraphics[width=0.8\linewidth]{figures/chap7/f722_porcia1} 

}

\caption{Simulated liana contributions to forest carbon pools and fluxes for the Paracou site in French Guiana, with observed liana contributions in parentheses. B, biomass; GPP, gross primary productivity; NEE, net ecosystem exchange (negative values mean carbon uptake); NPP, net primary productivity; R, respiration. (di Porcia et al. 2019) }\label{fig:f722}
\end{figure}

\begin{figure}

{\centering \includegraphics[width=0.8\linewidth]{figures/chap7/f723_porcia2} 

}

\caption{Forest demographic composition for two simulated sites: young forest in Gigante, Panama (a–b–c), and old growth forest in Paracou, French Guiana (d–e–f). Panels (a) and (d) show a representative area of modeled forest of 1 ha. To visualize the forest composition, the forest is decomposed into patches according to their simulated relative area, and the three cohort densities and sizes are preserved (as well as the liana tree tracking). Panels (b–c) and (e–f) compare the basal area distributions of liana and tree PFTs, respectively, as observed locally (black) or simulated according by ED2 (shades of blue and green). Tree basal area values (panels c and f) are compared for the simulations with (solid bars) or without (hatched bars) lianas. Σ represents the total basal area according to the model (blue or green) and field observations (black). Error bars represent the standard deviation of the different plot measurements (smaller error bars correspond to more homogeneous plots). The K–Sstat is the test statistic of the two‐sample Kolmogorov–Smirnov test between the observed and simulated size distributions (with a sampling size of 250 for each distribution). Liana basal area in Gigante was the only case in which the observed and simulated distribution did not significantly differ. (di Porcia et al. 2019)}\label{fig:f723}
\end{figure}

\begin{figure}

{\centering \includegraphics[width=0.8\linewidth]{figures/chap7/f724_porcia3} 

}

\caption{Comparison of simulations of forest succession with (solid lines) and without (dashed lines) lianas. The upper graphs (a–c–e) show the aboveground biomass (AGB), while the bottom graphs (b–d–f) represent LAI as a function of time for one patch (a–d) and for the forest aggregate (b–c–e–f). The gray zones represent the period during which the model outputs were averaged for all other plots (corresponding to the approximate stand age of the forest sites). The increases in LAI are caused by the crossing of the reproductive thresholds for the different plant functional types (PFTs). (di Porcia et al. 2019)}\label{fig:f724}
\end{figure}

\begin{figure}

{\centering \includegraphics[width=0.8\linewidth]{figures/chap7/f725_porcia4} 

}

\caption{Relative changes in carbon pools and fluxes for Paracou, French Guiana (brown), and Gigante, Panama (yellow), upon inclusion of the liana plant functional type in the simulations (by comparing a simulation with and without the liana PFT). B, biomass; GPP, gross primary productivity; NPP, net primary productivity; R, respiration. (di Porcia et al. 2019)}\label{fig:f725}
\end{figure}

\part{Upscaling and
applications}\label{part-upscaling-and-applications}

\chapter{Spatial heterogeneity, landscape scale and
disturbance}\label{spatial-heterogeneity-landscape-scale-and-disturbance}

\chaptermark{Heterogeneity}

\section{Introduction: landscape
scale}\label{introduction-landscape-scale}

In this chapter we discuss how vegetation models can deal with spatial
heterogeneity at the landscape scale and how disturbance can be
accounted for. These elements are key when applying vegetation models to
\textbf{large spatial and temporal scales}. We first define a few
important landscape ecology terms that are relevant to have in mind when
applying vegetation models at the landscape scale.

\textbf{Landscape ecology} studies the spatial patterns and the impact
of ecological processes on spatial patterns and vice-versa. Landscape
ecologists try to understand the causes and consequences of
\textbf{spatial heterogeneity} across spatial scales. For such
processes, it is important to consider the \textbf{role of humans}
changing/disturbing the landscape patterns (deforestation, land
conversion). Spatial heterogeneity in the landscape depends on the
abiotic factors (e.g.~soil, topography, climate, etc\ldots{}) and biotic
heterogeneity (land cover, land use, natural disturbances), and
anthropogenic disturbances. In many ecosystems (natural) disturbance
helps to maintain the heterogeneity of the landscape and habitats. The
\textbf{scale and resolution} are very important aspects of landscape
ecology and landscape scale modelling (Fig. \ref{fig:f81}).
Heterogeneity is present at every scale. Scale and resolution need to be
carefully defined based on the research questions, the objective and the
available data.

\begin{figure}

{\centering \includegraphics[width=0.8\linewidth]{figures/chap8/f81_landscape} 

}

\caption{Illustration of the importance of scale in landscape ecology.}\label{fig:f81}
\end{figure}

In vegetation models, we typically divide the landscape in homogeneous
patches. Models can then be used to analyze the \textbf{patch dynamics}
(Fig. \ref{fig:f82}). In other words, we need to study the changes of
landscape in terms of interacting patches, models are used to analyze:

\begin{itemize}
\tightlist
\item
  Spatial patterns, e.g.~to study the distribution of PFTs or biomass
  over the patches in the landscape.
\item
  Changes within and between patches.
\item
  Interaction between patches (however, neglected by many vegetation
  models); there are many processes of interaction (e.g.~via fires, seed
  dispersal, \ldots{})
\item
  Metapopulations (populations of a species or PFT spread over multiple
  patches) and the related processes of migration, extinction or
  colonization of species.
\item
  Fragmentation, edge effects, and connectivity (these aspects are
  important and depend on our objective of research). If we model the
  carbon stock is important to consider these aspects (fragmentation).
  For conservation studies these elements are key (see course
  `natuurbehoud').
\end{itemize}

Most (global) vegetation models consider patch dynamics in a very
aggregated way and are not spatially explicit. However, a few specific
models focus on landscape dynamics explicitly. For example the PICUS
model. This model is spatially explicit and it is very useful for
landscape scale applications but is more difficult to apply on very
large scale because it needs detailed spatial input data on soil and
land cover.

\begin{figure}

{\centering \includegraphics[width=0.8\linewidth]{figures/chap8/f82_patch} 

}

\caption{Illustration of 3 situations in landscape dynamics represented by patches on 3 time steps: a landscape in equilibrium, a landscape with patch dyanmics, an averaged landscape in steady-state but with dyanmics in de individual patches.}\label{fig:f82}
\end{figure}

\section{Land use change}\label{land-use-change}

Land use change is a very important factor for vegetation modelling at
large spatial scales. In most vegetation models land use change is
accounted for, by using land use maps as input. The model user has the
choice to simulate a static landscape based on a single land use/cover
map or to have dynamic maps as model input (e.g.~a land use/cover map
that is updated every simulation year). Specific models exist to
simulate historical or future land use change scenarios.

There are two general categories of land-use change. They need to be
accounted for in a different way in vegetation models. The first
category is \textbf{extensification}, which is the expansion of land
area used by people. Within this category there are two sub-categogies:
(1) \textbf{Land-use conversion}, which is the change in dominant plant
functional type (PFT) (e.g.~forest is converted into grassland); (2)
land-use modification, which is a significant change in human impact
without change in PFT (e.g.~introducing cattle grazing in a savanna).
The second land use change category is \textbf{intensification}, which
is the increase of inputs (subsidies) per unit area (e.g.~intensifying
the grazing of the savanna).

Large scale vegetation models are particularly suited to study the
impact of land use change on various ecosystem and land surface
processes. For example, in the study of Akkermans et al. 2014 (see
lecture slides) a deforestation scenario for the Congo Basin was
analyzed for its impact on biophysical fluxes and biogeochemistry. In
this study a land surface model (CLM) was coupled to a regional climate
model (COSMO), allowing to study the feedback of deforestation in the
Congo Basin on the regional climate. A second example is the study of
Guimberteau et al. (2017) where 3 vegetation models were used to
evaluate various land use change scenarios in the Amazon basin. The 3
vegetation models had a river routing component to evaluate the impact
of land use change on hydrology and river discharge. It illustrates that
deforestation scenarios reduced the evapotranspiration in multiple
regions in the Amazon resulting in a significant change in the Amazon
river discharge rate and seasonality (especially the impact on the
discharge peaks can result in important flooding). This example
illustrates the power of vegetation models to use for land and river
management.

\section{Disturbance}\label{disturbance}

Natural or anthropogenic disturbance can have very large impact on
ecosystem functioning and should be accounted for when applying
vegetation models at large spatial scales. Disturbances can be gradual
(e.g.~climate change) but are very often event-based and have a
stochastic nature. There are different types of disturbances: wind,
fire, management, drought, herbivory, \ldots{} and each of them has
different characteristics and processes through which they impact
ecosystem functioning. When modelling disturbance impacts it is
important to consider the properties of the disturbance :

\begin{itemize}
\tightlist
\item
  Intensity (e.g.~how intense is the forest fire, how intense is the
  thinning)
\item
  Severity (e.g.~how much biomass was affected)
\item
  Frequency (e.g.~how many fires per year in a region)
\item
  Size (e.g.~fraction of the landscape impacted by an insect outbreak)
\item
  Timing (e.g.~fire in the wet or dry season will have a very different
  impact)
\end{itemize}

These properties are of course interrelated, for example the disturbance
severity (\% of organic material removed) depends on the type of
disturbance (Fig \ref{fig:f83}). For example, herbivory has in many
cases a low impact while that agricultural clearing always has a larger
impact.

\begin{figure}

{\centering \includegraphics[width=0.8\linewidth]{figures/chap8/f83_disturbance_chapin} 

}

\caption{Spectrum of disturbance severity associated with major types of disturbance, ranging from normal steady-state functioning of ecosystems to primary succession. (Chapin 2012)}\label{fig:f83}
\end{figure}

Vegetation models are often used to study the effect of disturbance. We
know that the properties of the system and the disturbance will
determine the influence the disturbance will have. In many current
studies the `resilience' of ecosystems is highlighted. It is however
important to define multiple aspects of the influence of disturbance. In
the first place there is the \textbf{resistance} of an ecosystem, which
is its tendency not to change (example some forests are resistant to
moderate fires). The \textbf{response} of the ecosystem is the magnitude
of the change due to a perturbation. The \textbf{resilience} is defined
as the rate of return of the ecosystem to its original state. While the
\textbf{recovery} is the extent of return to the original state (Fig.
\ref{fig:f84}). In the sections below, we will highlight 3 types of
disturbance and how they are implemented in vegetation models: fire,
herbivory and forest management.

\begin{figure}

{\centering \includegraphics[width=0.8\linewidth]{figures/chap8/f84_disturbance2_chapin} 

}

\caption{Properties of a system that influence its probability of changing state. The solid ball represents the state of the system after a perturbation. The open ball shows the most likely future states of the system. A, A system shows a small response to a perturbation if the perturbation is small or the system is resistant to change.B, After a perturbation, a system can assume many possible states; if it is highly resilient, it may return quickly to its original state; if it is less resilient, or if the perturbation is large, the system may move to a new state.(Chapin 2002)}\label{fig:f84}
\end{figure}

\subsection{Fire}\label{fire}

Modeling of wildfires is a very specific branch of ecological modelling.
It is beyond the scope of this course to go into the technical details.
Therefore, we will highlight the key elements needed for fire modeling
based on an example of the LPJ model. LPJ was one of the first
vegetation models that included the impact of fire in a large scale
vegetation model (Thonicke et al. 2001). The model assumed litter
moisture as the main driver of the day-to-day vegetation fire
probability (if the litter is dry, the fire probability is high). To
estimate the burnt fractional area per year, they used an empirical
relation with the length of the fire season. And finally, the model
included a PFT parameter for fire resistance (e.g.~high resistance for
tropical evergreen trees and low resistance for grass PFT). This initial
model was able to simulate quite well large-scale global patterns of
fire occurrence and intervals.

In the decade that followed, multiple fully-developed fire models
emerged. The LPJ-spitfire model is one of them (Thonicke et al. 2010).
Such models account for many processes and interactions between
vegetation and climate, as illustrated by Fig. \ref{fig:f85}, and can
simulate detailed spatial patterns (Fig. \ref{fig:f86}). A process-based
fire model accounts for a multitude of inputs that influence wildfires:
climate, soil, PFT composition, vegetation structure and fuel (dead
organic matter). The fire model itself typically consists of 3 parts:

\begin{itemize}
\tightlist
\item
  A module simulating the \textbf{ignition} (probability) or fire
  occurrence
\item
  A module simulating the \textbf{fire spread}
\item
  A model simulating the \textbf{fire effects/impact} on the vegetation
\end{itemize}

Each of those modules depends on multiple input factors (Fig.
\ref{fig:f85}). the outputs of such a fire model are the carbon emission
by the biomass burning, other trace gas emission, the fire impact on the
PFT composition, carbon pools and stand structure. Other fire models
have a very similar overall structure (e.g.~Fig. \ref{fig:f87}) but
differ in the details of the equations, parameter values and inputs
accounted for. The burnt area is often simulated as an elliptical area
starting from the point of ignition into the direction of the wind (Fig.
\ref{fig:f88}). Fire models typically rely on empirical relationships
relating fire occurrence to various driving variables (Fig.
\ref{fig:f89}).

\begin{figure}

{\centering \includegraphics[width=0.8\linewidth]{figures/chap8/f85_spitfire} 

}

\caption{Scheme describing model features a process-based fire model for dynamic vegetation models or climate-vegetation models should consider. (Thonicke et al., 2010).}\label{fig:f85}
\end{figure}

\begin{figure}

{\centering \includegraphics[width=0.8\linewidth]{figures/chap8/f86_spitfire_output} 

}

\caption{Simulation results of the SPITFIRE model: (a) fire danger index, (b) number of fires, (c) fractional area burnt (all as annual averages for 1982–1999). (Thonicke et al. 2010).}\label{fig:f86}
\end{figure}

\begin{figure}

{\centering \includegraphics[width=0.8\linewidth]{figures/chap8/f87_CLM_fire} 

}

\caption{Structure of the fire parameterization developed for CLM by Li et al. 2012. Text boxes in yellow, red, and blue colors represent three parts in the fire module: fire occurrence, fire spread, and fire impact.}\label{fig:f87}
\end{figure}

\begin{figure}

{\centering \includegraphics[width=0.8\linewidth]{figures/chap8/f88_fire_ellips} 

}

\caption{Conceptual elliptical fire shape that is used to estimate the burned area with the wind direction along the major axis and the point of ignition at one of the foci in the CLM model. (Li et al. 2012)}\label{fig:f88}
\end{figure}

\begin{figure}

{\centering \includegraphics[width=0.8\linewidth]{figures/chap8/f89_CLM_fire_relations} 

}

\caption{Dependence of fire occurrence on (a) fuel availability fb, (b) relative humidity fRH, and (c) soil wetness f(theta) in the CLM model. (Li et al. 2012)}\label{fig:f89}
\end{figure}

\subsection{Herbivory (Case study 8.1)}\label{herbivory-case-study-8.1}

\textbf{Kurz, W. A., Dymond, C. C., Stinson, G., Rampley, G. J.,
Neilson, E. T., Carroll, A. L., \ldots{} \& Safranyik, L. (2008).
Mountain pine beetle and forest carbon feedback to climate change.
Nature, 452(7190), 987-990.}

We illustrate the modelling of herbivory by a study on the impact of an
enormous outbreak of the Mountain pine beetle, in British Columbia (CA)
in 2006 (Fig. \ref{fig:f810}). The study evaluated the impact of the
outbreak on tree mortality and the carbon cycle. These impacts were
estimated using the CBM-CFS3 model, which is an empirical data driven
forest model. The results (Fig. \ref{fig:f811} and \ref{fig:f812})
indicate a very large long-term impact of the carbon stock of 279
MegaTons of carbon, turning the affected area into a large source of
carbon. The study concluded that climate change contributed to the
extent of the outbreak.

\textbackslash{}begin\{figure\}

\{\centering \includegraphics[width=0.8\linewidth]{figures/chap8/f810_Kurz1}

\}

\textbackslash{}caption\{Geographic extent of mountain pine beetle
outbreak in North America. a, Extent (dark red) of mountain pine beetle.
b, The study area includes 98\% of the 2006 outbreak area. c, A
photograph taken in 2006 showing an example of recent mortality: pine
trees turn red in the first year after beetle kill, and grey in
subsequent years. (Kurz et al. 2008) Photo credit: Joan Westfall,
Entopath Management Ltd.\}\label{fig:f810} \textbackslash{}end\{figure\}

\begin{figure}

{\centering \includegraphics[width=0.8\linewidth]{figures/chap8/f811_Kurz2} 

}

\caption{Area infested with the mountain pine beetle during the simulation period. Statistics are used from 2000 to 2006 and projections are used from 2007 to 2020. a, Percentiles describing the parameter space resulting from 100 Monte Carlo projections of the beetle-infested area. Our projections of decreasing area after 2009 are largely based on the decline in available live host (pine) area. b, Area statistics for the single Monte Carlo simulation used for scenario analysis broken down by host mortality class. (Kurz et al. 2008)}\label{fig:f811}
\end{figure}

\begin{figure}

{\centering \includegraphics[width=0.8\linewidth]{figures/chap8/f812_Kurz3} 

}

\caption{Total ecosystem carbon stock change for three scenarios. The control simulation was run with no beetle outbreak, and with base harvest and fires. The beetle simulation added insect impacts to the control scenario. The additional harvest simulation added the management response of increased harvest levels from 2006 to 2016 to the beetle simulation. Negative ecosystem carbon stock change values represent fluxes from the forest to the atmosphere (net source of carbon). The source in 2003 was, in part, the result of the large area burned (2,440km2 in the study area) that was included in all three scenarios.(Kurz et al. 2008)}\label{fig:f812}
\end{figure}

\subsection{Forest management}\label{forest-management}

Demographic models that track individual trees or the number of trees in
cohorts allow to study the impact of forest management. Management
practices like thinning can be simulated by removing individuals in
specific size classes, clear cuts can be simulated by changing the PFT
in specific patches with a predefined rotation period, and species
conversion can be simulated by introducing additional PFTs (species) in
patches. These models allow then to simulate the impact of such
management practices on various model variables (productivity,
hydrology, etc\ldots{}). The LPJ-guess model is one of these models
where this is now possible (Fig. \ref{fig:f813}, \ref{fig:f814},
\ref{fig:f815}).

\begin{figure}

{\centering \includegraphics[width=0.8\linewidth]{figures/chap8/f813_LPJ_manag1} 

}

\caption{Forest management in LPJguess (Lindeskog et al. 2021). Examples of age structure setup at three different structural levels, patch, stand and stand type. Beech monocultures are created from clearcut of potential natural vegetation. The target in year 2000 was three cohorts of 100, 67 and 33 years. (a) Within-patch. One secondary stand with 1 patch created in 1901. Thinnings in 1933 and 1967. Age structure depends on timing of increased light 150 and subsequent reestablishment of seedlings. (b) Among-patch. One secondary stand with 3 patches created in 1901. Clearcut in patches 2 and 3 in 1933 and 1967 (evenly spread age distribution). (c) Among-stand. Three secondary stands with 1 patch created in 1901, 1933 and 1967. Age structure from area fraction input.}\label{fig:f813}
\end{figure}

\begin{figure}

{\centering \includegraphics[width=0.8\linewidth]{figures/chap8/f814_LPJmanag2} 

}

\caption{Example of forest management change in LPJ-GUESS (Lindeskog et al. 2021). Spruce monoculture changed to mixed broadleaved, both with automated thinning and clearcut. Management change is activated after first management has completed by a clearcut event.}\label{fig:f814}
\end{figure}

\begin{figure}

{\centering \includegraphics[width=0.8\linewidth]{figures/chap8/f815_LPJ_manag_result} 

}

\caption{Simulated European forests with LPJguess including forest management (Lindeskog et al. 2021): (a) total carbon pool 2010 in a simulation with thinning, (b) total carbon pool 2010 difference between simulations with and without wood harvest in regrowth forest, (c) Mean 2001-2010 NEE (net ecosystem exchange) in a simulation with thinning, (d) Mean 2001-2010 NEE difference between simulations with and without thinning.}\label{fig:f815}
\end{figure}

\subsection{Case study 8.2}\label{case-study-8.2}

\textbf{Naudts, K., Chen, Y., McGrath, M. J., Ryder, J., Valade, A.,
Otto, J., \& Luyssaert, S. (2016). Europe's forest management did not
mitigate climate warming. Science, 351(6273), 597-600.}

In this case study, we zoom in on a study by Naudts et al. (2016) that
evaluated the impact of the historical European forest management since
1750 on climate change. They used the ORCHIDEE-CAN model coupled to an
atmosphere model to evaluate the effects of European forest management
on climate via biogeochemical and biophysical effects and feedbacks.
Since 1750 forest management changed a lot in Europe, with coniferous
PFTs becoming more important and high stands replacing unmanaged or
coppice forest (Fig \ref{fig:f816}). The model simulations showed that
2.5 centuries of forests management in Europe have not cooled the
climate. They show that wood extraction caused a loss 3.1 Pg C and that
species conversion caused a temperature increase of 0.12 Kelvin (Fig.
\ref{fig:f817}), through albedo and energy balance effects.

\begin{figure}

{\centering \includegraphics[width=0.8\linewidth]{figures/chap8/f816_naudts1} 

}

\caption{Main changes in European forest management between 1750 and 2010. (A) Relative distribution (percent) of tree growth forms in 1750 and (B) 2010. Total forest area in 1750 was 1,929,000 km2 and increased to 2,126,000 km2 by 2010. (C) Relative distribution (percent) of wood extraction strategies in 1750 and (D) 2010. (Naudts et al. 2016)}\label{fig:f816}
\end{figure}

\begin{figure}

{\centering \includegraphics[width=0.8\linewidth]{figures/chap8/f817_naudts2} 

}

\caption{Effects of species conversion in Europe since 1750 as simulated by the ORCHIDEE-CAN model coupled to the LMDZ atmospheric model (Naudts et al. 2016). Temperature changes are for boundary layer temperature during summer (kelvin). (A) Temperature change due to changes in emissivity (DTa,ea) caused by species conversion, (B) changes in albedo (Da) due to species conversion, (C) total temperature change (DTa) due to species conversion, and (D) correlation between species-induced and land use–induced temperature change. In (C), black dots denote significant temperature changes at the 0.05 significance level, as determined by a modified paired one-sample t test.}\label{fig:f817}
\end{figure}

\chapter{Upscaling from the leaf to the
globe}\label{upscaling-from-the-leaf-to-the-globe}

\chaptermark{Globe}

\section{The problem of upscaling}\label{the-problem-of-upscaling}

\begin{figure}

{\centering \includegraphics[width=0.8\linewidth]{figures/chap9/f91_lucht} 

}

\caption{Illustration of the correlation between the spatial and temporal scales at which various vegetation processes dominate. (Wolfgang Lucht, PIK)}\label{fig:f91}
\end{figure}

\begin{figure}

{\centering \includegraphics[width=0.8\linewidth]{figures/chap9/f92_upscaling} 

}

\caption{Illustration from the course Inventory of Forest and Nature showing the temporal and spatial scale of different carbon cycle monitoring methods and the integrative power of vegetation models.}\label{fig:f92}
\end{figure}

\section{Vegetation models as part of earth system
models}\label{vegetation-models-as-part-of-earth-system-models}

\begin{figure}

{\centering \includegraphics[width=0.8\linewidth]{figures/chap9/f93_ESM_bonan} 

}

\caption{Scientific scope of (a) climate models and (b) Earth system models. Climate models simulate biogeophysical fluxes of energy, water, and momentum on land and also the hydrologic cycle. Terrestrial and marine biogeochemical cycles are new processes in Earth system models. The terrestrial carbon cycle includes carbon uptake through gross primary production (GPP) and carbon loss from autotrophic respiration, heterotrophic respiration, and wildfire. Many models also include the nitrogen cycle. Anthropogenic land use and land-cover change are additional processes. The fluxes of CO2, CH4, aerosols, biogenic volatile organic compounds (BVOCs), and wildfire chemical emissions are passed to the atmosphere to simulate atmospheric chemistry and composition. Nitrogen is carried in freshwater runoff to the ocean.(Bonan)}\label{fig:f93}
\end{figure}

\begin{figure}

{\centering \includegraphics[width=0.8\linewidth]{figures/chap9/f94_ESM_GCM} 

}

\caption{Illustration of the difference between climate models and earth system models. Climate models include the blue processes, earth system models include both the blue and green processes.}\label{fig:f94}
\end{figure}

\begin{figure}

{\centering \includegraphics[width=0.8\linewidth]{figures/chap9/f95_GCM_timeline} 

}

\caption{Illustration of the evolution of climate models into earth system models, with more and more processes and components added through time over the past decades.}\label{fig:f95}
\end{figure}

\section{MOdelling work flow}\label{modelling-work-flow}

\begin{figure}

{\centering \includegraphics[width=0.8\linewidth]{figures/chap9/f96_willimas} 

}

\caption{Model data fusion in every step of the model development cycle. (Williams et al. 2009)}\label{fig:f96}
\end{figure}

\begin{figure}

{\centering \includegraphics[width=0.8\linewidth]{figures/chap9/f97_dietze} 

}

\caption{Methodological workflow of model data fusion. (Dietze: Ecological Forecasting)}\label{fig:f97}
\end{figure}

\section{Case study 9.1}\label{case-study-9.1}

\begin{figure}

{\centering \includegraphics[width=0.8\linewidth]{figures/chap9/f98_humphrey1} 

}

\caption{Carbon fluxes in CTL and experiment A. a, IAV (inter annual variability) in global mean NBP as simulated by four ESMs (CCSM4, ECHAM6, GFDL and IPSL) in coupled model experiments with (CTL) and without (experiment A; ExpA) anomalies in soil moisture. Positive NBP indicates carbon uptake. b, Standard deviations of global mean NBP, GPP and respiration and disturbance (ReD) in the two experiments. c, Drivers of change in global mean NBP variance. Global mean NBP variance decreases in the experiment with prescribed seasonal soil moisture mainly because GPP variance is reduced. GPP and ReD fluxes are not available for the IPSL model. (Humphrey et al. 2021)}\label{fig:f98}
\end{figure}

\begin{figure}

{\centering \includegraphics[width=0.8\linewidth]{figures/chap9/f99_humprey2} 

}

\caption{Drivers of NBP IAV. a, b, Contribution of meteorological drivers to NBP IAV: direct soil moisture effects (NBPSM), indirect LAC-dependent (land atmosphere coupling) temperature and VPD effects (NBPLAC T\&VPD), non-LAC-dependent temperature and VPD effects (NBPnonLAC T\&VPD ) and radiation (R) effects (NBPR) globally (a; mean of the four models ±1σ) and from local to global scales (b). (Humphrey et al. 2021)}\label{fig:f99}
\end{figure}

\section{Case study 9.2}\label{case-study-9.2}

\begin{figure}

{\centering \includegraphics[width=0.8\linewidth]{figures/chap9/f910_sitch1} 

}

\caption{Validation of LPJ model at the site level. Comparison between simulated and observed Net Ecosystem Exchange at 6 fluxtower sites in Europe. (Sitch et al. 2003) }\label{fig:f910}
\end{figure}

\begin{figure}

{\centering \includegraphics[width=0.8\linewidth]{figures/chap9/f911_sitch2} 

}

\caption{Validation of the LPJ model at the site level. Observed vs. simulated soil moisture at seven sites.(Sitch et al. 2003)}\label{fig:f911}
\end{figure}

\begin{figure}

{\centering \includegraphics[width=0.8\linewidth]{figures/chap9/f912_sitch3} 

}

\caption{Validation of LPJ at the global scale, comparison with observed annual runoff per latitude. (Sitch et al. 2003)}\label{fig:f912}
\end{figure}

\begin{figure}

{\centering \includegraphics[width=0.8\linewidth]{figures/chap9/f913_sitch4} 

}

\caption{Global coparison of LPJ simulated distributions of woody vegetation with sattelite based maps. (Sitch et al. 2003)}\label{fig:f913}
\end{figure}

\begin{figure}

{\centering \includegraphics[width=0.8\linewidth]{figures/chap9/f914_sitch5} 

}

\caption{LPJ simulated vs. reconstructed interannual variation in net ecosystem exchange for the globe. (Sitch et al. 2003)}\label{fig:f914}
\end{figure}

\chapter{Model simulations, projections and scenario
analysis}\label{model-simulations-projections-and-scenario-analysis}

\chaptermark{Projections}

\section{Simulation setup}\label{simulation-setup}

\begin{figure}

{\centering \includegraphics[width=0.8\linewidth]{figures/chap10/f10_1_simu} 

}

\caption{Sequention of the different parts of a simulaion setup: the model spinup, the historical run, and the scenarion run.}\label{fig:f101}
\end{figure}

\begin{figure}

{\centering \includegraphics[width=0.8\linewidth]{figures/chap10/f10_2_ECMWF} 

}

\caption{The principle of the construction of a historical global climate reanalysis dataset. Global observations are combined with a climate model to obtain a complete gridded consistent global dataset. (ECMWF) }\label{fig:f102}
\end{figure}

\begin{figure}

{\centering \includegraphics[width=0.8\linewidth]{figures/chap10/f10_3_LPJguess_spinup} 

}

\caption{Example of a model spinup for teh LPJguess educational model. PFTs are initialized during the spinup, before a scenario run is performed.}\label{fig:f103}
\end{figure}

\begin{figure}

{\centering \includegraphics[width=0.8\linewidth]{figures/chap10/f10_4_CLM_spinup} 

}

\caption{Pre indusrial (pre 1850) spinup run for the CLM 5.0 model, showing various model state variables reaching equilibrium.}\label{fig:f104}
\end{figure}

\section{Model experiments}\label{model-experiments}

\section{Paleo studies}\label{paleo-studies}

\begin{figure}

{\centering \includegraphics[width=0.8\linewidth]{figures/chap10/f10_5_paleo} 

}

\caption{a) Simulated mid-Holocene biome distribution in  different models (ESMs) based on the PFT method, b) pollen-based biome reconstructions of the mid-Holocene biome distribution (BIOME6000 database)and c) the best neighbour score (BNS) for all individual sites showing the agreement of the reconstructed biomes and the biome distribution in the neighbourhood of the sites, ranging from 0 (no grid cell in the surrounding area shows the same biome as reconstructed) to 1 (the grid cell locating the site and the record at the site indicate the samebiome). (Dallmeyer et al. 2019) }\label{fig:f105}
\end{figure}

\section{Climate scenarios}\label{climate-scenarios}

\begin{figure}

{\centering \includegraphics[width=0.8\linewidth]{figures/chap10/f10_6_RCPs} 

}

\caption{Atmospheric CO2 concentration for the 21st century accoring to the IPCC Representative Concentration Pathways. (IPCC)}\label{fig:f106}
\end{figure}

\begin{figure}

{\centering \includegraphics[width=0.8\linewidth]{figures/chap10/f10_7_huntzinger} 

}

\caption{Long-term mean summer (June, July, August) net ecosystem productivity by model (2000–2005), as simulated by the models participating to the Model Intercomparison Project of the North American Carbon Program (NACP-MIP). A positive sign indicates net terrestrial carbon uptake from the atmosphere, while a negative sign signifies net carbon release to the atmosphere. Prognostic models are shown above with a green background; diagnostic models are below with a purple background. (Huntzinger et al. 2012)}\label{fig:f107}
\end{figure}

\begin{figure}

{\centering \includegraphics[width=0.8\linewidth]{figures/chap10/f10_8_Sitch_ensemble} 

}

\caption{Model Ensemble of 5 DGVMs, run for multiple climate SRES climate scenarios as simulated by the HAdGCM global climate model. Global mean land climatology (temperature, 1C, red; precipitation, mm/yr, blue),atmospheric CO2 content (black) and simulated land–atmosphere exchange over the 20th century by HyLand (HYL, black), Lund–Potsdam–Jena (LPJ, yellow), ORCHIDEE (ORC, blue), Sheffield (SHE, green), and TRIFFID (TRI, red). Red and blue dashes represent periods of strong El Nino (red) and La Nina (blue), respectively. Linear regressions are also plotted through the temperature and precipitation data. (Sitch et al. 2008) }\label{fig:f108}
\end{figure}

\begin{figure}

{\centering \includegraphics[width=0.8\linewidth]{figures/chap10/f10_9_sitch_ensemble} 

}

\caption{Model Ensemble of 5 DGVMs, here compared for 1 SRES climate scenario as simulated by the HAdGCM global climate model. Change in land carbon storage (TotC) and component vegetation (CV) and soils (CS) carbon stocks between 1860 and 2099 from the coupled climate-carbon cycle simulation under Special Report Emission Scenarios (SRES) emission scenario A1F1 (units are Pg C) for HyLand (HYL), Lund–Potsdam–Jena (LPJ), ORCHIDEE (ORC), Sheffield (SHE) and TRIFFID (TRI). (Sitch et al. 2008)}\label{fig:f109}
\end{figure}

\begin{figure}

{\centering \includegraphics[width=0.8\linewidth]{figures/chap10/f10_10_sitch_ensemble} 

}

\caption{Model Ensemble of 5 DGVMs, here th eresponse of the 5 models in compared for 2 SRES climate scenarios as simulated by the HAdGCM global climate model. Change in land carbon uptake, PgCyr-1, (top panels) relative to the present day (mean 1980–1999) for five Dynamic Global Vegetation Models (DGVMs) from coupled climate-carbon cycle simulations with two Special Report Emission Scenarios (SRES) emission scenarios, A1FI (solid lines), B1 (dashed lines), bracketing the range in emissions. Change in global vegetation (middle panels) and soil carbon (top panels), Pg C, between 2100 and 2000 under scenarios A1FI (solid lines) and B1 (dashed lines) for HyLand (HYL,black), Lund–Potsdam–Jena (LPJ, yellow), ORCHIDEE (ORC, blue), Sheffield (SHE, green), and TRIFFID (TRI, red). Note: only LPJ is run with interannual variations in climate. (Sitch et al. 2008)}\label{fig:f1010}
\end{figure}

\textbackslash{}begin\{figure\}

\{\centering \includegraphics[width=0.8\linewidth]{figures/chap10/f10_11_huntingford}

\}

\textbackslash{}caption\{Tropical forest biomass predictions for the
Americas (a), Africa (b) and Asia (c) by the MOSES--TRIFFID model forced
by 22 climate models. Climate models emulated are colour-coded, from
dark blue to dark red for decreasing year 2100 values of Cv. Grey
regions and squares are committed Cv values with climate constant at
year 2100 values, and small dashes link back to the same model in
transient predictions. Committed equilibrium values are
year-independent, hence the x-axis break (small vertical bars).
Normalized estimates of Cv from inventory data (2.5\%, mean and 97.5\%
confidence levels) are the short black curves for Americas and Africa.
Horizontal lines (large dashes) are estimated pre-industrial values,
year 1860 (Huntingford et al. 2013)\}\label{fig:f1011}
\textbackslash{}end\{figure\}

\begin{figure}

{\centering \includegraphics[width=0.8\linewidth]{figures/chap10/f10_12_zaehle} 

}

\caption{Comparing ESMs for multiple RCPs. Cumulative C sequestration from a few CMIP5 models and plausible range of C sequestration considering N constraints (CMIP5-N) for the HadGEM2-ES, IPSL-CM5A-LR, MPIESM-LR, and CESM1(BGC) models.(Zaehle et al. 2015)}\label{fig:f1012}
\end{figure}

\section{Land use scenarios}\label{land-use-scenarios}

\begin{figure}

{\centering \includegraphics[width=0.8\linewidth]{figures/chap10/f10_13_LU_inpe} 

}

\caption{Historical land use scenario for the Amazon (INPE)}\label{fig:f1013}
\end{figure}

\section{Management scenarios}\label{management-scenarios}

\begin{figure}

{\centering \includegraphics[width=0.8\linewidth]{figures/chap10/f10_14_restoration} 

}

\caption{ Simulating th eimpact of restoration on hydrology. Predicted total runoff from a study area of semi arid forests in SW-USA by the LANDIS II model, normalized by area, from 1990 to 2110 under future vegetation distributions and restoration rates. Precipitation inputs are not adjusted to account for the effects of climate change. The top row of panels shows runoff for a median (50th percentile) annual precipitation and the bottom row of panels shows runoff for a 10-yr drought (10th percentile) annual precipitation scenario. Values are means and SD across model runs. (Donnell et al. 2018)}\label{fig:f1014}
\end{figure}

\begin{figure}

{\centering \includegraphics[width=0.8\linewidth]{figures/chap10/f10_15_luyssaert} 

}

\caption{ Setup of simulation experiments in a study on European forest management scenarios.The experiments are described in publication of Lyussaert et al. 2018.Simulations with the ORCHIDEE-CAN vegetation model are shown in black and simulations with LMDzORCAN (couple climate model) are shown in red. Blue boxes denote intermediate calculations using the simulation results.The simulations shown in this figure correspond to runs with reduced air temperature (BBESTT2M), maximized surface albedo (BESTALBEDO), minimized surface albedo (BWORSTALBEDO), maximized carbon sink (BBESTLCA), minimized carbon sink (BWORSTLCA) and business as usual (BWAC). BWAC, BWAC-P1 and BWAC-P2 were used to calculate the minimal model noise. (Luyssaert et al. 2018)}\label{fig:f1015}
\end{figure}

\part{Appendix}\label{part-appendix}

\chapter*{Contributing to this
document}\label{contributing-to-this-document}
\addcontentsline{toc}{chapter}{Contributing to this document}

\section*{First steps}\label{first-steps}
\addcontentsline{toc}{section}{First steps}

First, visit the course webpage on
\url{https://github.com/femeunier/VegMod_course}, and fork it to your
own github account. Open a RStudio session and (if it is your first time
with git) introduce yourself:

\begin{Shaded}
\begin{Highlighting}[]
\FunctionTok{git}\NormalTok{ config --global user.name }\StringTok{"FULLNAME"}
\FunctionTok{git}\NormalTok{ config --global user.email you@yourdomain.example.com}
\end{Highlighting}
\end{Shaded}

Note that you can do every single step below using the terminal and the
git tabs in RStudio. Clone the newly forked folder to your local
machine:

\begin{Shaded}
\begin{Highlighting}[]
\FunctionTok{git}\NormalTok{ clone https://github.com/femeunier/VegMod_course.git}
\end{Highlighting}
\end{Shaded}

or using SSH (to set up it first, see for instance
\url{https://help.github.com/en/github/authenticating-to-github/connecting-to-github-with-ssh})

\begin{Shaded}
\begin{Highlighting}[]
\FunctionTok{git}\NormalTok{ clone git@github.com:femeunier/VegMod_course.git}
\end{Highlighting}
\end{Shaded}

Define upstream

\begin{Shaded}
\begin{Highlighting}[]
\BuiltInTok{cd}\NormalTok{ VegMod_course}
\FunctionTok{git}\NormalTok{ remote add upstream git@github.com:femeunier/VegMod_course.git}
\end{Highlighting}
\end{Shaded}

\section*{New pull request}\label{new-pull-request}
\addcontentsline{toc}{section}{New pull request}

Get the latest code from the main repository

\begin{Shaded}
\begin{Highlighting}[]
\FunctionTok{git}\NormalTok{ pull upstream master}
\end{Highlighting}
\end{Shaded}

Create a new branch (here new\_branch is the new branch's name)

\begin{Shaded}
\begin{Highlighting}[]
\FunctionTok{git}\NormalTok{ checkout -b new_branch}
\end{Highlighting}
\end{Shaded}

Do some coding, add files and commit them

\begin{Shaded}
\begin{Highlighting}[]
\FunctionTok{git}\NormalTok{ add filepath}
\FunctionTok{git}\NormalTok{ commit -m “Message”}
\end{Highlighting}
\end{Shaded}

Push your changes to your github (when a feature is working, a set of
bugs are fixed, or you need to share progress with others).

\begin{Shaded}
\begin{Highlighting}[]
\FunctionTok{git}\NormalTok{ push origin new_branch}
\end{Highlighting}
\end{Shaded}

Before submitting code back to the main repository, make sure that book
compiles (buikd book). Open the PR online by visiting your github
repository. To ease those previous steps you can take advantage of the
git GUI in RStudio. To do so, create a new project from an existing
directory.

\chapter*{Supporting material}\label{supporting-material}
\addcontentsline{toc}{chapter}{Supporting material}

Crash course, basic programming (R), theory about model evaluation etc.

\part{Practicals}\label{part-practicals}

\chapter*{Practical A}\label{practical-a}
\addcontentsline{toc}{chapter}{Practical A}

PC-room, supervised exercise

Simple model on diurnal variation in solar angle, radiation extinction
and photosynthesis in vegetation types with different and canopy
structure and LAI: grassland, broadleaved forest, coniferous forest

Scale: aggregated stand level (big leaf model)

Methodological focus: model formulation: translating a few equations
into code

Methodological focus: compiling code, running model, reading
input-output

\chapter*{Practical B}\label{practical-b}
\addcontentsline{toc}{chapter}{Practical B}

Group work, report, PC room

Modelling diurnal cycle of carbon and water fluxes for flux tower sites
(Savanna's Sahel)

Scale: aggregated stand level

Methodological focus: model-data comparison (goodness-of-fit), simple
parameter optimisation

\chapter*{Practical C}\label{practical-c}
\addcontentsline{toc}{chapter}{Practical C}

PC-room, supervised exercise

Modelling the size structure of a temperate forest (stand diameter
distribution)

Scale: forest stand

Methodological focus: initial conditions

\chapter*{Practical D}\label{practical-d}
\addcontentsline{toc}{chapter}{Practical D}

Group work, report, PC room

Modelling carbon stocks (above and belowground) and fluxes

Scale: ecosystem

Methodological focus: Spinup and sensitivity analysis (testing which
climate variables have strongest impact on stocks)

\chapter*{Practical E}\label{practical-e}
\addcontentsline{toc}{chapter}{Practical E}

PC-room, supervised exercise

Simulating forest succession, meta-analysis of trait dataset to
prescribe vegetation functional composition (using PEcAn-framework)

Scale: landscape

Methodological focus: parameter meta-analysis (PFT construction), data
assimilation

\chapter*{Practical F}\label{practical-f}
\addcontentsline{toc}{chapter}{Practical F}

PC-room, group work, microteaching

Climate/land use/management scenario analysis

Scale: site/globe? (Pecan framework) each group choses a question and a
model

Methodological focus: sensitivity and uncertainty analysis

\bibliography{book.bib,packages.bib}

\end{document}
